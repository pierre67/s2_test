\startcomponent c_80_maintenance_s2_120-fr


\startchapter [title={Maintenance et entretien},
							reference={chap:maintenance},
							]

\setups[pagestyle:marginless]

\startregister[index][maintenance:s2]{maintenance+instructions}

\startsection[title={Instructions générales}]


\subsection{Protection de l’environnement}

\starttextbackground [FC]
\setupparagraphs [PictPar][1][width=2.45em,inner=\hfill]

\startPictPar
\Penvironment
\PictPar
\Boschung\ pratique\index{protection de l’environnement} une politique globale de protection
de l’environnement qui cherche à éliminer les causes de pollution et qui englobe dans les décisions
de l’entreprise tous les effets des processus de production et du produit sur l’environnement.

Les objectifs poursuivis sont une utilisation parcimonieuse des ressources et une attitude responsable
par rapport aux fondements naturels de la vie, leur sauvegarde servant autant l’homme que la nature.
L’application de certaines règles lors de l’utilisation du moteur permet de contribuer à la protection
de l’environnement. Cela comprend l’utilisation de matériaux lors de la maintenance du véhicule dans
le respect des règles de protection de l’environnement (par exemple, l’élimination des produits
chimiques et déchets dangereux).

La consommation de carburant et l’usure du moteur dépendent des conditions d’utilisation.
C’est pourquoi nous vous prions de prendre en considération les points cités ci-dessous.

\startitemize
\item Ne faites pas chauffer le moteur en le laissant tourner au ralenti.
\item Arrêtez le moteur durant les temps d’attente.
\item Contrôlez régulièrement la consommation de carburant et d’huile.
\item {\em Effectuez les travaux d’entretien selon le plan d’entretien, par le service compétent.}
\stopitemize
\stopPictPar
\stoptextbackground

\page [yes]


\subsection{Consigne de sécurité}

\startSymList
\PHgeneric
\SymList
Pour\index{maintenance+prescriptions de sécurité} éviter de causer des dégâts au véhicule et à
l’agrégat ainsi que des accidents lors des travaux de maintenance, il est impératif de prendre
en considération les prescriptions de sécurité particulières mentionnées ci-après.
Référez-vous également aux prescriptions générales de sécurité (\about[safety:risques], \at{dès la page}[safety:risques]).
\stopSymList

\starttextbackground [FC]
\startPictPar
\PMgeneric
\PictPar
\textDescrHead{Prévention des accidents}
Contrôlez l’état du véhicule après chaque intervention de maintenance\index{prévention des accidents}
ou de réparation. Assurez-vous tout particulièrement avant de vous engager sur la voie publique,
que tous les éléments de sécurité et de signalisation (\eG\ éclairage) sont en parfait état de marche.
\stopPictPar
\stoptextbackground
\blank [big]

\start
\setupparagraphs [SymList][1][width=6em,inner=\hfill]
\startSymList
\framed[frame=off,offset=none]{\PHcrushing~\PHfalling}
\SymList
\textDescrHead{Stabilisation du véhicule} Avant chaque intervention de maintenance,
il faut assurer le véhicule contre tout déplacement intempestif: placez le sélecteur de marche sur {\em neutre},
activez le frein de stationnement et posez des cales de roue.
\stopSymList
\stop

\starttextbackground [CB]
\startPictPar \PHpoison \PictPar
\textDescrHead{Démarrage du moteur} Si \index{intoxication}vous devez démarrer le moteur dans un local mal aéré,
arrêtez-le dès\index{gaz d’échappement} que possible, afin d’éviter une intoxication aux monoxydes de carbone.
\stopPictPar

\startitemize
\item Ne démarrez le moteur que si les connexions de la batterie sont en ordre.
\item Ne débranchez jamais la batterie lorsque le moteur tourne.
\item Ne démarrez pas le moteur avec le concours d’un appareil d’aide au démarrage.\index{batterie+chargeur}
Lorsque vous utilisez un chargeur rapide, déconnectez la batterie du véhicule avant de brancher le chargeur!
Considérez les prescriptions d’utilisation qui accompagnent le chargeur rapide.
\stopitemize
\stoptextbackground

\page [yes]


\subsubsection{Protection des composants électroniques}

\startitemize
\item Avant d’entreprendre des travaux de soudure \index{soudure électrique}électriques,
déconnectez les câbles {\em positif} et {\em masse} de la batterie et pontez les deux câbles ensemble.
\item Ne branchez\index{électronic} ou débranchez les boitiers de commande électroniques que lorsqu’ils ne sont plus sous tension.
\item Une inversion de la polarité de la tension d’alimentation (par ex. inversion des câbles de batterie)
peut provoquer des dégâts \index{boitier de commande}irréversibles.
\item En cas de stationnement par une température ambiante \index{température ambiante+extrême} de 80\,°C
(par ex. dans un four de peinture), les boitiers de commande électronique devront être déposés.
\stopitemize


\subsubsection{Diagnostic et mesures}

\startitemize
\item Utilisez exclusivement des câbles de mesure {\em adéquats} (\eG, câbles d’origine de l’appareil),
pour prendre une mesure dans un connecteur.
\item Les téléphones \index{téléphone mobile}et autres appareils radio peuvent perturber le fonctionnement
du véhicule ou de l’appareil de diagnostic et ainsi affecter la sécurité de marche.
\stopitemize


%%%%%%%%%%%%%%%%%%%%%%%%%%%%%%%%%%%%%%%%%%%%%%%%%%%%%%%%%%%%%%%%%%%%%%%%%%%%%%%%%%%%%%%%%
\subsubsection{Qualification du personnel}

\starttextbackground[CB]
\startPictPar \PHgeneric \PictPar
\textDescrHead{Danger d’accident}
Une\index{qualification+personnel de maintenance} opération de maintenance ou de réparation incorrecte peut
affecter le fonctionnement et la sécurité de votre \ProductId. Cela augmente sensiblement le risque d’accident et de blessure.

% \placefig[margin,none]{}
% {\externalfigure[toni:mech:tools]}

Confiez les travaux de maintenance de votre \ProductId à un atelier\index{qualification+atelier} agréé, qui dispose
de toutes les connaissances et de l’outillage nécessaires.

Adressez-vous si besoin au service après||vente \Boschung.
\stopPictPar
\stoptextbackground


\page [yes]


L’utilisation, la maintenance et les réparations de votre \sdeux\ doivent être entreprises exclusivement par des
personnes formées par le service après||vente \Boschung.

Les compétences pour la maintenance et les réparations sont définies par le
service après||vente \Boschung.


\subsubsection{Modifications et transformations}

\starttextbackground[CB]
\startPictPar \PHgeneric \PictPar
\textDescrHead{Danger d’accident}
Toute\index{modification (véhicule)} modification entreprise de façon arbitraire peut affecter le fonctionnement
et la sécurité de votre machine. Il en résulte une augmentation sensible du risque d’accident et de blessure.
\stopPictPar

\startPictPar
\PMwarranty
\PictPar
Tout dommage causé par une intervention ou une modification arbitraire apportée à votre
\ProductId ou un agrégat, ne donne droit à aucune garantie\index{garantie+conditions} ou complaisance.
\stopPictPar
\stoptextbackground

\stopsection


\startsection[title={Consommables pour l’exploitation},
							reference={sec:liquids}]


\subsection{Prévention et manipulation}

\starttextbackground[CB]
\startPictPar \PHpoison \PictPar
\textDescrHead{\index{danger+intoxication}Danger de blessure et d’intoxication}
Lors\index{carburant}\index{lubrifiant}\index{carburant+sécurité}
de contact avec la peau ou d’ingestion d’ingrédients ou de lubrifiants, vous pouvez
subir des blessures ou une intoxication. Respectez les dispositions légales en vigueur lorsque
vous stockez, manipulez et éliminez des ingrédients et lubrifiants.
\stopPictPar
\stoptextbackground

\starttextbackground [FC]
\startPictPar
\PMproteyes\par
\PMprothands
\PictPar
Portez toujours une protection vestimentaire et respiratoire appropriée lorsque
vous manipulez des ingrédients et lubrifiants. Évitez autant que possible d’en inhaler les vapeurs.
Évitez tout contact avec les yeux, la peau et les vêtements. Nettoyez avec de l’eau et du savon les
parties de peau qui ont été en contact avec de tels produits. En cas de contact avec les yeux, rincez
soigneusement avec de l’eau pure et consultez un médecin si nécessaire.
Si vous avez avalé du carburant ou un lubrifiant, consultez immédiatement un médecin!
\stopPictPar
\stoptextbackground

\startSymList
\PPchildren
\SymList
Conservez les ingrédients et lubrifiants hors de portée des enfants.
\stopSymList

\startSymList
\PPfire
\SymList
\textDescrHead{Danger d’incendie}
La\index{danger+incendie} manipulation d’ingrédients et lubrifiants augmente le danger d’incendie,
car il s’agit de produits facilement inflammables. Ne fumez pas, évitez de provoquer des étincelles
et n’approchez pas avec une flamme nue lorsque vous manipulez de tels produits\index{interdiction de fumer}.
\stopSymList

\starttextbackground [FC]
\startPictPar
\PMgeneric
\PictPar
Les lubrifiants doivent être compatibles avec les pièces constitutives des composants
du \pquatre. N’utilisez que des produits testés et agréés par \Boschung. Ces produits sont
mentionnés dans les prescriptions relatives aux ingrédients et lubrifiants \at{page}[page:table:liquids].
\index{additifs}Les additifs pour lubrifiants ne sont pas nécessaires. Si vous utilisez des
additifs spéciaux, vos droits à la garantie\index{garantie+conditions} peuvent être limités.
Adressez-vous au service après||vente \Boschung\ pour tout renseignement complémentaire.
\stopPictPar
\stoptextbackground


\starttextbackground [FC]
\startPictPar
\Penvironment
\PictPar
\textDescrHead{Protection de l’environnement}
Veillez à éliminer les ingrédients et les lubrifiants\index{lubrifiant+élimination},
ainsi que les pièces entrées en contact avec ceux-ci (\eG\ filtres, chiffons),
dans le respect\index{protection de l’environnement} des règles de protection\index{ingrédients+élimination} de l’environnement.
\stopPictPar
\stoptextbackground

\page [yes]

\setups [pagestyle:normal]

\subsection[sec:liqquantities]{Spécifications et volumes de remplissage}

Tous \index{ingrédients+volume de remplissage}\index{lubrifiants+volume de remplissage}les volumes de remplissage
dans\index{maintenance+Consommables} le tableau ci-après sont à titre indicatif.
Après chaque vidange|/|remplissage\index{spécifications+consommables d’exploitation},
un contrôle du niveau est nécessaire.
\blank[big]

\placetable[margin][tab:glyco]{Antigel (\index{antigel}moteur)}
{\noteF\startframedcontent[FrTabulate]
\starttabulate[|Bp|r|r|]
\NC Temp. efficace {[}°C{]}\NC \bf \textminus 25 \NC \bf \textminus 40 \NC\NR
\NC Eau distillée [Vol.||\%] \NC 60 \NC 40 \NC\NR
\NC Antigel [Vol.||\%] \NC 40 \NC {\em max.} 60 \NC\NR
\stoptabulate\stopframedcontent\endgraf
Attention: au-delà de 60\hairspace\%\ d’antigel dans l’eau, la capacité de refroidissement et de protection contre le gel {\em diminue}!}

\placefig[margin][fig:hydrgauge]{\select{caption}{Indicateur de niveau du fluide hydraulique
(côté gauche)}{Indicateur de niveau hydraulique}}
	{\externalfigure[main:hy:level_temp]
	\noteF Le contrôle du niveau du fluide hydraulique se fait {\em quotidiennement} par le regard sur le réservoir.}


%\placetable[here,split][tbl:liquids]{Spécifications et volumes de remplissage des consommables}
%{\readfile{tbl_jb-fr_liquids}{}{\Warn}}

\start
\define [1] \TableSmallSymb {\externalfigure[#1][height=4ex]}
\define\UC\emptY
\pagereference[page:table:liquids]


\setupTABLE	[frame=off,style={\ssx\setupinterlinespace[line=.86\lH]},background=color,
			option=stretch,
			split=repeat]
\setupTABLE	[r]	[each]	[topframe=on,
						framecolor=TableWhite,
						% rulethickness=.8pt
						]

\setupTABLE	[c]	[odd]	[backgroundcolor=TableMiddle]
\setupTABLE	[c]	[even]	[backgroundcolor=TableLight]
\setupTABLE	[c]	[1]		[width=30mm]
\setupTABLE	[c]	[2]		[width=20mm]
\setupTABLE	[c]	[4]		[width=25mm]
\setupTABLE	[c]	[last]	[width=10mm]
\setupTABLE	[r] [first]	[topframe=off,style={\bfx\setupinterlinespace[line=.95\lH]},
						% backgroundcolor=TableDark
						]
\setupTABLE	[r]	[2]		[framecolor=black]

\bTABLE

\bTABLEhead
 \bTR
 \bTC Groupes \eTC
 \bTC Catégories \eTC
 \bTC Classifications \eTC
 \bTC Produits\note[Produkt] \eTC
 \bTC Vol. \eTC
 \eTR
\eTABLEhead

\bTABLEbody
 \bTR \bTD	Moteur diesel \eTD
  \bTD Huile moteur\eTD
  \bTD \liqC{SAE 5W-30}; \liqC{VW\,507.00}\eTD
  \bTD Total Quartz INEO Long Life \eTD
  \bTD	4,3\,l\eTD
  \eTR
 \bTR \bTD	Circuit hydraulique \eTD
  \bTD Huile ATF \eTD
  \bTD \liqC{dexron~iii} \eTD
  \bTD  Total Equiviz ZS 46 (réservoir env. 40\,l) \eTD
  \bTD env. 50\,l\eTD
  \eTR
 \bTR \bTD Circuit hydraulique (option biologique)\eTD
  \bTD Huile ATF \eTD
  \bTD \liqC{dexron~iii} \eTD
  \bTD  Total Biohydran TMP SE 46 (réservoir env. 45\,l) \eTD
  \bTD env. 60\,l\eTD
  \eTR
 \bTR \bTD	Électro||vannes: graissage du noyau \eTD
  \bTD Lubrifiant\eTD
  \bTD Graisse cuivrée \eTD
  \bTD \emptY\eTD
  \bTD s.\,b.\note[Bedarf] \eTD
  \eTR
 \bTR \bTD	Divers: serrure, mécanisme de porte, pédale de frein \eTD
  \bTD Lubrifiant\eTD
  \bTD Spray universel\eTD
  \bTD \emptY\eTD
  \bTD s.\,b.\note[Bedarf] \eTD
  \eTR
 \bTR \bTD	Graissage centralisé \eTD
  \bTD Graisse à roulement (universel)\eTD
  \bTD \liqC{nlg2}\eTD
  \bTD Total Multis EP 2\eTD
  \bTD s.\,b.\note[Bedarf] \eTD
  \eTR
 \bTR \bTD	Système de refroidissement \eTD
  \bTD Antigel|/|antioxydant\eTD
  \bTD TL VW 774 F/G; max. 60\hairspace\% vol.\eTD
  \bTD G12+|/|G12++ (rose|/|violet)\eTD
  \bTD env. 14\,l \eTD
  \eTR
 \bTR \bTD	Pompe à eau hp \eTD
  \bTD Huile moteur\eTD
  \bTD \liqC{SAE 10W-40}; \liqC{api cf~– acea e6}\eTD
  \bTD Total Rubia TIR 8900 \eTD
  \bTD	0,29\,l\eTD
  \eTR
 \bTR \bTD	Climatisation \eTD
  \bTD Réfrigérant\eTD
  \bTD + 20\,ml huile POE\eTD
  \bTD R\,134a\eTD
  \bTD	700\,g\eTD
  \eTR
 \bTR \bTD	Lave||glace \eTD
  \bTD [nc=2] Eau et concentré pour lave-glace, S pour l'été ou W pour l'hiver, respectez les proportions du mélange \eTD
  \bTD Disponible du commerce\eTD
  \bTD s.\,b.\note[Bedarf] \eTD
  \eTR
\eTABLEbody

\eTABLE

\stop
% \setups[pagestyle:normal]
\footnotetext[Bedarf]{{\it s.\,b.} selon besoin; complétez le niveau selon les instructions disponibles.}
\footnotetext[Produkt]{Produits utilisés par \Boschung. D’autres produits répondant aux mêmes spécifications sont autorisés.}

% \placefig[margin][fig:container:high]{Accès aux composants}
% {\externalfigure[main:container:high]
% \noteF Pour accéder aux principaux composants mécaniques, il faut lever la cuve à déchets.}

\stopsection

\page [yes]

\setups [pagestyle:marginless]


\startsection [title={Entretien du moteur diesel},
							reference={sec:workshop:vw},
							]


\subsection [sSec:vw:diagTool]{Système de diagnostic embarqué}

L’unité\startregister[index][reg:main:vw]{maintenance+moteur diesel} de commande du moteur~– J623~– est équipée d’une mémoire de défauts.
Les défauts détectés sur un élément surveillé sont enregistrés dans la mémoire de défauts,
notifiés et classés par type.

Les défauts\index{moteur diesel+diagnostic} sont évalués, puis triés et
enregistrés dans la mémoire jusqu’à ce qu’ils soient effacés.

Les défauts sporadiques sont affichés avec le suffixe \quote{SP}.
Les défauts sporadiques peuvent être causés par un mauvais contact ou
un débranchement temporaire d’un câble électrique.
Si un défaut ne se reproduit plus durant 50 démarrages, il est supprimé
de la mémoire de défauts.

Si un défaut détecté peut influencer le fonctionnement du moteur,
la lampe témoin de diagnostic du moteur~– K29~– et|/|ou la lampe témoin du système
de dépollution des gaz d’échappement s’allume\,(nt).

Le contenu de la mémoire de défauts peut être lu avec l’outil de diagnostic, de test
et d’information système~– VAS 5051/B~–.

Après avoir réparé un ou plusieurs défauts, il faut effacer la
mémoire de défauts.

\subsubsection[sSec:vw:diagTool:connect]{Branchez le système de diagnostic}

\starttextbackground [FC]
\startPictPar
\PMgeneric
\PictPar
Vous trouverez les instructions détaillées à propos du système de
diagnostic~– VAS 5051/B~– dans le manuel fourni avec l’appareil.

Il est également possible d’utiliser un appareil alternatif, par ex. DiagRA.
\stopPictPar
\stoptextbackground

\page [yes]


\subsubsubsubject{Prérequis}

\startitemize
\item Les fusibles doivent être en ordre
\item La tension de la batterie doit être supérieure à 11,5\,V
\item Tous les consommateurs électriques doivent être désactivés
\item Le branchement à la masse doit être en ordre
\stopitemize


\subsubsubsubject{Procédure}

\startSteps
\item Branchez la prise de diagnostic de l’appareil~– VAS 5051/B~– à l’interface de diagnostic du véhicule.
\item Mettez le contact ou, selon la fonction recherchée, démarrez le moteur.
\stopSteps

\subsubsubsubject{Sélection du mode opératoire}

\startSteps [continue]
\item Pressez le bouton \quote{Vehicle self-diagnosis} à l’écran.
\stopSteps


\subsubsubsubject{Sélection du mode système}

\startSteps [continue]
\item Pressez le bouton \quote{01-Engine electronics} à l’écran.
\stopSteps

L’écran affiche le code et l’identification de l’unité de commande du moteur.

Le cas échéant, vérifiez le code de l’unité de commande du moteur.


\subsubsubsubject{Sélection de la fonction de diagnostic}

Toutes les fonctions de diagnostic valables sont affichées à l’écran.

\startSteps [continue]
\item Pressez le bouton pour la fonction désirée à l’écran.
\stopSteps



\subsection [sSec:vw:faultMemory]{Mémoire de défauts}


\subsubsection{Lecture de la mémoire de défauts}

\subsubsubject{Procédure d’interrogation de la mémoire de défauts}

\startSteps
\item Mettez le moteur en marche, régime de ralenti.
\item Branchez le système de diagnostic (voir \in{section}[sSec:vw:diagTool:connect])
et sélectionnez \quote{engine control unit}.
\item Sélectionnez la fonction de diagnostic \quote{004-Contents of fault memory}.
\item Sélectionnez la fonction de diagnostic \quote{004.01-Read fault memory}.
\stopSteps

{\sla Uniquement si le moteur ne démarre pas:}

\startitemize [2]
\item Mettez le contact (contacteur à clé).
\item Si aucun défaut n’est enregistré dans l’unité de commande du moteur,
l’écran affiche \quote{0 fault detected}.
\item Si un ou plusieurs défauts sont enregistrés dans l’unité de commande du moteur,
ils sont affichés successivement à l’écran.
\item Quittez la fonction de diagnostic.
\item Coupez le contact.
\item Réparez chaque défaut potentiel en vous aidant de la table des défauts (voir documentation d’atelier),
effacez la mémoire de défauts.
\stopitemize

\starttextbackground [FC]
\startPictPar
\PMrtfm
\PictPar
Si un ou plusieurs défauts ne peuvent pas être effacés, faites appel au service après||vente \boschung.
\stopPictPar
\stoptextbackground


\subsubsubject{Défauts statiques}

En cas de présence d’un ou plusieurs défauts statiques dans la mémoire de défauts,
nous recommandons d’effectuer les réparations nécessaires, en accord avec le client,
en vous aidant du guide de recherche de défauts.



\subsubsubject{Défauts sporadiques}

Dans le cas où seuls des défauts sporadiques sont présents dans la mémoire de défauts,
et qu’aucun dysfonctionnement en relation avec le système électronique n’a été signalé par le client,
effacez la mémoire de défauts.

\startSteps [continue]
\item Pressez à nouveau sur \quote{continue} \inframed[strut=local]{>} pour poursuivre le test.
\item Pour quitter le guide de recherche de défauts, pressez sur \quote{skip}, puis \quote{close}.
\stopSteps

Maintenant toute la mémoire de défauts est à nouveau interrogée.

Un message est maintenant affiché pour confirmer que tous les défauts sporadiques ont été effacés.
Le protocole de diagnostic est automatiquement envoyé {\em online}.

Le test du système électronique du moteur est complet.


\subsubsection[sSec:vw:faultMemory:errase]{Effacement de la mémoire de défauts}


\subsubsubject{Procédure d’effacement de la mémoire de défauts}

{\sla Prérequis:}

\startitemize [2]
\item Les réparations nécessaires doivent avoir été effectuées.
\stopitemize

\page [yes]


{\sla Procédure:}

\starttextbackground [FC]
\startPictPar
\PMrtfm
\PictPar
Lorsque les défauts ont été corrigés, la mémoire de défauts doit être à nouveau interrogée
et ensuite effacée selon la procédure ci||dessous.
\stopPictPar
\stoptextbackground


\startSteps
\item Mettez le moteur en marche, régime de ralenti.
\item Branchez le système de diagnostic (voir \in{section}[sSec:vw:diagTool:connect])
et sélectionnez \quote{engine control unit}.
\item Sélectionnez la fonction de diagnostic \quote{004-Contents of fault memory}.
\item Sélectionnez la fonction de diagnostic \quote{004.10-Erase fault memory}.
\stopSteps


\starttextbackground [FC]
\startPictPar
\PMrtfm
\PictPar
Si un défaut ne peut pas être effacé, cela signifie que la faute n’est pas corrigée.
\stopPictPar
\stoptextbackground

\startSteps [continue]
\item Quittez la fonction de diagnostic.
\item Coupez le contact.
\stopSteps


\subsection [sSec:vw:lub] {Graissage du moteur diesel}

\subsubsection [ssSec:vw:oilLevel] {Vérification du niveau d’huile du moteur}

\starttextbackground [FC]
\startPictPar
\PMrtfm
\PictPar
Le niveau d’huile du moteur\index{huile moteur+niveau} doit impérativement rester au||dessous
de la marque {\CAP max.} de la jauge à huile.
Un niveau d’huile trop élevé\index{niveau+huile moteur} risque d’endommager le convertisseur catalytique.
\stopPictPar
\stoptextbackground

\startSteps
\item Arrêtez le moteur et laissez||le au repos environ trois minutes.
\item Retirez la jauge à huile de son orifice
et essuyez||la proprement; introduisez||la à nouveau complètement.
\item Retirez||la à nouveau et observez le niveau d’huile comme suit:
\startfigtext[right][fig:vw:gauge]{Lecture de la jauge à huile}
{\externalfigure[VW_Oil_Gauge][width=50mm]}
\startitemize [A]
\item Niveau maximum, aucun remplissage nécessaire.
\item Niveau correct, un complément jusqu’à la marque \LAa\ est facultatif.
\item Le niveau est insuffisant, un remplissage est nécessaire jusqu’à ce que le niveau
se situe dans la zone \LAb.
\stopitemize
{\em Si le niveau est au||dessus de la marque \LAa, cela risque d’endommager le convertisseur catalytique.}
% \item Si le niveau est au||dessous de la marque \LAc, il est nécessaire de remplir
% jusqu’à atteindre la marque \LAa.
\stopfigtext
\stopSteps


\subsubsection [ssSec:vw:oilDraining] {Vidange d’huile du moteur}

\starttextbackground [FC]
\startPictPar
\PMrtfm
\PictPar
Le filtre à huile du moteur diesel de la Urban||Sweeper S2 est monté en position verticale.
C’est pourquoi il est nécessaire de remplacer le filtre {\em avant} de vidanger l’huile.
Lorsque le filtre à huile est déposé, une soupape s’ouvre et l’huile dans boitier du filtre
à huile s’écoule dans le carter du moteur.
\stopPictPar
\stoptextbackground

\startSteps
\item Placez un bac\index{moteur diesel+vidange d’huile} récupérateur au||dessous du moteur.
\item Dévissez le bouchon de vidange d’huile\index{huile moteur+vidange} et laissez l’huile s’écouler entièrement.
\stopSteps

\starttextbackground [FC]
\startPictPar
\PMrtfm
\PictPar
Utilisez un bac suffisamment volumineux pour contenir la totalité de l’huile du moteur.
Consultez la \in{section}[sec:liqquantities] pour plus d’information sur la qualité requise et la quantité nécessaire.

Le bouchon de vidange est monté avec un joint incorporé et non remplaçable. Pour cette raison, il est nécessaire
de remplacer le bouchon lors de chaque vidange.
\stopPictPar
\stoptextbackground

\startSteps [continue]
\item Montez un bouchon de vidange neuf; \TorqueR 30\,Nm.
\item Remplir le moteur avec la quantité nécessaire et la qualité d’huile requise (voir \in{section}[sec:liqquantities]).
\stopSteps


\subsubsection [ssSec:vw:oilFilter] {Remplacement du filtre à huile du moteur}

\starttextbackground [FC]
\startPictPar
\PMrtfm
\PictPar
\startitemize [1]
\item Respectez\index{moteur diesel+filtre à huile} les dispositions légales relatives à l’élimination et le recyclage des déchets polluants.
\item Remplacez le filtre à huile {\em avant} d’effectuer la vidange du carter\index{filtre à huile+moteur diesel} (voir \in{section}[ssSec:vw:oilDraining]).
\item Lubrifiez les O-rings avec de l’huile avant le montage.
\stopitemize
\stopPictPar
\stoptextbackground

\startfigtext[right][fig:vw:oilFilter]{Filtre à huile}
{\externalfigure[VW_OilFilter_03][width=50mm]}
\startSteps
\item Déposer le couvercle \Lone\ du boitier à l’aide d’un système de clé à douille approprié.
\item Nettoyez les plans||joints du couvercle et du boitier du filtre à huile.
\item Remplacez l’élément filtrant \Lthree.
\item Remplacez les O||ring \Ltwo\ et \Lfour.
\item Remontez le couvercle du boitier de filtre à huile.
\stopSteps



\subsubsubject{Données techniques}


\hangDescr{Couple de serrage du couvercle:} \TorqueR 25\,Nm.

\hangDescr{Huile moteur prescrite:} Selon tableau \atpage[sec:liqquantities].

\stopfigtext



\subsubsection [ssSec:vw:oilreplenish] {Remplissage d’huile moteur}

\starttextbackground [FC]
\startPictPar
\PMrtfm
\PictPar
\startitemize [1]
\item Nettoyez soigneusement\index{huile moteur} le pourtour de l’orifice de remplissage avec un chiffon {\em avant} d’enlever le bouchon.
\item Remplissez le moteur uniquement\index{moteur diesel+remplissage d’huile} avec de l’huile approuvée par \boschung.
\item Remplissez progressivement avec de petites quantités.
\item Afin d’éviter un surremplissage, laissez l’huile s’écouler dans le carter du moteur et
vérifiez le niveau avec la jauge (voir \in{section}[ssSec:vw:oilLevel])
après chaque ajout.
\stopitemize
\stopPictPar
\stoptextbackground

\startfigtext[right][fig:vw:oilFilter]{Remplissage d’huile}
{\externalfigure[s2_bouchonRemplissage][width=50mm]}
\startSteps
\item Retirez la jauge à huile de son orifice d’une dizaine de centimètres, de manière
à laisser s’échapper l’air durant le remplissage.
\item Enlevez le bouchon de l’orifice de remplissage du moteur.
\item Remplissez le moteur selon les prescriptions (voir encadré ci||dessus).
\item Remettre soigneusement le bouchon de l’orifice de remplissage.
\item Mettez le moteur en marche.
\item Effectuez un contrôle du niveau (voir \in{section}[ssSec:vw:oilLevel]).
\stopSteps

\stopfigtext


\subsection [sSec:vw:fuel] {Système d’alimentation en carburant}

\subsubsection [ssSec:vw:fuelFilter] {Remplacement du filtre à carburant}

\starttextbackground [FC]
\startPictPar
\PMrtfm
\PictPar
\startitemize [1]
\item Respectez\index{moteur diesel+filtre à carburant} les dispositions légales relatives à l’élimination et le recyclage des déchets polluants.
\item Ne débranchez pas les conduites de carburant de la partie supérieure du filtre.
\item N’exercez aucune traction sur les points de connexions des conduites de carburants,
vous risquez de provoquer des dégâts à la partie supérieure du filtre.
\stopitemize
\stopPictPar
\stoptextbackground

\adaptlayout [height=+5mm]

\startfigtext[right][fig:vw:oilFilter]{Filtre à carburant}
{\externalfigure[s2_fuelFilter_location][width=50mm]}

{\sla Préambule:}

Le boitier\index{filtre à carburant} du filtre à carburant est fixé à l’avant du moteur sur le châssis côté droit.
Déposez les deux vis de fixation au moyen d’une clé à béquille 10\,mm et d’une clé à bague 10\,mm.

\stopfigtext


\page [yes]

\setups [pagestyle:normal]

{\sla Procédure:}

\startLongsteps
\item Déposez toutes les vis de la partie supérieure du filtre;
séparez-la du filtre.
\stopLongsteps

\starttextbackground [FC]
\startPictPar
\PMrtfm
\PictPar
Si nécessaire, insérez un tournevis plat dans la fente (\in{\LAa, fig.}[fig:fuelfilter:detach])
pour décoller la partie supérieure du filtre.
\stopPictPar
\stoptextbackground

\placefig [margin] [fig:fuelfilter:detach]{Séparation du filtre à carburant}
{\externalfigure[fuelfilter:detach]}

\placefig [margin] [fig:fuelfilter:explosion]{Filtre à carburant}
{\externalfigure[fuelfilter:explosion]}

\startLongsteps [continue]
\item Extrayez l’élément filtrant de la partie inférieure du filtre.
\item Déposez le joint (\in{\Ltwo, fig.}[fig:fuelfilter:explosion]) de la partie supérieure du filtre.
\item Nettoyez soigneusement les parties inférieure et supérieure du filtre.
\item Placez un élément filtrant neuf dans la partie inférieure.
\item Montez un joint (\in{\Ltwo, fig.}[fig:fuelfilter:explosion])
neuf légèrement humecté de carburant sur la partie supérieure.
\item Assemblez les deux parties du filtre de manière à ce qu’elles soient en contact
tout autour du filtre.
\item Vissez à la main chaque vis de manière à {\em plaquer} les deux parties ensemble
tout autour du filtre, puis serrez-les au couple prescrit, en ordre croisé.
\stopLongsteps


\subsubsubject{Données techniques}

\hangDescr{Couple de serrage des vis de fixation du couvercle:} \TorqueR 5\,Nm.

\startLongsteps [continue]
\item Mettez le contact pour purger le système, démarrez le moteur
et laissez||le tourner 1 à 2 minutes au régime de ralenti.
\item Effacez la mémoire de défauts selon procédure \atpage[sSec:vw:faultMemory:errase].
\stopLongsteps



\subsection [sSec:vw:cooling] {Système de refroidissement}


\starttextbackground [FC]
\startPictPar
\PMrtfm
\PictPar
\startitemize [1]
\item Le liquide\index{moteur diesel+refroidissement} de refroidissement doit correspondre à la norme prescrite (voir tableau \atpage[sec:liqquantities]).
\item Pour préserver ses propriétés anticorrosion et antigel, le mélange ne doit être dilué
qu’avec de l’eau distillée\index{liquide de refroidissement} selon le tableau ci||dessous.
\item Le liquide de refroidissement ne doit jamais être complété avec de l’eau, car ses propriétés
anticorrosion et antigel seraient modifiées.
\stopitemize
\stopPictPar
\stoptextbackground


\subsubsection [sSec:vw:coolingLevel] {Niveau du liquide de refroidissement}

\placefig [margin] [fig:coolant:level] {Niveau du liquide de refroidissement}
{\externalfigure[coolant:level]}


\placefig [margin] [fig:refractometer] {Réfractomètre VW T\,10007}
{\externalfigure[coolant:refractometer]}

\placefig [margin] [fig:antifreeze] {Contrôle du point de congélation}
{\externalfigure[coolant:antifreeze]}


\startSteps
\item Levez la cuve à déchets et placez le dispositif de sécurité sur les vérins de levage.
\item Observez le niveau du liquide\index{niveau+liquide de refroidissement} de refroidissement dans le vase d’expansion:
le niveau doit se situer au||dessus de la marque {\cap min}.
\stopSteps

\start
\define [1] \TableSmallSymb {\externalfigure[#1][height=4ex]}
\define\UC\emptY
\pagereference[page:table:liquids]


\setupTABLE	[frame=off,style={\ssx\setupinterlinespace[line=.86\lH]},background=color,
			option=stretch,
			split=repeat]
\setupTABLE	[r]	[each]	[topframe=on,
						framecolor=TableWhite,
						% rulethickness=.8pt
						]

\setupTABLE	[c]	[odd]	[backgroundcolor=TableMiddle]
\setupTABLE	[c]	[even]	[backgroundcolor=TableLight]
\setupTABLE	[r] [first]	[topframe=off,style={\bfx\setupinterlinespace[line=.95\lH]},
						% backgroundcolor=TableDark
						]
\setupTABLE	[r]	[2]		[framecolor=black]

\bTABLE

\bTABLEhead
 \bTR
 \bTC Protection antigel, jusqu’à… \eTC
 \bTC Ratio antigel G12\hairspace ++\eTC
 \bTC Vol. antigel \eTC
 \bTC Vol. eau distillée \eTC
 \eTR
\eTABLEhead

\bTABLEbody
 \bTR \bTD	–\,25\,°C \eTD
  \bTD 40\hairspace\% \eTD
  \bTD 3,8\,L \eTD
  \bTD 4,2\,L \eTD
  \eTR
 \bTR \bTD	–\,35\,°C \eTD
  \bTD 50\hairspace\% \eTD
  \bTD 4,0\,L \eTD
  \bTD 4,0\,L \eTD
  \eTR
 \bTR \bTD	–\,40\,°C \eTD
  \bTD 60\hairspace\%  \eTD
  \bTD 4,2\,L \eTD
  \bTD 3,8\,L \eTD
  \eTR
\eTABLEbody

\eTABLE

\stop

\subsubsection [sSec:vw:coolingFreeze] {Niveau du liquide de refroidissement}

Contrôlez le point de congélation\index{point de congélation} avec un réfractomètre approprié (voir \in{fig,}[fig:refractometer]: VW T\,10007\,A).
Observez l’échelle 1: G12\hairspace ++ (voir \in{fig.}[fig:antifreeze]).

\page [yes]


\subsection [sSec:vw:airFilter] {Système d’alimentation d’air}

Le système d’alimentation d’air comprend le filtre à air accessible depuis le portillon arrière
du côté droit de la \sdeux (voir \in{fig.}[fig:airFilter]).

\placefig [margin] [fig:airFilter] {Filtre à air du moteur}
{\externalfigure[vw:air:filter]
\noteF
\startLeg
\item Languette de sûreté
\item Capot inférieur
\item Orifice de purge
\item Capteur de pression
\stopLeg}

\subsubsubject{Conditions d’utilisation}

La balayeuse travaille fréquemment dans un environnement poussiéreux. Pour cette raison,
il est nécessaire de contrôler|/|nettoyer le filtre à air chaque semaine (voir \about[table:scheduleweekly], \atpage[table:scheduleweekly])
et de procéder à son remplacement si nécessaire.


\subsubsubject{Autodiagnostic}


Le conduit d’aspiration est équipé d’un capteur de pression (\in{\Lfour, fig.}[fig:airFilter]) permettant de mesurer la perte de charge\footnote{%
Réduction du débit d'air due à la résistance du filtre au passage de l'air.}
au niveau du filtre à air.
Si le filtre à air est obstrué, le témoin \textSymb{vpadWarningFilter} est affiché à l’écran du Vpad, ainsi que le message d’erreur \VpadEr{851}.

\subsubsubject{Entretien|/|remplacement}

\startSteps
\item Tirez la languette de sûreté \Lone vers le bas (voir \in{fig.}[fig:airFilter]).
\item Pivotez le capot \Ltwo dans le sens antihoraire, dégagez||le.
\item Déposez l’élément filtrant pour le contrôler, remplacez||le si nécessaire.
\item Nettoyez l’intérieur avant de remonter le filtre en suivant la procédure inverse.
\stopSteps

\page [yes]


\subsection [sSec:vw:belt] {Courroie Multi-V}

La courroie\index{moteur diesel+courroie} Multi-V du moteur transmet le mouvement de rotation de la poulie du vilebrequin à l’alternateur
ainsi qu’au compresseur de climatisation (selon équipement). La tension est assurée par
la pression exercée par un galet tendeur sur le dernier segment\index{courroie Multi||V} (entre l’alternateur et le vilebrequin).

\subsubsection [sSec:belt:change] {Remplacement de la courroie Multi-V}

\placefig [margin] [fig:belt:tool] {Pige VW T\,10060\,A}
{\externalfigure[vw:belt:tool]}

\placefig [margin] [fig:belt:overview] {Galet tendeur}
{\externalfigure[vw:belt:overview]}

\placefig [margin] [fig:belt:tens] {Emplacement de la pige}
{\externalfigure[vw:belt:tens]}


\subsubsubject{Avec compresseur de climatisation}


{\sla Outillage spécial nécessaire:}

Pige VW T\,10060\,A pour le maintien du galet tendeur.

\startSteps
\item Repérez le sens de rotation de la courroie Multi-V.
\item Avec une clé à bague contre||coudée, pivotez le bras du galet tendeur
dans le sens horaire (voir \in {fig.}[fig:belt:overview]).
\item Placez la pige de maintien lorsque les trous prévus à cet effet sont alignés
(voir flèches, \in {fig.}[fig:belt:tens]).
\item Déposez la courroie Multi-V.
\stopSteps

Le montage de la courroie se fait dans le sens inverse de la procédure ci||dessus.

\starttextbackground [FC]
\startPictPar
\PMrtfm
\PictPar
\startitemize [1]
\item Respectez le sens de rotation de la courroie.
\item Après le montage, vérifiez l’alignement de la courroie sur chaque poulie.
\item Démarrez le moteur et contrôlez le passage de la courroie.
\stopitemize
\stopPictPar
\stoptextbackground


\subsubsubject{Sans compresseur de climatisation}


{\sla Matériel nécessaire:}

Jeu de réparation, comprenant les instructions de réparation, la courroie et l’outillage spécial nécessaire\footnote{%
Référez||vous au catalogue de pièces de rechange, sous \quote{pièces de service}.}.

\startSteps
\item Sectionnez la courroie Multi-V.
\item Continuez selon les instructions fournies.
\stopSteps

\starttextbackground [FC]
\startPictPar
\PMrtfm
\PictPar
\startitemize [1]
\item Après le montage, vérifiez l’alignement de la courroie sur chaque poulie.
\item Démarrez le moteur et contrôlez le passage de la courroie.
\stopitemize
\stopPictPar
\stoptextbackground


\subsubsection [sSec:belt:tens] {Remplacement du galet tendeur}

{\sla Version avec compresseur de climatisation uniquement}

\blank [medium]

\placefig [margin] [fig:belt:tens:change] {Remplacement du galet tendeur}
{\externalfigure[vw:belt:tens:change]
\noteF
\startLeg
\item Galet tendeur
\item Boulon de sécurité
\stopLeg

{\bf Couple de serrage}

Boulon de sécurité:

\TorqueR 20\,Nm\:+\:½~tour (180\,°).}

\startSteps
\item Déposez la courroie Multi-V selon la procédure décrite \atpage[sSec:belt:change].
\item Déposez les éléments périphériques nécessaires (selon équipement).
\item Déposez le boulon de sécurité (\in{\Ltwo, fig.}[fig:belt:tens:change]).
\stopSteps

Le montage du galet tendeur se fait dans le sens inverse de la procédure ci||dessus.

\starttextbackground [FC]
\startPictPar
\PMrtfm
\PictPar
\startitemize [1]
\item Remontez impérativement un boulon de sécurité neuf.
\item Couple de serrage: voir \in{fig.}[fig:belt:tens:change].
\stopitemize
\stopPictPar
\stoptextbackground

\stopregister[index][reg:main:vw]

\stopsection

\page[yes]

\setups[pagestyle:marginless]

\startsection[title={Installation hydraulique},
							reference={sec:hydraulic}]

\starttextbackground [FC]
% \startfiguretext[left,none]{}
% {\externalfigure[toni_melangeur][width=30mm]}

\startSymPar
\externalfigure[toni_melangeur][width=4em]
\SymPar
\textDescrHead{Recyclage des huiles}
Les huiles usagées et les lubrifiants ne peuvent être ni abandonnés, ni brulés à l’air libre.\par

Les huiles usagées et les lubrifiants ne peuvent pas être déversés dans les réseaux d’évacuation d’eaux usées ni rejetés dans le milieu naturel ou les ordures ménagères.\par

Les huiles usagées ne doivent pas être mélangées à d’autres liquides car le mélange génère un risque d’introduction de produits toxiques ou difficiles à éliminer.
\stopSymPar
\stoptextbackground
\blank [big]

% \starthangaround{\PMgeneric}
% \textDescrHead{Qualification du personnel}
% Toute intervention sur l’installation hydraulique de votre véhicule ne peut être réalisée que par une personne dument qualifiée, ou par un service reconnu par \boschung.
% \stophangaround
% \blank[big]

\startSymList
\PHgeneric
\SymList
\textDescrHead{Exigence particulière} L’installation est très sensible aux impuretés contenues dans l’huile. Il est impératif de travailler dans un milieu qui présente les conditions de propretés impeccables.
\stopSymList

\startSymList
\PHhot
\SymList
\textDescrHead{Risque d’éclaboussure}
Avant d’intervenir sur l’installation hydraulique de la \sdeux, il est impératif de libérer la pression résiduelle dans le circuit concerné. Des éclaboussures d’huile chaude peuvent provoquer des brulures.
\stopSymList

\startSymList
\PHhand
\SymList
\textDescrHead{Risque d’écrasement}
La cuve à déchets doit impérativement reposer en position basse, ou être assurée mécaniquement~— avec un support approprié~— avant d’intervenir sur l’installation hydraulique de la \sdeux.
\stopSymList

\startSymList
\PImano
\SymList
\textDescrHead{Mesure de la pression}
Branchez un manomètre sur l’un des raccords de contrôle type \quote{minimess} disposé sur le circuit pour mesurer la pression. Utilisez un manomètre avec une plage de mesure appropriée.
\stopSymList

\page [yes]

\setups[pagestyle:normal]

\subsection{Intervalles de maintenance}

\start

	\setupTABLE	[frame=off,
				style={\ssx\setupinterlinespace[line=.93\lH]},
				background=color,
				option=stretch,
				split=repeat]
	\setupTABLE	[r]	[each]	[
							topframe=on,
							framecolor=white,
							backgroundcolor=TableLight,
							% rulethickness=.8pt,
							]

	% \setupTABLE	[c]	[odd]	[backgroundcolor=TableMiddle]
	% \setupTABLE	[c]	[even]	[backgroundcolor=TableLight]
	\setupTABLE	[c]	[1]		[ % width=30mm,
							style={\mdx\setupinterlinespace[line=.93\lH]},
							]
	\setupTABLE	[r] [first]	[topframe=off,
							style={\bfx\setupinterlinespace[line=.93\lH]},
							backgroundcolor=TableMiddle,
							]
	% \setupTABLE	[r]	[2]		[style={\ssBfx\setupinterlinespace[line=.93\lH]}]


\bTABLE

\bTABLEhead
\bTR\bTD Opérations \eTD\bTD Intervalles \eTD\eTR
\eTABLEhead

\bTABLEbody
\bTR\bTD Contrôlez les fuites d’huiles \eTD\bTD Quotidien \eTD\eTR
\bTR\bTD Contrôlez le niveau d’huile \eTD\bTD Quotidien \eTD\eTR
\bTR\bTD Contrôlez l’état les conduites hydrauliques, remplacez||les si nécessaire \eTD\bTD 600\,h|/|12\,mois \eTD\eTR
\bTR\bTD Remplacez le filtre de retour|/|aspiration hydraulique \eTD\bTD 600\,h|/|12\,mois \eTD\eTR
\bTR\bTD Graissez le noyau des électrovannes avec de la graisse cuivrée  \eTD\bTD 600\,h|/|12\,mois \eTD\eTR
\bTR\bTD Vidangez le fluide hydraulique \eTD\bTD 1'200\,h|/|24\,mois \eTD\eTR
\eTABLEbody
\eTABLE
\stop


\subsection[niveau_hydrau]{Niveau du fluide hydraulique}

\placefig[margin][fig:hydraulic:level]{Niveau du fluide hydraulique}
{\externalfigure[hydraulic:level]
\noteF
\startLeg
\item Niveau optimal du fluide hydraulique.
\stopLeg}

Un voyant transparent\index{niveau+fluide hydraulique}\index{maintenance+installation hydraulique}
permet de vérifier le niveau du fluide hydraulique. Lorsqu’une baisse du niveau est constatée,
il faut déterminer la cause avant de rétablir le niveau. Respectez la fréquence de remplacement~–
selon tableau ci||dessus~– et la qualité du fluide hydraulique~– selon le tableau \at{page}[sec:liqquantities].


\subsubsection{Compléter le niveau du fluide hydraulique}

Remplissez le réservoir hydraulique jusqu’à ce qu’il remplisse le voyant du milieu. Mettez le moteur diesel en marche puis complétez le niveau.

\subsection{Vidange du réservoir hydraulique}

Le volume de remplissage et la qualité requise du fluide hydraulique sont indiqués dans le tableau \at{page}[sec:liqquantities].

\startSteps
\item Déposez le bouchon de remplissage du réservoir hydraulique.
\item Vidangez le réservoir au moyen d’un aspirateur à huile ou déposez le bouchon de vidange.

Le bouchon est placé au bas du réservoir devant la roue arrière gauche (\in{fig.}[fig:hydraulic:fluidDrain]).
\item Remplissez le réservoir hydraulique jusqu’à ce qu’il remplisse le voyant du milieu.
Mettez le moteur diesel en marche puis complétez le niveau simultanément.
\stopSteps

\placefig[margin][fig:hydraulic:fluidDrain]{Bouchon de vidange}
{\externalfigure[hydraulic:fluidDrain]}


\placefig[margin][fig:hydraulic:returnFilter]{Filtre hydraulique}
{\externalfigure[hydraulic:returnFilter]}

\subsection[filtres:nettoyage]{Filtre de retour et d’aspiration}

\startSteps
\item Levez la cuve à déchets, placez le dispositif de sécurité anti||écrasement.
\item Déposez le couvercle du filtre sur le réservoir hydraulique (\in{fig.}[fig:hydraulic:returnFilter]).
\item Remplacez\index{filtre à huile+hydraulique} l’élément filtrant par un élément neuf.
\item Montez un joint torique neuf, humectez-le avec du fluide hydraulique.
\item Vissez le couvercle à la force des deux mains (env. 20\,Nm).
\stopSteps

\page [yes]


\subsection[sec:solenoid]{Graissage des électrovannes}

\placefig[margin][graissage_bobine]{Graissage des électrovannes}
{\externalfigure[graissage_bobine][M]
\noteF
\startLeg
\item Bobine d’électrovanne
\item Noyau de la bobine
\stopLeg}

L’humidité et les résidus de sel infiltrés au cœur des bobines électromagnétiques
favorisent l’altération des noyaux par réaction chimique (corrosion).
Les noyaux d’électrovannes doivent être enduits une fois par an avec de la graisse cuivrée,
résistante à une température d’au moins 50\,°C, résistante à l’eau et anticorrosive:
\startSteps
\item Déposez la bobine d’électrovanne (\in{\Lone, fig.}[graissage_bobine]).
\item Enduisez le noyau (\in{\Ltwo, fig.}[graissage_bobine]) de graisse anticorrosion et remontez la bobine.
\stopSteps

\subsection{Remplacement des flexibles}

Le caoutchouc\index{flexibles+intervalle de remplacement} de protection et les tissus de renforcement des flexibles
subissent inexorablement un vieillissement naturel.
Le chargé de maintenance de la \sdeux\ doit veiller à ce que les flexibles de l’installation hydraulique soient remplacés aux intervalles prescrits,
même s’ils ne présentent pas de défauts apparents.

Les flexibles doivent être bridés correctement sur le véhicule de manière à éviter une usure anticipée due aux frottements.
Veillez à ce qu’ils n’aient aucun contact avec des éléments qui peuvent provoquer des frottements (dus aux vibrations, par exemple).

\stopsection

\page [yes]

\setups [pagestyle:bigmargin]


\startsection[title={Système de freinage},
							reference={sec:brake}]

\placefig[margin][fig:brake:rear]{Frein à tambour}
{\startcombination [1*2]
{\externalfigure[brake:wheelHub]}{\slx Moyeu de roue arrière}
{\externalfigure[brake:drum]}{\slx Mécanisme et garnitures}
\stopcombination}

Lors de chaque service d’entretien régulier, il est nécessaire\index{maintenance+système de freinage}
de déposer les tambours \Lfour de frein pour le nettoyage
du mécanisme de frein \Lseven et le contrôle visuel de l’état des garnitures \Lfive et \Lsix, \in{fig.}[fig:brake:rear].

\subsubject {Démontage}

\startSteps
\item Placez le véhicule sur un élévateur approprié et soulevez les roues.
\item Déposez les roues.
\stopSteps


{\sla Démontage des freins avant}

\startSteps [continue]
\item Déposez le tambour de frein \Lfour.
\stopSteps

{\sla Démontage des freins arrière}

\startSteps [continue]
\item Déposez le cache au centre du moyeu de roue \Lone.
\item Déposez la vis \Ltwo et déposez la pièce intermédiaire.
\item Dévissez l’écrou de moyeu \Lthree avec une clé à douille.
\item Déposez le moyeu avec le tambour de frein.
\stopSteps


\subsubject {Remontage}

Remontez les tambours en suivant la procédure inverse (serrez l’écrou de chaque moyeu de roues arrières \Lthree avec au couple prescrit: \TorqueR 190\,Nm)

\stopsection

\page [yes]

\setups [pagestyle:normal]


\startsection[title={Contrôle et entretien des pneus},
							reference={sec:pneumatiques}]

Les pneus\index{pneus+usure}\index{maintenance+pneus} doivent toujours être en parfait état pour remplir leurs deux fonctions principales : une bonne tenue de route et un bon freinage.
L’usure des pneus et, plus encore, les problèmes de gonflage, en particulier de sous||gonflage, sont d’importants facteurs d’accidents.

\subsection{Les points à vérifier pour rouler en sécurité}

\subsubsection{Contrôle de l’usure des pneus}
Il faut surveiller l’usure des pneus en contrôlant les indicateurs d’usure placée au fond d’une rainure (voir~\in{fig.}[pneususure]). Par ailleurs, un examen visuel rapide permet de déterminer l’origine des principales anomalies du pneu.

\placefig[margin][pneususure]{Contrôle d’usure}
{\Framed{\externalfigure[pneusUsure][M]}}

\placefig[margin][pneusdomages]{Pneus endommagés}
{\Framed{\externalfigure[pneusDomages][M]}}

\startitemize
\item Usure sur les bords de la bande de roulement: le pneu est sous||gonflé.
\item Usure plus marquée au centre : le pneu est surgonflé.
\item Usure asymétrique sur un des côtés du pneu: le train avant (parallélisme ou géométrie) est mal réglé.
\item Présence de craquelures sur la bande de roulement:
le pneu est trop vieux, le caoutchouc se durci avec le temps et se fend (voir~\in{fig.}[pneusdomages]).
\stopitemize

\starttextbackground[CB]
\startPictPar
\PHgeneric
\PictPar
\textDescrHead{Les dangers d’un pneu usé}
Un pneu usé ne remplit plus sa fonction, notamment pour l’évacuation de l’eau ou de la boue.
L’efficacité au freinage diminue et la tenue de route devient incertaine.
Un pneu usé glisse facilement, surtout si la route est mouillée ou boueuse.
Les risques de perte d’adhérence augmentent.
\stopPictPar
\stoptextbackground

\subsubsection{Le gonflage ou la pression des pneus}
Sur les flancs du pneu ou à l’intérieur de l’habitacle figure un marquage
qui indique une pression de gonflage et de charge.

Même\index{pneus+gonflage} si les pneus sont en bon état,
ils se dégonflent plus ou moins lentement chaque mois (plus on roule, plus ils se dégonflent).
Il faut vérifier leur pression tous les mois et à froid. S’ils sont chauds,
il faut ajouter 0,3 bar à la pression prescrite.

\start
\setupcombinations[M]
\placefig[margin][pneuspression]{Pression de gonflage}
{\Framed{\externalfigure[pneusPression][M]}
\noteF
\startLeg
\item Gonflage normal
\item Pression trop élevée
\item Pression insuffisante
\stopLeg
La pression de gonflage prescrite est inscrite sur la plaquette d’identification des roues, dans la cabine du côté passager.}
\stop

\starttextbackground[CB]
\startPictPar
\PHgeneric
\PictPar
\textDescrHead{Les dangers d’un sous||gonflage}
Un pneu risque d’éclater quand il n’est pas suffisamment gonflé. En effet,
le pneu s’écrase plus à chaque tour de roue quand il est sous||gonflé ou en cas de surcharge.
Le caoutchouc s’échauffe exagérément et peut provoquer un détachement de la bande de roulement dans un virage.
\stopPictPar
\stoptextbackground

\stopsection

\page [yes]

\setups[pagestyle:marginless]


\startsection[title={Châssis},
							reference={main:chassis}]

\subsection{Fixation des éléments de sécurité}

Lors de chaque service d’entretien, il est important de contrôler\index{maintenance+châssis}
le serrage~– au couple prescrit~– des vis de fixation des éléments de sécurité
du véhicule, en particulier le système de direction articulé et les essieux.

\blank [big]

\startfigtext [left] [fig:frontAxle:fixing] {Essieu avant}
{\externalfigure [frontAxle:fixing]}
{\sla Fixation de l’essieu avant}
\startLeg
\item Fixation de la lame de ressort: \TorqueR 150\,Nm
\item Fixation des unités de traction: \TorqueR 78\,Nm
\stopLeg

{\sla Fixation de l’essieu arrière}
\startLeg
\item Fixation de la lame de ressort: \TorqueR 150\,Nm
\stopLeg

\stopfigtext

\start

\setupTABLE	[frame=off,style={\ssx\setupinterlinespace[line=.93\lH]},background=color,
			option=stretch,
			split=repeat]

\setupTABLE	[r]	[each]	[topframe=on,
						framecolor=white,
						% rulethickness=.8pt
						]

\setupTABLE	[c]	[odd]	[backgroundcolor=TableMiddle]
\setupTABLE	[c]	[even]	[backgroundcolor=TableLight]
\setupTABLE	[c]	[1]		[style={\mdx\setupinterlinespace[line=.93\lH]}]
\setupTABLE	[r] [first]	[topframe=off,style={\bfx\setupinterlinespace[line=.93\lH]},
						]
% \setupTABLE	[r]	[2]		[style={\bfx\setupinterlinespace[line=.93\lH]}]


\bTABLE

\bTABLEhead
\bTR [backgroundcolor=TableDark] \bTD [nc=3] Couples de serrage \eTD\eTR
% \bTR\bTD Position \eTD\bTD Type de vis \eTD\bTD Couple \eTD\eTR
\eTABLEhead

\bTABLEbody
\bTR\bTD Moteurs d’entrainement gauche|/|droit \eTD\bTD M12\:×\:35~8.8 \eTD\bTD 78\,Nm \eTD\eTR
\bTR\bTD Pompe de travail \eTD\bTD M16\:×\:40~100 \eTD\bTD 330\,Nm \eTD\eTR
\bTR\bTD Pompe d’entrainement \eTD\bTD M12\:×\:40~100 \eTD\bTD 130\,Nm \eTD\eTR
\bTR\bTD Lames de ressort avant|/|arrière \eTD\bTD M16\:×\:90|/|160~8.8 \eTD\bTD 150\,Nm \eTD\eTR
% \bTR\bTD Fixation du système oscillant \eTD\bTD M12\:×\:40~8.8 \eTD\bTD 78\,Nm \eTD\eTR
\bTR\bTD Fixation de la cuve à déchets \eTD\bTD M10\:×\:30 Verbus Ripp~100 \eTD\bTD 80\,Nm \eTD\eTR
\bTR\bTD Écrous de roues \eTD\bTD M14\:×\:1,5 \eTD\bTD 180\,Nm \eTD\eTR
\bTR\bTD Fixation du balai frontal \eTD\bTD M16\:×\:40~100 \eTD\bTD 180\,Nm \eTD\eTR
\eTABLEbody
\eTABLE
\stop


\stopsection

\page [yes]

\startmode [main:centralLubrication]

\startsection[title={Graissage centralisé},
							reference={main:graissageCentral}]


\subsection{Description du module de commande}

La \sdeux\ peut être équipée~— en option~— d’un dispositif de graissage\index{graissage centralisé}
centralisé\index{maintenance+graissage centralisé} qui alimente périodiquement chaque point de graissage du véhicule.

\startfigtext [left] [vogel_affichage] {Module d’affichage}
{\externalfigure[vogel_base2][W50]}
\blank
\startLeg
\item Afficheur à sept segments: valeurs et état de service
\item \LED, système en pause
\item \LED, pompe en fonctionnement
\item \LED, contrôle du système par commutateur de cycle
\item \LED, contrôle du système par manocontacteur
\item \LED, message d’erreur
\item Touches de défilement:
\startLeg [R]
\item activer l’afficheur
\item visualiser les valeurs
\item modifier les valeurs
\stopLeg
\item Touche de changement de mode, validation des valeurs
\item Déclenchement d’un cycle de graissage intermédiaire
\stopLeg
\stopfigtext

Le dispositif comprend une pompe, le réservoir de graisse transparent~— fixé sur le côté gauche du châssis~— et le module de commande, placé dans la centrale électrique.
\blank

\page [yes]


\subsubsubject {Affichage et touches du module de commande}

\start

\setupTABLE	[frame=off,style={\ssx\setupinterlinespace[line=.93\lH]},background=color,
			option=stretch,
			split=repeat]

\setupTABLE	[r]	[each]	[topframe=on,
						framecolor=white,
						% rulethickness=.8pt
						]

\setupTABLE	[c]	[odd]	[backgroundcolor=TableMiddle]
\setupTABLE	[c]	[even]	[backgroundcolor=TableLight]
\setupTABLE	[c]	[1]		[width=9mm,style={\bfx\setupinterlinespace[line=.93\lH]}]
\setupTABLE	[r] [first]	[topframe=off,style={\bfx\setupinterlinespace[line=.93\lH]},
						]
\setupTABLE	[r]	[2]		[style={\bfx\setupinterlinespace[line=.93\lH]}]


\bTABLE

\bTABLEhead
% \bTR [backgroundcolor=TableDark] \bTD [nc=4] Affichage et touches du module de commande \eTD\eTR
\bTR\bTD Pos. \eTD\bTD \LED \eTD\bTD Mode affichage \eTD\bTD Mode programmation \eTD\eTR
\eTABLEhead

\bTABLEbody
\bTR\bTD 2 \eTD\bTD État de service {\em pause} \eTD\bTD L’installation est en pause \eTD\bTD Le temps de pause peut être modifié \eTD\eTR
\bTR\bTD 3 \eTD\bTD État de service {\em contact} \eTD\bTD La pompe travaille \eTD\bTD Le temps de travail peut être modifié \eTD\eTR
\bTR\bTD 4 \eTD\bTD Système contrôle {\em CS} \eTD\bTD Par commutateur de cycle externe \eTD\bTD Le mode de contrôle peut être désactivé ou modifié \eTD\eTR
\bTR\bTD 5 \eTD\bTD Système contrôle {\em PS} \eTD\bTD Par manocontacteur externe \eTD\bTD Le mode de contrôle peut être désactivé ou modifié \eTD\eTR
\bTR\bTD 6 \eTD\bTD Dérangement {\em Fault} \eTD\bTD [nc=2] L’appareil est en dérangement. La cause peut être affichée sous forme de code d’erreur une fois que la touche \textSymb{vogel_DK} a été pressée. L’exécution des fonctions est arrêtée. \eTD\eTR
\bTR\bTD 7 \eTD\bTD Touches fléchées \textSymb{vogelTop} \textSymb{vogelBottom} \eTD\bTD [nc=2] \items[symbol=R]{Activation de l’écran d’affichage,Appel du paramètre suivant (mode affichage),Augmenter ou diminuer la valeur affichée de 1 (mode programmation)} \eTD\eTR
\bTR\bTD 8 \eTD\bTD Touche \textSymb{vogelSet} \eTD\bTD [nc=2] Commutation entre les modes d’affichage et de programmation, ou validation des valeurs saisies. \eTD\eTR
\bTR\bTD 9 \eTD\bTD Touche \textSymb{vogel_DK} \eTD\bTD [nc=2] Une pression lorsque l’appareil est à l’état \quote{pause} déclenche un cycle intermédiaire de graissage. Les messages d’erreur sont confirmés et effacés. \eTD\eTR
\eTABLEbody
\eTABLE
\stop
\vfill

\startfigtext [left] [vogel_touches]{Module d’affichage}
{\externalfigure[vogel_base][width=50mm]}
\textDescrHead{Mode affichage} Pressez brièvement l’une des touches fléchées \textSymb{vogelTop} ou \textSymb{vogelBottom} pour activer l’écran d’affichage à sept segments \textSymb{led_huit}. Une nouvelle pression sur la touche \textSymb{vogelTop} permet de faire défiler les différents paramètres et leur valeur. En mode {\em affichage}, les \LED sont allumées en continu (\in{pos. 2 à 6, fig.}[vogel_affichage]).
\blank [medium]
\textDescrHead{Mode programmation} Pour modifier les valeurs, appuyez environ 2\,s sur la touche \textSymb{vogelSet} pour passer en mode {\em programmation}: les \LED clignotent. Pressez brièvement la touche \textSymb{vogelSet} pour changer l’affichage, puis modifiez\index{graissage centralisé+programmation} la valeur souhaitée avec les touches \textSymb{vogelTop} et \textSymb{vogelBottom}. Validez\index{graissage centralisé+affichage} avec la touche \textSymb{vogelSet}.
\stopfigtext


\subsection{Sous||menus en mode {\em affichage}}

\vskip -9pt

\adaptlayout [height=+5mm]

\startcolumns[balance=no]\stdfontsemicn
\startSymVogel
\externalfigure[vogel_tpa][width=26mm]
\SymVogel
\textDescrHead{Temps de pause [h]} Pressez sur \textSymb{vogelTop} pour faire défiler toutes les valeurs programmées.
\stopSymVogel

\startSymVogel
\externalfigure[vogel_068][width=26mm]
\SymVogel
\textDescrHead{Temps de pause restant [h]} Affichage du temps restant avant le prochain cycle de graissage.
\stopSymVogel

\startSymVogel
\externalfigure[vogel_090][width=26mm]
\SymVogel
\textDescrHead{Temps de pause global [h]} Affichage du temps de pause total entre deux cycles de graissage.
\stopSymVogel

\startSymVogel
\externalfigure[vogel_tco][width=26mm]
\SymVogel
\textDescrHead{Temps de graissage [mn]} Pressez sur \textSymb{vogelTop} pour faire défiler les valeurs programmées.
\stopSymVogel

\startSymVogel
\externalfigure[vogel_tirets][width=26mm]
\SymVogel
\textDescrHead{Appareil en pause} Affichage impossible, l’appareil est en pause.
\stopSymVogel

\startSymVogel
\externalfigure[vogel_026][width=26mm]
\SymVogel
\textDescrHead{Temps de graissage [mn]} Affichage du temps de fonctionnement de la pompe de graissage.
\stopSymVogel

\startSymVogel
\externalfigure[vogel_cop][width=26mm]
\SymVogel
\textDescrHead{Contrôle système} Pressez sur \textSymb{vogelTop} pour faire défiler les valeurs programmées.
\stopSymVogel

\startSymVogel
\externalfigure[vogel_off][width=26mm]
\SymVogel
\textDescrHead{Valeur contrôle système} PS\:= manocontacteur, CS\:= commutateur de cycle, OFF\:= désactivé.
\stopSymVogel

\startSymVogel
\externalfigure[vogel_0h][width=26mm]
\SymVogel
\textDescrHead{Heures de service} Pressez sur la touche \textSymb{vogelTop} pour afficher
la valeur en deux temps (voir ci-dessous).
\stopSymVogel

\startSymVogel
\externalfigure[vogel_005][width=26mm]
\SymVogel
\textDescrHead{Première partie: 005} Le temps de service s’affiche en deux parties, affichez la suite avec \textSymb{vogelTop}.
\stopSymVogel

\startSymVogel
\externalfigure[vogel_338][width=26mm]
\SymVogel
\textDescrHead{Deuxième partie: 33.8} La suite du nombre est 33.8. L’assemblage du nombre d’heures donne 533.8\,h.
\stopSymVogel

\startSymVogel
\externalfigure[vogel_fh][width=26mm]
\SymVogel
\textDescrHead{Heures de dysfonctionnement} Pressez sur la touche \textSymb{vogelTop} pour afficher
la valeur en deux temps (voir ci-dessous).
\stopSymVogel

\startSymVogel
\externalfigure[vogel_000][width=26mm]
\SymVogel
\textDescrHead{Première partie: 000} Le temps de dysfonctionnement s’affiche en deux parties, affichez la suite avec \textSymb{vogelTop}.
\stopSymVogel

\startSymVogel
\externalfigure[vogel_338][width=26mm]
\SymVogel
\textDescrHead{Deuxième partie: 33.8} La suite du nombre est 33.8. L’assemblage du nombre donne le temps 33.8\,h.
\stopSymVogel

\stopcolumns

\stopsection

\page [yes]

\stopmode % central lubrication

\setups [pagestyle:marginless]


\startsection	[title={Plan de graissage manuel},
				 reference={sec:grasing:plan}]

\starttextbackground [FC]
\startPictPar
\PMgeneric
\PictPar
Graissez régulièrement les éléments présentés sur le plan de graissage (\in{fig.}[fig:greasing:plan]).
Un graissage régulier est indispensable\index{maintenance+graissage manuel}
pour maintenir un {\md film antifriction} permanent et éliminer {\md l’eau et les agents corrosifs} parasites.
\stopPictPar
\stoptextbackground

\blank [big]

\start

\setupcombinations [width=\textwidth]

\placefig[here][fig:greasing:plan]{Plan de graissage du véhicule}
{\startcombination [3*1]
{\externalfigure[frame:steering:greasing]}{\ssx Système oscillant articulé}
{\externalfigure[frame:axles:greasing]}{\ssx Graissage des essieux}
{\externalfigure[frame:sucMouth:greasing]}{\ssx Bouche d’aspiration}
\stopcombination}

\stop

\vfill

\startLeg [columns,three]
\item Vérin de direction articulée\crlf {\sl 2 graisseurs sur chaque vérin}
\item Palier de direction articulée\crlf {\sl 2 graisseur côté gauche}
\item Palier du système oscillant\crlf {\sl 1 graisseur devant le réservoir}
\item Ressorts à lame\crlf {\sl 2 graisseurs par lame}
\item Bouche d’aspiration\crlf {\sl 1 graisseur par roue}
\item Bouche d’aspiration\crlf {\sl 1 graisseur sur le bras de traction}
\stopLeg

\page [yes]


\setups [pagestyle:bigmargin]

\subsubject{Graissage de la cuve à déchet}


La cuve à déchets comprend 8 points de graissage (2\:×\:4) à graisser chaque semaine de travail.

\blank [big]


\placefig[here][fig:greasing:container]{Mécanisme de levage de la cuve}
{\externalfigure[container:mechanisme]}


\placelegende [margin,none]{}
{{\sla Légendes:}

\startLeg
\item Palier de cuve gauche (2\:×)
\item Palier de cuve droit (2\:×)
\item Vérin hydraulique gauche (tête)
\item Vérin hydraulique gauche (base)

{\em Idem côté droit (voir point \in[greasing:point;hide]).}
\item Vérin hydraulique droit (tête)
\item [greasing:point;hide] Vérin hydraulique droit (base)
\stopLeg}

\stopsection

\page [yes]


\startsection[title={Installation électrique},
							reference={sec:main:electric}]

\subsection{Centrale électrique châssis}

\startbuffer [fuses:preventive]
\starttextbackground [CB]
\startPictPar
\PHvoltage
\PictPar
\textDescrHead{Mesure préventive} Observez\index{maintenance+installation électrique} toutes les mesures préventives indiquées dans ce manuel;
remplacez les fusibles en respectant l’ampérage prescrit\index{fusibles+châssis}\index{relais+châssis}.
N’intervenez pas sur l’installation électrique\index{installation électrique} si vous portez des bijoux (bagues, bracelets, pendentifs, etc.).
\stopPictPar
\stoptextbackground
\stopbuffer


\subsubsubject{Fusibles \cap{midi}}


\starttabulate[|l|l|p|]
\HL
\NC\md F\,1 \NC 5\,A  \NC Feu de stop, \quote{+\:15} OBD \NC\NR
\NC\md F\,2 \NC 5\,A  \NC \quote{+\:15} Gestion moteur \NC\NR
\NC\md F\,3 \NC 7,5\,A \NC \quote{+\:30} Gestion moteur et OBD \NC\NR
\NC\md F\,4 \NC 20\,A \NC Pompe d’alimentation en carburant \NC\NR
\NC\md F\,5 \NC 20\,A \NC \quote{D\:+} alternateur, \quote{+\:15} relai K\,1 \NC\NR
\NC\md F\,6 \NC 5\,A \NC Gestion moteur \NC\NR
\NC\md F\,7 \NC 10\,A\NC Dépollution moteur \NC\NR
\NC\md F\,8 \NC 20\,A \NC Électronique du moteur (gestion) \NC\NR
\NC\md F\,9 \NC 15\,A \NC Dépollution moteur , pompe d'alimentation, préchauffage \NC\NR
\NC\md F\,10\NC 30\,A \NC Gestion moteur \NC\NR
\NC\md F\,11\NC 5\,A \NC Feux de marche arrière \NC\NR
\HL
\stoptabulate

\placefig [margin] [fig:electric:power:rear] {Centrale électrique châssis}
{\externalfigure [electric:power:rear]
\noteF
\startKleg
\sym{K\,1} Unité de commande électronique du moteur
\sym{K\,2} Pompe d’alimentation en carburant
\sym{K\,3} Autorisation fonction démarreur
\sym{K\,4} Feux de stop
\sym{K\,5} {[}Réserve{]}
\sym{K\,6} Feux de marche arrière
\sym{K\,7} Préchauffage du moteur
\stopKleg
}


\subsubsubject{Fusibles \cap{maxi}}

% \startcolumns [n=2]
\starttabulate[|l|l|p|]
\HL
\NC\md F\,15 \NC 50\,A \NC Alimentation principale de la centrale électrique \NC\NR
\HL
\stoptabulate

\page [yes]

\setups[pagestyle:marginless]


\subsection{Centrale électrique dans la cabine}


\startcolumns[rule=on]

\placefig [bottom] [fig:fuse:cab] {Fusibles et relais dans la cabine}
{\externalfigure [electric:power:front]}

\columnbreak

\subsubsubject{Relais}

\vskip -12pt

\index{fusibles+cabine}\index{relais+cabine}

\starttabulate[|lB|p|]
\NC K\,2 	\NC Compresseur A/C\NC\NR
\NC K\,3 	\NC Compresseur A/C\NC\NR
\NC K\,4 	\NC Pompe à eau électrique\NC\NR
\NC K\,5 	\NC Gyrophare\NC\NR
\NC K\,10 \NC Boite clignotante\NC\NR
\NC K\,11 \NC Feux de croisement\NC\NR
\NC K\,12 \NC Feux de route {[}Réserve{]}\NC\NR
\NC K\,13 \NC Projecteurs de travail\NC\NR
\NC K\,14 \NC Essuie-glace intermittent\NC\NR
\stoptabulate

\vskip -24pt

\placefig [bottom] [fig:fuse:access] {Porte d’accès à la centrale}
{\externalfigure [electric:power:cabin]}


\stopcolumns

\page [yes]


\subsubsubject{Fusibles \cap{mini}}


\startcolumns[rule=on]
% \setuptabulate[frame=on]
%\placetable[here][tab:fuses:cab]{Fusibles dans la cabine}
%{\noteF
\starttabulate[|lB|l|p|]
\NC F\,1  \NC 3\,A \NC Feux de position côté gauche \NC\NR
\NC F\,2  \NC 3\,A \NC Feux de position côté droit \NC\NR
\NC F\,3  \NC 7,5\,A \NC Feux de croisement côté gauche \NC\NR
\NC F\,4  \NC 7,5\,A \NC Feux de croisement côté droit \NC\NR
\NC F\,5  \NC 7,5\,A \NC Feux de route côté gauche {[}Réserve{]}\NC\NR
\NC F\,6  \NC 7,5\,A \NC Feux de route côté droit {[}Réserve{]}\NC\NR
\NC F\,7  \NC 10\,A \NC Projecteurs de travail haut \NC\NR
\NC F\,8  \NC 10\,A \NC Projecteurs de travail bas {[}Réserve{]} \NC\NR
\NC F\,9  \NC 10\,A \NC Balai frontal \NC\NR
\NC F\,10 \NC 10\,A \NC Essuie||glace \NC\NR
\NC F\,11 \NC 5\,A \NC Commandes d’éclairage et feux de détresse \NC\NR
\NC F\,12 \NC 5\,A \NC {[}Réserve{]} \NC\NR
\NC F\,13 \NC 10\,A \NC Rétroviseurs chauffants \NC\NR
\NC F\,14 \NC 7,5\,A \NC +\:15 radio et caméra \NC\NR
\NC F\,15 \NC 10\,A \NC +\:30 feux de détresse \NC\NR
\NC F\,16 \NC 5\,A \NC Interrupteur d'éclairage colonne direction \NC\NR
\NC F\,17 \NC 7,5\,A \NC +\:30 radio et éclairage intérieur \NC\NR
\NC F\,18 \NC — \NC {[}Libre{]} \NC\NR
\NC F\,19 \NC 20\,A \NC +\:30 RC 12 avant \NC\NR
\NC F\,20 \NC 20\,A \NC +\:30 RC 12 arrière \NC\NR
\NC F\,21 \NC 15\,A \NC Prise 12\,V \NC\NR
\NC F\,22 \NC 5\,A \NC Clé de contact, console multifonction, Vpad \NC\NR
\NC F\,23 \NC 5\,A \NC Arrête d’urgence, console centrale, RC 12 avant \NC\NR
\NC F\,24 \NC 5\,A \NC Arrête d’urgence, console centrale, RC 12 arrière \NC\NR
\NC F\,25 \NC 2\,A \NC +\:15 RC 12 avant \NC\NR
\NC F\,26 \NC 2\,A \NC +\:15 RC 12 arrière \NC\NR
\NC F\,27 \NC 25\,A \NC Ventilateur de chauffage \NC\NR
\NC F\,28 \NC 10\,A \NC Compresseur A/C, graissage central \NC\NR
\NC F\,29 \NC 25\,A \NC Condenseur A/C \NC\NR
\NC F\,30 \NC 5\,A \NC Thermostat A/C \NC\NR
\NC F\,31 \NC 5\,A \NC +\:15 console multifonction|/|Vpad \NC\NR
\NC F\,32 \NC 15\,A \NC Pompe à eau élec., gyrophare \NC\NR
\NC F\,33 \NC — \NC {[}Libre{]} \NC\NR
\NC F\,34 \NC — \NC {[}Libre{]} \NC\NR
\NC F\,35 \NC — \NC {[}Libre{]} \NC\NR
\NC F\,36 \NC — \NC {[}Libre{]} \NC\NR
\stoptabulate
\stopcolumns

\page [yes]

\setups [pagestyle:bigmargin]


\subsection[sec:lighting]{Éclairage et signalisation}


\placefig [here] [fig:lighting] {Éclairage et signalisation du véhicule}
{\externalfigure [vhc:electric:lighting]}

\placelegende [margin,none]{}{%
\vskip 30pt
{\sla Légende:}
\startLongleg
\item Feux de position\hfill 12\,V-5\,W
\item Feux de croisement\hfill H7~12\,V-55\,W
\item Feux clignotants\hfill orange 12\,V-21\,W
\item Proj. de travail\hfill G886~12\,V-55\,W
\item Feux clignotants\hfill 12\,V-21\,W
\item Feux de position|/|stop\hfill 12\,V-5|/|21\,W
\item Feux de marche arrière\hfill 12\,V-21\,W
\item {[}Libre{]}
\item Feux de plaque\hfill 12\,V-5\,W
\item Gyrophare\hfill H1~12\,V-55\,W
\stopLongleg}

\subsubsubject{Réglage du faisceau lumineux}

\placefig [margin] [fig:lighting:adjustment] {Projection du faisceau à 5\,m}
{\externalfigure [vhc:lighting:adjustment]
\startitemize
\sym{H\low{1}} Hauteur du filament: 100\,cm
\sym{H\low{2}} Correction à 2\hairspace\%: 10\,cm
\stopitemize}

{\md Conditions:} réservoirs d’eau claire et recyclage pleins, conducteur au volant.

L’orientation du faisceau lumineux est réglé en usine. La hauteur et l’inclinaison du faisceau sont modifiables
en pivotant le support plastique.

Si lors d’un contrôle un ajustement s’avère nécessaire, débloquez la vis de blocage et ajustez l’inclinaison
de manière à ce qu’elle corresponde aux exigences légales (voir \in{fig.}[fig:lighting:adjustment]); serrez
la vis de blocage.




\page [yes]

\setups [pagestyle:marginless]

\subsection[sec:battcheck]{Batterie}

\subsubsection{Prescriptions particulières}

\startSymList
\PPfire
\SymList
\textDescrHead{Danger d’explosion}
Lors\index{Batterie+prescription de sécurité}\index{danger+explosion} de la charge de la batterie, il se dégage un gaz
explosif au pôle\index{gaz explosif} négatif. Chargez la batterie dans un local bien aéré! Ne provoquez pas d’étincelles! N’approchez pas de la batterie avec une flamme nue, ne fumez pas.
\stopSymList

\startSymList
\PHvoltage
\SymList
\textDescrHead{Risque de court||circuit}
La \index{batterie+maintenance}\index{maintenance+batterie} cosse positive
de la batterie encore branchée risque de provoquer un court||circuit si elle entre
en contact avec la carrosserie du véhicule. Le mélange gazeux qui se dégage peut
facilement\index{risque+incendie}\index{risque+explosion} exploser. Vous ou d’autres personnes pourriez être blessés.

\startitemize
\item Ne posez pas d’objets métalliques ni d’outils sur la batterie.
\item Lorsque vous débranchez une batterie, commencez toujours par
débrancher la cosse négative. Débranchez ensuite la cosse positive.
\item Lorsque vous branchez une batterie, commencez toujours par brancher la
cosse positive. Branchez ensuite la cosse négative.
\item Ne desserrez ni ne débranchez les cosses de la batterie lorsque le moteur tourne.
\stopitemize
\stopSymList


\startSymList
\PHcorrosive
\SymList
\textDescrHead{Risque de brûlure}
Portez des \index{risque+brûlure}lunettes et des gants de protection résistants aux acides.
L’électrolyte contenu dans la batterie est composé d’environ 27\,\%\ d’acide sulfurique (H\low{2}SO\low{4})
pouvant \index{risque+brûlure}provoquer des brûlures. Neutraliser\index{batterie+danger}\index{batterie+électrolyte}
toute solution d’acide sulfurique déversée ou projetée avec une solution de bicarbonate de soude,
et rincer la partie du corps touchée avec de l’eau propre. En cas de projection, rincez immédiatement les yeux avec de l’eau claire et consultez un médecin!
\stopSymList

\page [yes]


\startSymList
\startcombination[1*2]
{\PHcorrosive}{}
{\PHfire}{}
\stopcombination
\SymList
\textDescrHead{Entreposage de la batterie}
Entreposez toujours la batterie avec \index{batterie+dépôt}les bornes vers le haut:
l’électrolyte contenu dans la batterie peut s’échapper et provoquer des brûlures,
ou en interaction avec d’autres substances provoquer un incendie.\par\null\par\null
\stopSymList

\starttextbackground [FC]
\setupparagraphs [PictPar][1][width=2.4em,inner=\hfill]

\startPictPar
\PMproteyes
\PictPar
\textDescrHead{Lunettes de protection}
Lorsque \index{risque+projections}\index{risque+accident oculaire}vous mélangez l’eau et l’acide, du liquide peut gicler et atteindre vos yeux. En cas de projection, rincez immédiatement les yeux avec de l’eau claire et consultez immédiatement un médecin!
\stopPictPar
\blank [small]

\startPictPar
\PMrtfm
\PictPar
\textDescrHead{Documentation}
Manipulez la batterie en tenant compte des consignes de sécurité,
des mesures de protection et des procédures qui figurent dans les
présentes instructions de service.
\stopPictPar
\blank [small]

\startPictPar
\PStrash
\PictPar
\textDescrHead{Protection de l’environnement}
La batterie\index{environnement+protection} contient des
substances polluantes. Ne jetez pas la batterie usagée dans les ordures
ménagères. Déposez la batterie dans un atelier qualifié, par
exemple dans un point de récupération pour batteries usagées.\par

Transportez et entreposez la batterie en position verticale quand elle
est remplie. Veillez à ce que la batterie soit correctement arrimée durant son transport,
sans quoi l’électrolyte pourrait s’écouler
par les orifices de ventilation qui se trouvent sur les bouchons et
provoquer une pollution de l’environnement.
\stopPictPar
\stoptextbackground

\page [yes]

\setups[pagestyle:normal]


\subsubsection{Conseils pratiques}

Afin de garantir une durée de vie élevée, la batterie doit rester chargée.

Le \index{batterie+durée de vie}maintien de la charge de la batterie durant
les périodes d’immobilisation prolongées garantit un démarrage sans
problème.

\placefig[margin][fig:batterycompartment]{\select{caption}{Compartiment de la
batterie}{Compartiment de batterie}}
{\externalfigure[batt:compartment]}

\subsubsection{Entretien}

La batterie (accumulateur au plomb) de la \sdeux est du type {\em sans entretien}. Hormis le maintien de l’état de charge
et de propreté, la batterie ne nécessite aucun entretien particulier.

\startitemize
\item Veillez à ce que les bornes de la batterie soient toujours propres et sèches; enduisez légèrement les bornes de graisse anti||acide.
\item Rechargez\index{batterie+charge} la batterie avec une tension de repos\index{batterie+tension de repos} inférieure à 12,4 volts.
\stopitemize

\placefig[margin][fig:bclean]{Nettoyer les pôles de contact}
{\externalfigure[batt:clean]
\noteF
Utilisez\index{batterie+nettoyage}\index{nettoyage+batterie} de l’eau tiède pour éliminer la poudre blanche synonyme de corrosion.
Si une borne est très corrodée, débranchez la batterie et nettoyez les bornes avec une brosse métallique.
Appliquez une fine couche de graisse sur les bornes après le nettoyage.}


\subsubsection[sec:battery:switch]{Fonctionnement du robinet coupe||circuit}

\startSteps
\item Coupez\index{robinet coupe||circuit} le contact, puis attendez environ 1 minute.
\item Ouvrez le compartiment de la batterie (\inF[fig:batterycompartment]).
\item Appuyez sur la tête rouge du robinet coupe||circuit\index{coupe||batterie} pour découpler la batterie.
\item Pivotez le robinet ¼ de tour dans le sens horaire pour fermer le circuit.
\stopSteps

\starttextbackground [FC]
\startPictPar
\PMgeneric
\PictPar
Le robinet coupe||circuit est prévu pour découpler la batterie durant certains travaux d’entretien et de réparation.
Il n’est pas recommandé de découpler la batterie quotidiennement: certains composants électroniques nécessitent
une mise sous tension en permanence, sans quoi ils peuvent générés des codes d’erreurs.
\stopPictPar
\stoptextbackground

\stopsection

\page [yes]

%%%%%%%%%%%%%%%%%%%%%%%%%%%%%%%%%%%%%%%%%%%%

\setups[pagestyle:marginless]


\startsection [title={Lavage du véhicule},
				reference={sec:cleaning}]

Éliminez\startregister[index][vhc:lavage]{maintenance+lavage} les grosses saletés de la carrosserie avec de l’eau avant le nettoyage proprement dit.
Ne lavez pas que les flancs de la carrosserie, mais également les passages de roues et le dessous du châssis.\par

En hiver, votre machine de travail \boschung\ doit être lavée en profondeur,
afin d’éliminer toute substance\index{corrosion+prévention} hautement corrosive comme le sel d’épandage.

\starttextbackground [FC]
\startPictPar
\PHgeneric
\PictPar
\textDescrHead{Prévention contre les dégâts d’eau}
Ne lavez jamais votre machine de travail \boschung\ avec un {\em canon à eau} (\eG\ véhicule d’intervention anti||incendie) ou avec des produits {\em à base d’hydrocarbure}. Si vous utilisez une installation de lavage à la vapeur sous pression, veuillez respecter les prescriptions mentionnées plus loin (encadré) dans ce chapitre.
\stopPictPar\vskip 3pt

\startPictPar
\PSwelt
\PictPar
\textDescrHead{Protection de l’environnement}
Le lavage d’un véhicule peut provoquer une pollution aux hydrocarbures. Ne lavez votre machine de travail \boschung
que sur une place équipée d’un séparateur d’huile. Respectez\index{lavage+protection de l’environnement} les normes environnementales en vigueur.
\stopPictPar\vskip 3pt

\startPictPar
\PMwarranty
\PictPar
\textDescrHead{Risque de perte de la garantie}
Tout dommage causé par le non||respect des prescriptions de lavage citées dans ce chapitre ne bénéficiera d’aucune complaisance ou prise en charge par la garantie de la part de \BosFull{boschung}.
\stopPictPar
\stoptextbackground


\subsection{Lavage au jet sous pression}

Vous pouvez utiliser un appareil usuel\index{lavage+jet sous pression} pour laver votre machine avec un jet sous pression.

Lors du lavage au jet, veuillez respecter les points suivants:


\startitemize
	\item Pression de travail max. 50\,bar
	\item Direction du jet avec un angle de 25°
	\item Distance de la buse min. 80\,cm
	\item Température d’eau max. 40\,°C
	\item Référez||vous à la section \about[reiMi] \at{page}[reiMi].
\stopitemize

Le non||respect de ces prescriptions peut provoquer des dégâts à la peinture et à la couche\index{corrosion+dégâts} de protection\index{peinture+dégâts} anticorrosion.

Considérez les prescriptions du fabricant de l’appareil de lavage.


\blank[big]
% \setupnarrower[left=.4em,right=.4em,middle=.6em]
% \startnarrower
\starttextbackground[FC]
\startPictPar
\PPspray
\PictPar
Le lavage au jet sous pression présente des risques de pénétration d’eau non souhaitée aux endroits sensibles.\par

Ne dirigez jamais le jet sous pression en direction des appareils et autres éléments mentionnés ci||après:
\stopPictPar

\startitemize
	\item Capteurs, prises et connexions électriques
	\item Démarreur, alternateur, système d’injection
	\item Électrovannes
	\item Ouvertures d’aération
	\item Composants mécaniques encore chauds
	\item Autocollants de signalisation des dangers
	\item Boîtiers de commande électroniques
\stopitemize

\textDescrHead{Lavage du moteur}
Évitez les entrées d’eau dans les conduits d’admission, d’aération et de ventilation.
Ne pas diriger le jet de vapeur ou pression directement sur les composants électriques ou sur les connexions.
Ne jamais diriger le jet sur les composants du système d’injection! Après le lavage du moteur,
appliquez un produit de conservation; protégez la courroie avant d’appliquer le produit de conservation.
\stoptextbackground

\page [yes]


\starttextbackground [FC]
\setupparagraphs [PictPar][1][width=6em,inner=\hfill]
\startPictPar
\framed[frame=off,offset=none]{\PMproteyes~\PMprotears}
\PictPar
\textDescrHead{Séchage de l’eau résiduelle}
Lors du lavage, de l’eau s’est accumulées dans certains points de la machine (\eG\ dans les cavités du bloc||moteur ou de la boite de vitesse), qu’il faut éliminer avec de l’air comprimé. L’emploi de l’air comprimé requiert le port d’équipement de protection individuelle et d’une installation conforme (soufflette à buse multi||trous).
\stopPictPar
\stoptextbackground


\subsubsection[reiMi]{Produits de nettoyage appropriés}

Choisissez\index{produits de nettoyage} un produit de nettoyage en tenant compte des caractéristiques suivantes:

\startitemize
	\item Sans abrasif
	\item Valeur PH comprise entre 6 et 7
	\item Exempt de solvant
\stopitemize


Pour éliminer les taches tenaces, il est possible d’utiliser, avec prudence et sur de petites surfaces de carrosserie, de l’essence ou de l’alcool. N’utilisez aucun autre solvant. Il faut ensuite éliminer immédiatement les solvants résiduels de la peinture.

Nettoyez les surfaces avec étiquettes\index{lavage+autocollants} autocollantes signalant des dangers particuliers ou comprenant des instructions d’utilisation avec de l’eau claire et une éponge douce.

Évitez les pénétrations d’eau dans les composants électriques: ne lavez pas les blocs de feux avec une brosse de lavage automatique, mais avec une éponge douce ou un chiffon.

\starttextbackground [CB]
\setupparagraphs [PictPar][1][width=6em,inner=\hfill]
\startPictPar
\framed[frame=off,offset=none]{\GHSgeneric~\GHSfire}
\PictPar
\textDescrHead{Danger relatif à l’utilisation de produits chimiques} Les produits de nettoyage peuvent présenter un risque pour la santé ou la sécurité (produits extrêmement inflammable). Veuillez respecter les règles de sécurité applicables aux produits de nettoyage utilisés (consultez si besoin la fiches des données de sécurité (FDS) relative au produit).
\stopPictPar
\stoptextbackground

\page [yes]

% \setups[pagestyle:normal]


\stopregister[index][vhc:lavage]

\stopsection

\page [yes]


\setups [pagestyle:bigmargin]

\startsection	[title={Réglage de la bouche d'aspiration},
				 reference={sec:main:suctionMouth}]


L'espace optimal\index{bouche d'aspiration+réglage} entre la chaussée et la bande d'usure en caoutchouc de la bouche d'aspiration est de 10\,mm.
Pour contrôler|/|ajuster cette valeur, utilisez les trois jauges qui se trouvent dans la boite à outil
(dans la cabine, côté conducteur).


\placefig [margin] [fig:suctionMouth] {Réglage de la bouche d'aspiration}
{\Framed{\externalfigure [suctionMouth:adjust]}}

\placeNote[][service_picto]{}{%
\noteF
\starttextrule{\PHasphyxie\enskip Risque d'intoxication et d'asphyxie \enskip}
{\md Note:} il est nécessaire de laisser tourner le moteur durant la procédure de réglage
pour maintenir la bouche d'aspiration en position flottante. Pour prévenir les risques d'intoxication ou
d'asphyxie, il est impératif de placer un système d'aspiration des gaz d'échappement et d'exécuter les travaux
dans un endroit bien aéré.
\stoptextrule}

\startSteps
\item Placez le véhicule sur un sol plat et horizontal dans un endroit bien aéré.
\item Activez\index{aspiration} le mode \quote{travail} (pressez sur le bouton à l’extrémité du sélecteur de marche).

Laissez tourner le moteur au ralenti (pressez le bouton \textSymb{joy_key_engine_decrease}
sur la console de commande pour diminuer le régime du moteur).
\item Tirez le frein à main et placez une cale de part et d'autre de chacune des roues arrières.
\item Pressez la touche \textSymb{joy_key_suction} pour abaisser la bouche d'aspiration.
\item Placez les trois jauges \LAa\ sous la bande d'usure en caoutchouc de la bouche d'aspiration selon
l'illustration ci||contre.
\item [sucMouth:adjust]Débloquez les vis de fixation \Lone\ et de réglage \Ltwo\ de chaque roue;
les quatre roues reposent sur le sol.
\item Bloquez les vis \Lone\ et \Ltwo, puis retirez les trois jauges.
\item Levez|/|abaissez la bouche d'aspiration et vérifiez le réglage avec les jauges, le cas échéant recommencez la procédure au point \in[sucMouth:adjust].

\stopSteps


\stopsection

\stopregister[index][maintenance:s2]
\stopchapter

\stopcomponent

