\startcomponent c_90_serviceplan_s2_120-fr


\chapter[sec:schedule]{Plan d’entretien}


\startregister[index][wartplan]{plan d’entretien+véhicule}

\section{Généralités}


\placeNote[][service_picto]{}{%
\noteF
\starttextrule{\Pmtcheck \enskip Qualification {\em contrôle} \enskip}
Contrôles pouvant être effectués par une\index{calification+personnel de maintenance} personne du service d’exploitation sans aide externe, à condition de posséder les connaissance de base en mécanique automobile.
\stoptextrule \blank[line]
\starttextrule{\Pmtpro \enskip Qualification {\em entretien} \enskip}
Travaux d’entretien devant être effectués dans le cadre du programme de maintenance.
L’entretien doit être effectué dans un atelier agréé disposant d’une main-d’oeuvre qualifiée et l’outillage nécessaire
\stoptextrule \blank[line]
\starttextrule{\Pmtspecial \enskip Qualification {\em travaux spéciaux} \enskip}
Travaux d’entretien particuliers\index{travaux spéciaux} ne pouvant être effectués que par une personne ayant reçu une formation spécifique reconnue par le service après||vente \BoschungNote.
\stoptextrule \blank[line]
\starttextrule{\Pmtvisual \enskip Type {\em Contrôle visuel} \enskip}
Point de contrôle à effectuer par un\index{contrôle visuel} contrôle visuel. Les éventuels dommages ou défauts de fonctionnement sont à communiquer à la personne compétente.
\stoptextrule
}

Lors des travaux d’entretien, il faut distinguer trois échelons de qualification (personnel|/|équipement) et quatre types d’entretien (type de travail).

\startTwoPar
{\em Échelons de qualification:}	\par \blank [medium]
\start \setupwhitespace	[none]
\symDescr{\textSymb{maint:check}} Contrôle (év. correction)\par
\symDescr{\textSymb{maint:pro}} Travaux de service\par
\symDescr{\textSymb{maint:special}} Travaux particuliers\par
\stop
\TwoPar
{\em Types d’entretien:}\par \blank [medium]
\start \setupwhitespace	[none]
\symDescr{\textSymb{maint:visual}} Contrôle visuel\par
\symDescr{\textSymb{maint:function}} Contrôle de fonctionnement\par
\symDescr{\textSymb{maint:level}} Contrôle|/|remplissage des niveaux\par
\symDescr{\textSymb{maint:exchange}}Vidange de liquide\par
\symDescr{\textSymb{maint:generic}}Autres activités\par
\stop
\stopTwoPar

\handItem{Consultez les explications dans les colonnes périphériques.}


\placeNote[][service_picto]{}{%
 \noteF
	\starttextrule{\Pmtfunction \enskip Contrlôle de fonctionnement \enskip}
		Contrôler\index{contrôle+fonctionnement}, par contrôle visuel à l’extérieur, \eG\ le fonctionnement des freins,
		l’état des composants, contrôler que tout soit bien fixé.
	\stoptextrule \blank[big]
	\starttextrule{\Pmtlevel \enskip Contrlôle de niveau \enskip}
		Le\index{contrôle+niveaux} contrôle de niveau d’un élément comprend le contrôle~—
		visuel ou avec la jauge~— ainsi que le complètement avec le liquide requis, {\em y compris le graissage central}.
		Vous trouverez les informations nécessaires concernant les quantités et qualités requises \atpage[sec:liqquantities].
	\stoptextrule \blank[big]
	\starttextrule{\Pmtexchange \enskip Vidange \enskip}
		La vidange\index{vidange} d’un élément comprend le remplacement du liquide ainsi que le contrôle du niveau.
		Vous trouverez les informations nécessaires concernant les quantités et qualités requises \atpage[sec:liqquantities].
	\stoptextrule \blank[big]
	\starttextrule{\Pmtgeneric \enskip Divers \enskip}
		Veuillez effectuer chaque opération d’entretien conformément aux instructions.
	\stoptextrule
}%

\blank[big]
\starttextbackground[FC]
\setupparagraphs [PictPar][1][width=6em,inner=\hfill]
\startPictPar\PMgeneric~\Penvironment\PictPar
Veuillez considérer les prescriptions de sécurité et de protection de l’environnement lorsque vous entreprenez des travaux relatifs au plan de maintenance.
\BosFull{boschung} ne répond pas des dommages matériels ou sur des personnes consécutifs au non respect des prescriptions.
\stopPictPar
\stoptextbackground

\page [yes]


%%%%%%%%%%%%%%%%%%%%%%%%%%%%%%%%%%%%%%%%%%%%%%%%%%%%%%%%%%%%%%%%%%%%%%%%%%%%%%
\section{Plan d’entretien du véhicule}

La maintenance du véhicule comprend les travaux d’entretien réguliers\index{contrôle périodique}, de contrôles périodiques (contrôles), ainsi qu’un premier service unique\footnote{Toutes les échéances de maintenance sont indiquées en heures de service.}:

\starttextbackground[FC]
 \starttabulate [|w(39mm)B|w(17mm)B|p({\dimexpr\textwidth-(56mm+2.5em)\relax})|]
 \NC \Pmtpro\enskip\bf Travaux d’entretien \NC\bf 50\,h \NC Premier service après 50\,h\NC \NR
 \NC \NC\bf 600\,h \NC Entretien régulier chaque 600\,h|/|12 mois \NC \NR
 \NC \NC\bf 1'200\,h \NC Entretien régulier chaque 1'200\,h|/|2 ans \NC \NR
 \NC \NC\bf 2'400\,h \NC Entretien régulier chaque 2'400\,h|/|4 ans \NC \NR
 \NC \NC\bf 4'800\,h \NC Entretien régulier chaque 4'800\,h|/|8 ans \NC \NR
 \NC \Pmtcheck\enskip\bf Liste de contrôle \NC Quotidien \NC Durant toute la saison de travail \NC\NR
 \stoptabulate
\stoptextbackground
\blank [big]

Le plan d’entretien ci||après se réfère au véhicule de base. Considérez également le plan d’entretien de chaque agrégat
(\eG\ unité de balayage, cuve à déchets) dès \at{la page}[sec:schedaggr].


La priorité dans le système de double échéance des intervalles de maintenance (\eG\ \quote{chaque 1'000\,h ou 2~ans}) est donnée à la première échéance atteinte.

Le plan d’entretien~— excepté le premier service unique après 50\,h~— est cumulatif:
le service 1'200\,h comprend le service des 600\,h {\em et} le service des 1'200\,h;
le service des 2'400\,h comprend les services des 600\,h {\em et} le service des 1'200\,h
{\em et} le service des 2'400\,h; etc.

\page[yes]

\setup[pagestyle:marginless]

\start
\setup[tbl:schedule]

\subsection[table:scheduledaily]{Contrôle quotidien}

\bTABLE
\bTABLEhead
\bTR \bTD Type \eTD \bTD Contrôle quotidien \eTD \bTD \Tcheck \eTD \bTD Ref. \eTD \eTR
\eTABLEhead
\bTABLEbody
	\bTR \bTD \Tgen \eTD \bTD Nettoyez le véhicule \eTD \bTD \Tcheck \eTD \bTD \inP[sec:cleaning] \eTD \eTR
	\bTR \bTD \Tvis \eTD \bTD Contrôlez les dommages éventuels au véhicule \eTD \bTD \Tcheck \eTD \bTD \emptY \eTD \eTR
	\bTR \bTD \Tvis \eTD \bTD Contrôlez les fuites d’huile \eTD \bTD \Tcheck \eTD \bTD \emptY \eTD \eTR
	\bTR \bTD \Tlev \eTD \bTD Contrôlez le niveau d’huile moteur (avec la jauge!) \eTD \bTD \Tcheck \eTD \bTD \inP[ssSec:vw:oilLevel] \eTD \eTR
	\bTR \bTD \Tlev \eTD \bTD Contrôlez le niveau du liquide de refroidissement du moteur \eTD \bTD \Tcheck \eTD \bTD \inP[sSec:vw:cooling] \eTD \eTR
	\bTR \bTD \Tlev \eTD \bTD Contrôlez le niveau hydraulique (regard sur le réservoir) \eTD \bTD \Tcheck \eTD \bTD \inP[sec:hydraulic] \eTD \eTR
	\bTR \bTD \Tlev \eTD \bTD Contrôlez le niveau du carburant \eTD \bTD \Tcheck \eTD \bTD \emptY \eTD \eTR
	\bTR \bTD \Tlev \eTD \bTD Contrôlez le niveau du lave||glace \eTD \bTD \Tcheck \eTD \bTD \inP[sec:liqquantities] \eTD \eTR
	\bTR \bTD \Tfun \eTD \bTD Contrôlez le fonctionnement des témoins lumineux et de l’éclairage des instruments au tableau de bord \eTD \bTD \Tcheck \eTD \bTD \emptY \eTD \eTR
	\bTR \bTD \Tfun \eTD \bTD Contrôlez le fonctionnement du frein de stationnement \eTD \bTD \Tcheck \eTD \bTD \emptY \eTD \eTR
	\bTR \bTD \Tfun \eTD \bTD Contrôlez le fonctionnement de l’éclairage et de la signalisation du véhicule \eTD \bTD \Tcheck \eTD \bTD \inP[sec:lighting] \eTD \eTR
\eTABLEbody
\eTABLE

\subsection[table:scheduleweekly]{Contrôle hebdomadaire}

\bTABLE
\bTABLEhead
\bTR \bTD Type \eTD \bTD Contrôle hebdomadaire \eTD \bTD \Tcheck \eTD \bTD Ref. \eTD \eTR
\eTABLEhead
\bTABLEbody
	\bTR \bTD \Tfun \eTD \bTD Contrôlez la pression des pneus (selon pressions indiquées sur la plaquette dans la cabine) \eTD \bTD \Tcheck \eTD \bTD \inP[sec:pneumatiques] \eTD \eTR
	\bTR \bTD \Tgen \eTD \bTD Contrôlez|/|nettoyez la cartouche du filtre à air \eTD \bTD \Tcheck \eTD \bTD \inP[sSec:vw:airFilter] \eTD \eTR
	\bTR \bTD \Tgen \eTD \bTD Graissez tous les points de graissage (châssis, direction articulée) \eTD \bTD \Tcheck \eTD \bTD \inP[sec:grasing:plan] \eTD \eTR
\eTABLEbody
\eTABLE
%%%%%%%%%%%%%%%%%%%%%%%%%%%%%%%%%%%%%%%%%%%%%%%%%%%%%%%%%%%%%%%%%%%%%%%%%%%%%%

\subsection [sec:50h]{Entretien après 50\,h~— service unique}

\bTABLE
\bTABLEhead
\bTR \bTD Type \eTD \bTD Entretien après 50\,h~— service unique \eTD \bTD Q. \eTD \bTD Ref. \eTD \eTR
\eTABLEhead
\bTABLEbody
	\bTR \bTD \Tlev \eTD \bTD Contrôlez le niveau du liquide de refroidissement du moteur \eTD \bTD \Tcheck \eTD \bTD \inP[sSec:vw:cooling] \eTD \eTR
	\bTR \bTD \Tgen \eTD \bTD Contrôlez|/|nettoyez la cartouche du filtre à air \eTD \bTD \Tcheck \eTD \bTD \inP[sSec:vw:airFilter] \eTD \eTR
	\bTR \bTD \Tgen \eTD \bTD Contrôlez la tension de la courroie multi-V, remplacez||la si nécessaire \eTD \bTD \Tpro \eTD \bTD \inP[sSec:vw:belt] \eTD \eTR
	\bTR \bTD \Tgen \eTD \bTD Remplacez le filtre retour|/|aspiration hydraulique \eTD \bTD \Tcheck \eTD \bTD \inP[sec:hydraulic] \eTD \eTR
	\bTR \bTD \Tlev \eTD \bTD Contrôlez le niveau hydraulique (regard sur le réservoir) \eTD \bTD \Tcheck \eTD \bTD \inP[sec:hydraulic] \eTD \eTR
% 	\bTR \bTD \Tlev \eTD \bTD Graissage central (Option): Contrôlez la réserve et la consistance du lubrifiant \eTD \bTD \Tpro \eTD \bTD \inP[main:graissageCentral] \eTD \eTR
	\bTR \bTD \Tgen \eTD \bTD Graissez tous les points de graissage (châssis, direction articulée) \eTD \bTD \Tcheck \eTD \bTD \inP[sec:grasing:plan] \eTD \eTR
	\bTR \bTD \Tlev \eTD \bTD Contrôlez le niveau du liquide lave||glace \eTD \bTD \Tcheck \eTD \bTD \inP[sec:liqquantities] \eTD \eTR
	\bTR \bTD \Tfun \eTD \bTD Contrôlez les éléments de fixation de la cabine \eTD \bTD \Tcheck \eTD \bTD \emptY \eTD \eTR
	\bTR \bTD \Tfun \eTD \bTD Contrôlez les éléments de fixation du moteur au châssis \eTD \bTD \Tcheck \eTD \bTD \emptY \eTD \eTR
	\bTR \bTD \Tfun \eTD \bTD Contrôlez la fixation des pompes au moteur diesel \eTD \bTD \Tcheck \eTD \bTD \emptY \eTD \eTR
	\bTR \bTD \Tfun \eTD \bTD Contrôlez la fixation du radiateur combiné de refroidissement \eTD \bTD \Tcheck \eTD \bTD \emptY \eTD \eTR
	\bTR \bTD \Tfun \eTD \bTD Contrôlez la fixation des essieux \eTD \bTD \Tcheck \eTD \bTD \emptY \eTD \eTR
	\bTR \bTD \Tfun \eTD \bTD Contrôlez\index{couple de serrage+roues} le serrage des roues (180\,Nm) \eTD \bTD \Tcheck \eTD \bTD \emptY \eTD \eTR
	\bTR \bTD \Tfun \eTD \bTD Contrôlez la pression des pneus (selon pressions indiquées sur la plaquette dans la cabine) \eTD \bTD \Tcheck \eTD \bTD \inP[sec:pneumatiques] \eTD \eTR
	\bTR \bTD \Tgen \eTD \bTD Contrôlez|/|réglez la course du levier de frein à main \eTD \bTD \Tpro \eTD \bTD \emptY \eTD \eTR
	\bTR \bTD \Tlev \eTD \bTD Contrôlez l’état de la batterie, nettoyer les bornes de contact \eTD \bTD \Tcheck \eTD \bTD \inP[sec:battcheck] \eTD \eTR
	\bTR \bTD \Tgen \eTD \bTD Contrôlez|/|ajustez la diffusion du faisceau lumineux selon la législation en vigueur \eTD \bTD \Tcheck \eTD \bTD \inP[sec:lighting] \eTD \eTR
	\bTR \bTD \Tgen \eTD \bTD Graissez le noyau des électrovannes avec de la graisse cuivrée \eTD \bTD \Tpro \eTD \bTD \inP[sec:solenoid] \eTD \eTR
	\bTR \bTD \Tgen \eTD \bTD Effectuez une lecture des codes de défauts (Vpad et moteur diesel), effectuez les réparations nécessaires \eTD \bTD \Tcheck \eTD \bTD \inP[sSec:vw:faultMemory], \inP[vpad:error] \eTD \eTR
\eTABLEbody
\eTABLE

%%%%%%%%%%%%%%%%%%%%%%%%%%%%%%%%%%%%%%%%%%%%%%%%%%%%%%%%%%%%%%%%%%%%%%%%%%%%%%



\subsection {Entretien chaque 600\,h|/|12 mois}

\bTABLE
\bTABLEhead
\bTR \bTD Type \eTD \bTD Entretien chaque 600\,h|/|12 mois \eTD \bTD Q. \eTD \bTD Ref. \eTD \eTR
\eTABLEhead
\bTABLEbody
	\bTR \bTD \Tchg \eTD \bTD Vidangez le moteur diesel \eTD \bTD \Tpro \eTD \bTD \inP[ssSec:vw:oilDraining] \eTD \eTR
	\bTR \bTD \Tgen \eTD \bTD Remplacez le filtre à huile du moteur diesel \eTD \bTD \Tpro \eTD \bTD \inP[ssSec:vw:oilFilter] \eTD \eTR
	\bTR \bTD \Tgen \eTD \bTD Remplacez le filtre à carburant \eTD \bTD \Tpro \eTD \bTD \inP[ssSec:vw:fuelFilter] \eTD \eTR
	\bTR \bTD \Tlev \eTD \bTD Contrôlez l’étanchéité du moteur et des composants dans le compartiment du moteur \eTD \bTD \Tpro \eTD \bTD \emptY \eTD \eTR
	\bTR \bTD \Tlev \eTD \bTD Contrôlez l’étanchéité et la fixation de la tubulure d’échappement \eTD \bTD \Tpro \eTD \bTD \emptY \eTD \eTR
	\bTR \bTD \Tlev \eTD \bTD Contrôlez l’état et la tension de la courroie multi-V, remplacez-la si nécessaire \eTD \bTD \Tpro \eTD \bTD \inP[sSec:vw:belt] \eTD \eTR
	\bTR \bTD \Tlev \eTD \bTD Contrôlez le niveau du liquide de refroidissement du moteur \eTD \bTD \Tcheck \eTD \bTD \inP[sSec:vw:cooling] \eTD \eTR
	\bTR \bTD \Tgen \eTD \bTD Remplacez la cartouche du filtre à air \eTD \bTD \Tcheck \eTD \bTD \emptY \eTD \eTR
	\bTR \bTD \Tgen \eTD \bTD Remplacez le filtre retour|/|aspiration hydraulique \eTD \bTD \Tcheck \eTD \bTD \inP[sec:hydraulic] \eTD \eTR
	\bTR \bTD \Tchg \eTD \bTD Contrôlez le niveau du réservoir hydraulique \eTD \bTD \Tpro \eTD \bTD \inP[sec:hydraulic] \eTD \eTR
% 	\bTR \bTD \Tlev \eTD \bTD Graissage central (Option): Contrôlez la réserve et la consistance du lubrifiant \eTD \bTD \Tpro \eTD \bTD \inP[main:graissageCentral] \eTD \eTR
	\bTR \bTD \Tgen \eTD \bTD Graissez tous les points de graissage (châssis, direction articulée) \eTD \bTD \Tcheck \eTD \bTD \inP[sec:grasing:plan] \eTD \eTR
	\bTR \bTD \Tlev \eTD \bTD Contrôlez le niveau du liquide lave||glace \eTD \bTD \Tcheck \eTD \bTD \inP[sec:liqquantities] \eTD \eTR
	\bTR \bTD \Tfun \eTD \bTD Contrôlez les éléments de fixation de la cabine \eTD \bTD \Tcheck \eTD \bTD \emptY \eTD \eTR
	\bTR \bTD \Tfun \eTD \bTD Contrôlez les éléments de fixation du moteur au châssis \eTD \bTD \Tcheck \eTD \bTD \emptY \eTD \eTR
	\bTR \bTD \Tfun \eTD \bTD Contrôlez la fixation des pompes au moteur diesel \eTD \bTD \Tcheck \eTD \bTD \emptY \eTD \eTR
	\bTR \bTD \Tfun \eTD \bTD Contrôlez la fixation du radiateur combiné de refroidissement \eTD \bTD \Tcheck \eTD \bTD \emptY \eTD \eTR
	\bTR \bTD \Tfun \eTD \bTD Contrôlez la fixation des essieux \eTD \bTD \Tcheck \eTD \bTD \emptY \eTD \eTR
	\bTR \bTD \Tgen \eTD \bTD Contrôlez|/|réglez la course du levier de frein à main \eTD \bTD \Tpro \eTD \bTD \emptY \eTD \eTR
	\bTR \bTD \Tgen \eTD \bTD Nettoyez|/|contrôlez l’état des tambours et des garnitures de freins; nettoyez le mécanisme \eTD \bTD \Tpro \eTD \bTD \inP[sec:brake] \eTD \eTR
	\bTR \bTD \Tfun \eTD \bTD Contrôlez\index{couple de serrage+roues} le serrage des roues (\torqueSymb\,180\,Nm) \eTD \bTD \Tcheck \eTD \bTD \emptY \eTD \eTR
	\bTR \bTD \Tfun \eTD \bTD Contrôlez la pression des pneus (selon pressions indiquées sur la plaquette dans la cabine) \eTD \bTD \Tcheck \eTD \bTD \inP[sec:pneumatiques] \eTD \eTR
	\bTR \bTD \Tlev \eTD \bTD Contrôlez l’état de la batterie, nettoyer les bornes de contact \eTD \bTD \Tcheck \eTD \bTD \inP[sec:battcheck] \eTD \eTR
	\bTR \bTD \Tgen \eTD \bTD Contrôlez|/|ajustez la diffusion du faisceau lumineux selon la législation en vigueur \eTD \bTD \Tcheck \eTD \bTD \inP[sec:lighting] \eTD \eTR
	\bTR \bTD \Tgen \eTD \bTD Graissez le noyau des électrovannes avec de la graisse cuivrée \eTD \bTD \Tpro \eTD \bTD \inP[sec:solenoid] \eTD \eTR
%	\bTR \bTD \Tgen \eTD \bTD Prévention contre la corrosion: contrôlez|/|renouvelez la couche de protection \eTD \bTD \Tspecial \eTD \bTD \inP[sec:anticorrosion] \eTD \eTR
	\bTR \bTD \Tgen \eTD \bTD Effectuez une lecture des codes de défauts (Vpad et moteur diesel), effectuez les réparations nécessaires \eTD \bTD \Tcheck \eTD \bTD \inP[sSec:vw:faultMemory], \inP[vpad:error] \eTD \eTR
\eTABLEbody
\eTABLE

%%%%%%%%%%%%%%%%%%%%%%%%%%%%%%%%%%%%%%%%%%%%%%%%%%%%%%%%%%%%%%%%%%%%%%%%%%%%%%
\subsection {Entretien chaque 1'200\,h|/|24 mois}
\bTABLE
\bTABLEhead
\bTR \bTD Type \eTD \bTD Entretien chaque 1'200\,h|/|24 mois \eTD \bTD Q. \eTD \bTD Ref. \eTD \eTR
\eTABLEhead
\bTABLEbody
	\bTR \bTD \Tchg \eTD \bTD Vidangez le réservoir hydraulique \eTD \bTD \Tpro \eTD \bTD \inP[sec:hydraulic] \eTD \eTR
	\bTR \bTD \Tgen \eTD \bTD Régénérez le fluide frigorigène R134a de l’installation de climatisation \eTD \bTD \Tspecial \eTD \bTD \emptY \eTD \eTR
\eTABLEbody
\eTABLE

%%%%%%%%%%%%%%%%%%%%%%%%%%%%%%%%%%%%%%%%%%%%%%%%%%%%%%%%%%%%%%%%%%%%%%%%%%%%%%
\subsection {Entretien chaque 2'400\,h|/|4 ans}
\bTABLE

\bTABLEhead
\bTR \bTD Type \eTD \bTD Entretien chaque 2'400\,h|/|4 ans \eTD \bTD Q. \eTD \bTD Ref. \eTD \eTR
\eTABLEhead
\bTABLEbody
	\bTR \bTD \Tgen \eTD \bTD Remplacez la courroie de distribution du moteur \eTD \bTD \Tspecial \eTD \bTD \emptY \eTD \eTR
\eTABLEbody
\eTABLE

%%%%%%%%%%%%%%%%%%%%%%%%%%%%%%%%%%%%%%%%%%%%%%%%%%%%%%%%%%%%%%%%%%%%%%%%%%%%%%
\subsection {Entretien chaque 4'800\,h|/|8 ans}
\bTABLE

\bTABLEhead
\bTR \bTD Type \eTD \bTD Entretien chaque 4'800\,h|/|8 ans \eTD \bTD Q. \eTD \bTD Ref. \eTD \eTR
\eTABLEhead
\bTABLEbody
	\bTR \bTD \Tgen \eTD \bTD Contrôlez|/|remplacez les conduites souples du circuit hydraulique si nécessaire
	\eTD \bTD \Tpro \eTD \bTD \inP[sec:hydraulic] \eTD \eTR
	\bTR \bTD \Tlev \eTD \bTD Remplacez la pompe à eau (simultanément avec la courroie crantée)
	\eTD \bTD \Tspecial \eTD \bTD \emptY \eTD \eTR
	\eTABLEbody
\eTABLE

\stopregister[index][wartplan]

\page [yes]


\section[sec:schedaggr]{Maintenance des agrégats}

La maintenance\startregister[index][wartplanAgg]{plan d’entretien+agrégats} des agrégats comprend
les travaux d’entretien réguliers\index{contrôle périodique} et les contrôles quotidien et hebdomadaire:

\starttextbackground[FC]
 \starttabulate [|w(39mm)B|w(25mm)B|p({\dimexpr\textwidth-(64mm+2.5em)\relax})|]
 \NC \Pmtpro\enskip\bf Travaux d’entretien \NC\bf 50\,h \NC Premier service après 50\,h\NC \NR
 \NC \NC\bf 600\,h \NC Entretien régulier chaque 600\,h|/|12 mois \NC \NR
 \NC \Pmtcheck\enskip\bf Liste de contrôle \NC Quotidien \NC Durant toute la saison de travail \NC\NR
 \NC \Pmtcheck\enskip\bf Liste de contrôle \NC Hebdomadaire \NC Durant toute la saison de travail \NC\NR
 \stoptabulate
\stoptextbackground
\blank [big]


Le plan d’entretien ci||après se réfère aux agrégats standards qui équipent généralement le \pquatre. Considérez également le plan d’entretien de chaque agrégat conçu pour une opération spécifique, non décrit dans le présent manuel.



\subsection[table:scheduledaily]{Contrôle quotidien}

\bTABLE
\bTABLEhead
\bTR \bTD Type \eTD \bTD Contrôle quotidien \eTD \bTD Q. \eTD \bTD Ref. \eTD \eTR
\eTABLEhead
\bTABLEbody
	\bTR \bTD \Tgen \eTD \bTD Nettoyez la cuve et le dispositif de recyclage \eTD \bTD \Tcheck \eTD \bTD \inP[sec:cleaning] \eTD \eTR
	\bTR \bTD \Tgen \eTD \bTD Nettoyez la bouche et le canal d’aspiration \eTD \bTD \Tcheck \eTD \bTD \inP[sec:cleaning] \eTD \eTR
\eTABLEbody
\eTABLE


\subsection[table:scheduledaily]{Contrôle hebdomadaire}

\bTABLE
\bTABLEhead
\bTR \bTD Type \eTD \bTD Contrôle hebdomadaire \eTD \bTD Q. \eTD \bTD Ref. \eTD \eTR
\eTABLEhead
\bTABLEbody
	\bTR \bTD \Tgen \eTD \bTD Contrôlez l’usure des balais latéraux et du balai frontal (option) \eTD \bTD \Tcheck \eTD \bTD \emptY \eTD \eTR
	\bTR \bTD \Tgen \eTD \bTD Contrôlez les gommes et le clapet de la bouche d’aspiration \eTD \bTD \Tpro \eTD \bTD \emptY \eTD \eTR
	\bTR \bTD \Tgen \eTD \bTD Graissez tous les points de graissage (cuve, balais, bouche d’aspiration) \eTD \bTD \Tcheck \eTD \bTD \inP[sec:grasing:plan] \eTD \eTR
\eTABLEbody
\eTABLE


\subsection {Entretien après 50\,h~– service unique}

\bTABLE
\bTABLEhead
\bTR \bTD Type \eTD \bTD Entretien après 50\,h~– service unique \eTD \bTD Q. \eTD \bTD Ref. \eTD \eTR
\eTABLEhead
\bTABLEbody
	\bTR \bTD \Tfun \eTD \bTD Contrôlez les dispositifs de fixation des balais \eTD \bTD \Tcheck \eTD \bTD \emptY \eTD \eTR
	\bTR \bTD \Tgen \eTD \bTD Réglez les gommes et le clapet de la bouche d’aspiration \eTD \bTD \Tpro \eTD \bTD \emptY \eTD \eTR
\eTABLEbody
\eTABLE


\subsection {Entretien chaque 600\,h|/|12 mois}


\bTABLE
\bTABLEhead
\bTR \bTD Type \eTD \bTD Entretien chaque 600\,h|/|12 mois \eTD \bTD Q. \eTD \bTD Ref. \eTD \eTR
\eTABLEhead
\bTABLEbody
	\bTR \bTD \Tfun \eTD \bTD Contrôlez les dispositifs de fixation des balais \eTD \bTD \Tcheck \eTD \bTD \emptY \eTD \eTR
	\bTR \bTD \Tgen \eTD \bTD Réglez les gommes et le clapet de la bouche d’aspiration \eTD \bTD \Tpro \eTD \bTD \inP[sec:main:suctionMouth] \eTD \eTR
	\bTR \bTD \Tchg \eTD \bTD Vidangez la pompe à eau haute pression (option) \eTD \bTD \Tpro \eTD \bTD \emptY \eTD \eTR
\eTABLEbody
\eTABLE

\stopregister[index][wartplanAgg]
\stop


\stopcomponent
% vim: fdm=indent
