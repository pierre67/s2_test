\startcomponent c_30_overview_s2_120-fr


\chapter{Présentation du véhicule}

\setups [pagestyle:marginless]


\placefig [here] [] {Aperçu véhicule côté gauche}
{\externalfigure [overview:side:left]}


\page [yes]


\placefig [here] [] {Aperçu véhicule côté droit}
{\externalfigure [overview:side:right]}

\page [yes]

\setups [pagestyle:normal]


\section{Description du véhicule}

\placefig[margin][p4_vue_01]{\sdeux en dislocation}
{%
\startcombination [1*3]
{\externalfigure[overview:vhc:01]}{}
{\externalfigure[overview:vhc:02]}{}
{\externalfigure[overview:vhc:03]}{}
\stopcombination}


Avec le \BosFull{sdeux}, Marcel Boschung SA a voulu faire profiter à ses clients d’un savoir||faire issu d’une longue expérience obtenue grâce à une étroite collaboration avec ses fidèles clients et partenaires commerciaux. Les besoins des communes en matière de mobilité et de polyvalence ont évolué au fil des décennies, obligeant les concepteurs de la \sdeux à être constamment à l’écoute des clients et des utilisateurs d’une part, et du service après||vente d’autre part. De cette politique d’écoute et de partenariat et de sa synthèse est née la \sdeux.


\subsection{Une technologie innovante}

La balayeuse compacte \BosFull{sdeux} se distingue dans sa catégorie par son faible poids (2300\,kg),
sa grande capacité (cuve à déchets classe 2,5\,m³), ses dimensions compactes (largeur 1,15\,m)
et son poste de conduite à l’ergonomie particulièrement bien adaptée.

Sa conception étroite et compacte confère au \sdeux un statut de passe-partout idéal
pour le balayage des rues et des trottoirs dans les villes et villages.
Son puissant moteur diesel, accouplé à une transmission hydrostatique compacte
(moteurs hydrauliques à pistons radiaux dans les roues avant),
lui assure une mobilité confortable, quelque soient la topographie du lieu et le volume dans la cuve.

Les pompes hydrauliques sont entrainées par un moteur diesel de type {\tt VW 2.0 CDI}
répondant à la norme EURO~V. Il fournit un couple de 285\,Nm @ 1750\,min\high{-1}
et une puissance maximale de 75\,kW @ 3000\,min\high{-1}.
La machine peut ainsi être utilisée avec un faible régime moteur, donc un niveau de bruit réduit.
La \sdeux est équipée de série d’un filtre antiparticules.


\section{Innovations au service de l’opérateur}

Son châssis articulé offre un rayon de braquage réduit pour une mobilité maximale.
Les matériaux spécifiques (Domex®) utilisés pour sa conception et
son développement entièrement réalisé sur système CAD confèrent à la \sdeux une charge utile de 1200\,kg.

\placefig[margin][overview:cab:frontright]{\sdeux prêtes pour leur mission}
{\externalfigure[overview:cab:twoleft][width=\Bildwidth]}

La cabine est totalement vitrée et dispose de deux sièges confortables avec ceintures de sécurité à trois points.
La \sdeux peut être équipée, en option, d’un système de climatisation.

Avec une vitesse de déplacement atteignant 40\,km/h, on se déplace à l’aise dans la circulation urbaine.
La suspension des essieux avant et arrière garantit un confort de conduite ainsi qu’une bonne stabilité,
même sur les chaussées les plus accidentées.

Le dispositif de balayage~– monté sur deux bras articulés et placé dans le champ de vision de
l’opérateur~– et la bouche d’aspiration~– placée en porte à faux de l’essieu avant~–
précèdent idéalement la trajectoire de la machine.
Un balai frontal à double articulation peut être ajouté en option.

\page [yes]


\subsection{Cabine silencieuse et confortable}

La cabine\index{habitacle} de la \sdeux\ est prévue
pour deux personnes avec conduite à droite. Elle est insonorisée et montée sur silentblocs antivibrations.

Les portes et le plancher sont vitrés pour élargir le champ de vision.
Le parebrise comprend toute la face avant de la cabine, offrant ainsi une vision parfaite sur le travail des balais.

Le siège du conducteur est équipé d’un ressort avec réglage mécanique ou pneumatique (option).
Le siège du conducteur et celui du passager sont montés sur glissières réglables.


\subsubsubject{Ergonomie du poste de travail}

\startfigtext[right][overview:joy:sideview]{Console de commande}
{\externalfigure[overview:joy:top]}
La console multifonction placée à gauche du siège conducteur offre toutes les fonctions nécessaires dans une seule main.
Les deux balais sont pilotés indépendamment par les deux joysticks au moyen du pouce et de l’index.
Les commandes des balais et du balai frontal (option),
le régime de rotation du moteur, le tempomat, etc., sont également disponibles sur la console multifonction.
\stopfigtext

Un écran de contrôle tactile, placé juste au||dessus du champ de vision, affiche les informations relatives
aux fonctions de la machine en temps réel sans perturber le regard de l’opérateur.

\placefig[margin][overview:vhc:left]{\sdeux sur site historique}
{\externalfigure[overview:vhc:left]}

\page [yes]


\subsubsubject{Le poste de conduite du véhicule}

Le sélecteur\index{poste de conduite} de marche (2 marches avant et une marche arrière) est disposé sur la colonne de direction,
à portée de la main droite. Un bouton poussoir placé à l’extrémité du sélecteur permet d’alterner
les modes \quote{travail} et \quote{transport} sans arrêter la machine (lisez le chapitre \about[sec:using:work] \atpage[sec:using:work]
pour plus d’informations sur l’utilisation de votre \sdeux).

\placefig[margin][fig:overview:steeringwheel]{Poste de conduite}
{\externalfigure[overview:driver:place]}

Lors de manœuvres en marche arrière, un écran de contrôle affiche la trajectoire grâce à une petite caméra montée à l’arrière
de la machine. Un avertisseur sonore de marche arrière~– qui peut être désactivé depuis l’écran du Vpad~– est activé simultanément.

Le levier multifonction placé à gauche de la colonne de direction comprend les commandes d’essuie-glace (2 vitesses + intermittent)
ainsi que les appels de feux et l’avertisseur acoustique.

Consulter le chapitre \about[chap:using] \at{page}[chap:using] pour plus de détails sur
les fonctions de votre balayeuse \sdeux.

\page [yes]

\setups[pagestyle:marginless]


\subsection[overview:brushsystem]{Dispositif de balayage et aspiration}

\subsubsubject{Les balais}

\startfigtext[left][fig:overview:steeringwheel]{Dispositif de balayage|/|aspiration}
{\externalfigure[system:brush]}
Les balais\index{balayage} sont montés sur des têtes orientables à l’extrémité de bras articulés. La poussière
générée par le balayage est éliminée par humectage. Chaque balai est muni de deux buses
de pulvérisation d’eau qui provient soit du réservoir d’eau claire, soit du dispositif de recyclage.

Un interrupteur\index{aspiration} sur la console multifonction active la rotation des balais et la pompe à eau simultanément\footnote{%
Lisez le chapitre \in[chap:using] \about[chap:using]~– plus particulièrement la section \about[sec:using:work] \atpage[sec:using:work]~–
pour plus de détails sur la fonction pompe à eau.}.
Le positionnement des balais et leur inclinaison sur les deux axes (transversal et longitudinal) sont
pilotés directement depuis leur joystick respectif sur la console de commande.
\stopfigtext

Les balais sont équipés d’un système anticollision mécanique et hydraulique.


\subsubsubject{La bouche d’aspiration}

La bouche d’aspiration occupe tout l’espace entre les balais lorsque ceux||ci sont complètement écartés.

En position de travail (abaissée), la bouche d’aspiration repose sur 4 roulettes.
Sa position \quote{tractée} protège le mécanisme en cas de collision avec un obstacle.

En marche arrière, la bouche d’aspiration se lève automatiquement.

L’étanchéité est assurée par une épaisse bande d’usure en caoutchouc remplaçable.

Un clapet à l’avant de la bouche d’aspiration, actionné par commande électro||hydraulique, permet
le passage des gros déchets.



\subsubsubject{La cuve à déchets}

La cuve~– en aluminium~– permet un angle de basculement de 50° et une hauteur de vidange 1.5\,m.
Elle est alimentée par le canal d’aspiration placé au||dessous avec diamètre intérieur 180\,mm.

Le flux d’aspiration est assuré par une turbine haute performance placée horizontalement dans la cuve,
avec une porte de visite pour le nettoyage et les inspections.

Deux grilles d’aspiration en acier inox~– rabattable et nettoyable sans outillage~– sont placées
dans le couvercle de la cuve. Ce dernier est déverrouillable et relevable simplement à la main.

Un clapet à commande manuelle permet d’alterner facilement entre le canal d’aspiration
et le tuyau d’aspiration à main (disponible en option).



\subsection{Dispositif d’humectage}


\subsubsubject{Distribution d’eau claire}

Le réservoir (PE moulé)\index{balayage+humectage} est disposé verticalement au dos de la cabine.
Sa contenance\index{eau claire+réservoir} est de 190\,l.

Une pompe électrique (6,5\,l/min) alimente le système jusqu’aux buses de pulvérisation disposées au||dessus
de chaque balai, y compris le 3\high{e} balai disponible en option.


\subsubsubject{Recyclage de l’eau souillée}

L’eau souillée traverse les microperforations des parois intérieures de la cuve à déchets
pour descendre dans le réservoir d’eau de recyclage disposé au||dessous, par le clapet de recyclage.
Sa contenance\index{eau de recyclage+réservoir}  est de 140\,l.

Une pompe hydraulique immergée alimente le système jusqu’aux buses de pulvérisation disposées
à l’intérieur de la bouche et du canal d’aspiration.


\subsubsubject{Réservoir d’eau de recyclage}

Le réservoir d’eau de recyclage comprend un échangeur thermique eau|/|fluide hydraulique à double fonction:

\startitemize[width=30mm,style=\md]
\sym{Fonction en été} L’eau en mouvement conduit la chaleur du fluide hydraulique par convection
vers les parois en aluminium du réservoir, qui la diffuse à l’air libre par rayonnement.

\sym{Fonction en hiver} La chaleur créée par le fluide hydraulique réchauffe l’eau contenue dans le réservoir.
Cette solution permet d’humecter le canal et la bouche d’aspiration même lorsque la température est légèrement
inférieure à 0\,°C.
\stopitemize



\subsubsubject{Surveillance du niveau d’eau dans les réservoirs}

\startitemize[width=30mm,style=\md]
\sym{Eau claire} Lorsque le niveau est insuffisant, le symbole \textSymb{vpad_water}
apparait à l’écran du Vpad.
\sym{Eau de recyclage} Lorsque le niveau est au||dessous de l’échangeur (voir ci||dessus), le symbole \textSymb{vpad_rwater_orange} (jaune) apparait à l’écran du Vpad. Lorsque le niveau est insuffisant, le symbole \textSymb{vpad_rwater} (rouge) est affiché.
\stopitemize

\page [yes]

\setups[pagestyle:normal]

\section{Identification du véhicule}

\subsection{Plaque d’identification du véhicule}

La plaque signalétique générale se trouve dans la cabine\index{identification+véhicule}
face à la console, sous le siège du passager (voir \in{fig.}[fig:identity:location], \atpage[fig:identity:location]).

\subsection{Code|/|numéro du moteur}

Le code moteur est indiqué sur l’étiquette (autocollant) d’identification du moteur, qui se trouve sur le conduit métallique coudé
du circuit de refroidissement à l’avant du moteur (levez la cuve).\par

Le numéro du moteur est gravé sur le moteur (voir \in{fig.}[identity:engine:number]). Il comprend neuf caractères alphanumériques.
Les trois premiers caractères forment le code moteur, et les six autres forment
le numéro de série du moteur.


\placefig[margin][idvhc]{Identification du véhicule}
{\externalfigure[s2:id:plaque]}

\placefig[margin][identity:engine:code]{Identification du moteur}
{\externalfigure[engine:id:code]}

\placefig[margin][identity:engine:number]{Numéro du moteur}
{\externalfigure[engine:id:number]}

\page [yes]


\subsection{Plaque signalétique des pneus}

La plaque\index{identification+pression des pneus} signalétique
des pneus se trouve dans la cabine, au||dessous du siège du passager.


\subsection{Numéro du châssis}

Le numéro du châssis\index{identification+numéro de châssis} est frappé
dans le châssis du côté droit du véhicule sous la cabine.

\subsection{Conformité et Label \symbol [europe] [CEsign]}

Le label \symbol [europe] [CEsign] se trouve dans la cabine, face à la console,
sous le siège du passager.

La \sdeux\ répond\index{pneus+gonflage}\index{roues+dimensions} aux exigences de base de sécurité et de santé selon la directive\index{certificat+conformité européenne}\index{norme machine} machines 2006/42/\symbol [europe] [CEsign]\footnote{DIRECTIVE 2006/42/CE DU PARLEMENT EUROPÉEN ET DU CONSEIL du 17 mai 2006
concernant le rapprochement des législations des États membres relatives aux machines}.
\textrule

\placefig[margin][idpneus]{Pression des pneus}
{\externalfigure[identity:tires]}

\placefig[margin][fig:identity:location]{Plaquettes du constructeur}
{\externalfigure[identity:location]}

\page [yes]


\setups [pagestyle:marginless]



\startsection[title={Données techniques},
							reference={donnees_techniques}]

\subsection [sec:measurement] {Dimensions du véhicule}


\placefig[here][fig:measurement]{\select{caption}{Largeur~– balais au repos ou écartés~–, longueur et hauteur du véhicule}{Dimensions du véhicule}}
{\Framed{\externalfigure[s2:measurement]}}

\page [yes]

\placefig[here][fig:measurement]{\select{caption}{Hauteur du véhicule avec cuve déployée}{Hauteur du véhicule }}
{\Framed{\externalfigure[s2:measurement:02]}}

\page [yes]




\starttabulate[|lBw(45mm)|p|l|rw(35mm)|]
\FL
\NC Dimension\index{dimensions} \NC \bf Mesure \NC \bf Unité \NC \bf Valeurs \NC\NR
\ML
\NC Dimensions de la machine \NC Longueur (hors tout) \NC \unite{mm} \NC 4588,00 \NC\NR
\NC\NC Longueur avec 3\high{e} balai  \NC \unite{mm} \NC 5020,00 \NC\NR
\NC\NC Largeur du véhicule  \NC \unite{mm} \NC 1150,00 \NC\NR
\NC\NC Largeur du véhicule (hors tout) \NC \unite{mm} \NC 1575,00 \NC\NR
\NC\NC Hauteur sans gyrophare \NC \unite{mm} \NC 1990,00 \NC\NR
\NC\NC Empattement             \NC \unite{mm} \NC 1740,00 \NC\NR
\NC\NC Voie                    \NC \unite{mm} \NC 894,00 \NC\NR
\ML
\NC Largeur de balayage \NC Balais standard \NC \unite{mm} \NC 2300,00 \NC\NR
\NC\NC Avec 3\high{e} balais \NC \unite{mm} \NC 2600,00 \NC\NR
\NC\NC Diamètre des balais \NC \unite{mm} \NC 800,00 \NC\NR
\NC\NC Bouche d’aspiration \NC \unite{mm} \NC 800,00 \NC\NR
\ML
\NC Répartition de la charge \NC Poids à vide\note[weight:empty] (essieu avant) \NC \unite{kg} \NC \~\,1100,00 \NC\NR
\NC\NC Poids à vide\note[weight:empty] (essieu arrière) \NC \unite{kg} \NC  \~\,1200,00 \NC\NR
\NC\NC Poids total à vide\note[weight:empty] \NC \unite{kg} \NC  \~\,2300,00 \NC\NR
\NC\NC Poids total \NC \unite{kg} \NC 3500,00 \NC\NR
\LL
\stoptabulate


\subsection{Rayons de braquage et de balayage}


\starttabulate[|lBw(45mm)|p|l|rw(35mm)|]
\FL
\NC Dimension\index{dimensions} \NC \bf Mesure \NC \bf Unité \NC \bf Valeurs \NC\NR
\ML
\NC Rayon de braquage\index{rayon de braquage}\index{dimensions+rayon de braquage} \NC encombrement minimum avec balais  \NC \unite{mm}	\NC 3325,00 \NC\NR
\ML
\NC Rayon de balayage \NC extérieur \NC \unite{mm} \NC 3425,00 à 3850,00 \NC\NR
\NC\NC intérieur \NC \unite{mm} \NC 2025,00 à 1675,00 \NC\NR
\LL
\stoptabulate

\footnotetext[weight:empty]{Configuration standard, avec chauffeur (env. 75\,kg).}

\placefig[here][pict:steerin_sweeping:radius]{Rayons de braquage et d’encombrement et rayon de balayage}
{\externalfigure[steerin_sweeping:radius]}

\page [yes]


\subsection{Roues et pneumatiques}

\starttabulate[|lBw(45mm)|p|rw(55mm)|]
\FL
\NC Objet \NC \bf Équipement \NC \bf Valeurs \NC\NR
\ML
\NC Pneumatiques \NC Dimensions standards \NC 205/70 R 15 C \NC\NR
\ML
\NC Jantes \NC Dimensions standards \NC 6J\;×\;15 H2 ET 60 \NC\NR
\ML
\NC Pression de gonflage \NC Équipement standard, av/ar \NC 4,5|/|5,8\,bar \NC\NR
\LL
\stoptabulate


\subsection{Moteur diesel}

\starttabulate [|lBw(45mm)|l|rp|]
\FL
\NC \bf Système\index{moteur diesel+identification} \NC \bf Objet \NC \bf Caractéristiques\NC\NR
\ML
\NC Type du moteur      \NC  \NC VW CJDA TDI 2.0 – 475 NE  \NC\NR
\NC Généralité  \NC 	Cycle de fonctionnement        \NC Diesel à quatre temps  \NC\NR
\NC\NC Nombre de cylindres \unite{n} \NC  4    \NC\NR
\NC\NC Alésage x course \unite{mm} \NC  81\;×\;95,5   \NC\NR
\NC\NC Cylindrée totale \NC 1968\,cm\high{3}   \NC\NR
\NC\NC Soupapes par cylindre \NC 4 \NC\NR
\NC\NC Ordre de distribution  \NC 1-3-4-2  \NC\NR
\NC\NC Régime minimum à vide  \unite{min\high{-1}}    \NC 830 +\,50/-\,25 \NC\NR
\NC Puissance et couple \NC 	Régime maximum \unite{min\high{-1}} \NC 3400    \NC\NR
\NC\NC Puissance maxi \unite{kW} à \unite{min\high{-1}}    \NC 75 à 3000   \NC\NR
\NC\NC Couple maxi \unite{Nm} à \unite{min\high{-1}}   \NC 285 à 1750 \NC\NR
\NC Consommation spécifique\index{moteur diesel+consommation} \NC Carburant \unite{g/kWh}  \NC 224 (puissance max.) \NC\NR
\NC\NC Huile \unite{g/kWh}  \NC 0,22 \NC\NR
\NC Système d’alimentation  \NC Système d’injection \NC Injection directe \quote{Common Rail}   \NC\NR
\NC\NC Alimentation en combustible \NC Pompe à engrenages  \NC\NR
\NC\NC Suralimentation \NC Oui \NC\NR
\NC\NC Refroidissement de l’air d’admission \NC Oui \NC\NR
\NC\NC Pression de suralimentation \unite{mbar} \NC 1300\NC\NR
\NC Circuit de graissage\index{moteur diesel+graissage} \NC	Type de graissage   \NC Graissage forcé avec échangeur huile/eau \NC\NR
\NC\NC Alimentation du circuit \NC Pompe à rotors \NC\NR
\NC\NC Consommation d’huile \unite{L/20\,h} \NC <\:0,1 \NC\NR
\NC Circuit de refroidissement\index{moteur diesel+refroidissement} \NC	Capacité totale du circuit \unite{L}   \NC env.~12  \NC\NR
\NC\NC Tarage bouchon vase exp. \unite{bar} \NC 1,4 \NC\NR
\NC\NC Ouverture du thermostat (début) \unite{°C} \NC 87 \NC\NR
\NC\NC Ouverture du thermostat (pleine) \unite{°C} \NC 102 \NC\NR
\NC Émissions polluantes \NC Filtre anti||particule \NC Oui \NC\NR
\NC\NC Recyclage des gaz d’échappement \NC Oui \NC\NR
\NC\NC Norme \NC EWG Euro 5  \NC\NR
\LL
\stoptabulate


\subsection{Performances}


\starttabulate[|lBw(45mm)|p|l|rw(35mm)|]
\FL
\NC Performance\index{performances} \NC \bf Condition \NC \bf Unité \NC \bf Valeurs \NC\NR
\ML
\NC Vitesse \NC Mode \quote{travail} \NC \unite{km/h}	\NC 0 à 18 (continu) \NC\NR
\NC\NC Mode \quote{transport} \NC \unite{km/h}	\NC 0 à 40 \NC\NR
\ML
\NC Vitesse limitée \NC Configuration personnalisée \NC \unite{km/h}	\NC 0 à 25 \NC\NR
\LL
\stoptabulate











\subsection{Installation électrique}


\starttabulate[|lw(55mm)|p|rw(30mm)|]
\FL
\NC \bf Objet \NC \bf Composant \NC \bf Caractéristiques \NC\NR
\ML
\NC Source d'énergie électrique \NC Accumulateur au plomb \NC 12\,V 75\,Ah \NC\NR
\NC Alimentation en courant \NC Alternateur \NC 14,8\,V 140\,A \NC\NR
\NC Démarreur électrique \NC Puissance du démarreur 			\NC 1,8\,kW \NC\NR
\NC Équipement audio \NC Prise radio\index{prise radio} et haut-parleurs\index{haut-parleurs} \NC Prémontés \NC\NR
% \NC Sécurité et surveillance			\NC  	Tachygraphe\index{tachygraphe} 					\NC En option				\NC\NR
% \NC\NC Enregistreur de fin de parcours\index{fin de parcours}	\NC En option				\NC\NR
\NC Éclairage et signalisation avant 	\NC 	Feux de position 			\NC 12\,V 5\,W		 		\NC\NR
\NC\NC Feux de croisement 				\NC H7, 12\,V 55\,W 		\NC\NR
\NC\NC Feux de travail 				\NC G886, 12\,V 55\,W 		\NC\NR
\NC\NC Feux clignotants 				\NC 12\,V 21\,W 			\NC\NR
\NC Éclairage et signalisation arrière 	\NC 	Feux de stop combiné 			\NC 12\,V 5/21\,W 			\NC\NR
\NC\NC Feux clignotants 				\NC 12\,V 21\,W 			\NC\NR
\NC\NC Feux de marche arrière 			\NC 12\,V 21\,W 			\NC\NR
\NC\NC Éclairage de plaque 			\NC 12\,V 5\,W 			\NC\NR
\NC Signalisation supplémentaire		\NC 	Gyrophare 						\NC H1, 12\,V 55\,W 		\NC\NR
\LL
\stoptabulate

\stopsection


\stopcomponent
