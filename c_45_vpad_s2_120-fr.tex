\startcomponent c_45_vpad_s2_120-fr


\startchapter[title={Ordinateur de bord (Vpad)},
							reference={sec:vpad}]

\setups[pagestyle:marginless]


\startsection[title={Description du Vpad},
							reference={vpad:description}]

\startfigtext [left] {Implantation du Vpad SN}
{\externalfigure[vpad:inside:view]}
\textDescrHead{Innovant, intelligent… } Le \Vpad\ a été conçu spécialement pour la commande des agrégats techniques municipaux dont la technologie de plus en plus complexe présente des fonctions multiples.
Avec le \Vpad, l’opérateur dispose d’un système qui ne se contente pas d’informer à tout moment de tous les déroulements,
que ce soit visuellement ou~— par le biais de la sortie vocale~— aussi de manière acoustique,
mais qui fixe également de nouvelles normes dans le domaine du guidage de l’opérateur et de la logique de commande.\par

Les performances du \Vpad, qui peut être engagé de façon multifonctionnelle, vont aussi bien au-delà des tâches de commande proprement dites.
\stopfigtext

\textDescrHead{…universel} Lors du développement du \Vpad, une attention toute particulière a été accordée à la compatibilité et à la flexibilité.
Et ce aussi bien dans le cas du boîtier modulaire, qui peut être adapté individuellement à des situations locales et à des variantes d’équipement,
que dans le cas des interfaces électroniques et des technologies de transmission de données qui vont jusqu’au WLAN.
A l’intérieur, le \Vpad\ présente l’électronique la plus moderne avec technologie 32 bits et système d’exploitation en temps réel.
\vfill


\startfigtext[left]{Console multifonction}
{\externalfigure[console:topview]}
\textDescrHead{…et modulaire} La modularité du \Vpad\ présente un grand avantage:
par exemple la version SN montée sur la \sdeux\ peut être étendue pas à pas avec d’autres composants,
tels qu’un modem ou une console (voir ci||contre).
Toutefois, le \Vpad\ ne présente pas seulement une structure modulaire au niveau matériel,
mais également au niveau logiciel, ce qui rend le système extensible dans sa globalité en fonction des besoins.

La console multifonction de la \sdeux\ est une interface entre l’opérateur et la machine.
Elle permet de piloter l’ensemble du dispositif de balayage et d’aspiration.
\stopfigtext

\page [yes]


\subsection[vpad:home]{Écran d’accueil}

% \placefig[left][fig:vpad:engineData]{Accueil mode transport}
% {\scale[sx=1.5,sy=1.5]
% {\setups[VpadFramedFigureHome]
% \VpadScreenConfig{
% \VpadFoot{\VpadPictures{vpadClear}{vpadBeacon}{vpadEngine}{vpadSignal}}}
% \framed{\null}}
% }


\start

\setupcombinations[width=\textwidth]

\placefig [here][fig:vpad:engineData]{Écran d’accueil}
{\startcombination [2*1]
{\setups[VpadFramedFigureHome]% \VpadFramedFigureK pour bande noire
\VpadScreenConfig{
\VpadFoot{\VpadPictures{vpadClear}{vpadBeacon}{vpadEngine}{vpadSignal}}}%
\scale[sx=1.5,sy=1.5]{\framed{\null}}}{Mode \quote{transport}}
{\setups[VpadFramedFigureWork]% \VpadFramedFigureK pour bande noire
\VpadScreenConfig{
\VpadFoot{\VpadPictures{vpadClear}{vpadBeacon}{vpadEngine}{vpadSignal}}}%
\scale[sx=1.5,sy=1.5]{\framed{\null}}}{Mode \quote{travail}}
\stopcombination}

\stop

\blank [1*big]


L’écran d’accueil du \Vpad comprend tous les éléments de surveillance nécessaire au bon fonctionnement de votre \sdeux.
La partie du haut affiche les témoins d’avertissement.\par

La partie médiane affiche les données en temps réel: vitesse, régime et température du moteur, niveau de carburant, niveau d’eau dans le réservoir de recyclage, etc.\par

Le mode \quote{transport} est symbolisé par un lapin \textSymb{transport_mode}, le mode \quote{travail} par une tortue \textSymb{working_mode}.

Le rang du bas affiche les menus disponibles: pressez le centre de l’écran tactile pour afficher des menus supplémentaires.

\page [yes]


\startcolumns

\startSymVpad
\externalfigure[vpadTEnginOilPressure][height=1.7\lH]
\SymVpad
\textDescrHead{Pression d’huile} (rouge) Pression d’huile trop faible, arrêtez le moteur immédiatement.

+\:message d’erreur \# 604
\stopSymVpad

\startSymVpad
\externalfigure[vpadWarningBattery][height=1.7\lH]
\SymVpad
\textDescrHead{Charge batteries} (rouge) Courant de charge trop faible; contactez votre garage.
\stopSymVpad

\startSymVpad
\externalfigure[vpadWarningEngine1][height=1.7\lH]
\SymVpad
\textDescrHead{Diagnostic moteur} (jaune) Erreur détectée dans le système de gestion du moteur; contactez votre garage.
\stopSymVpad

\startSymVpad
\externalfigure[vpadWarningService][height=1.7\lH]
\SymVpad
\textDescrHead{Aller au garage} (jaune) L’échéance du prochain service d’entretien est atteinte ou un défaut du moteur
est enregistré.


+\:message d’erreur \# 650 à \# 653, ou \# 703
\stopSymVpad

\startSymVpad
\externalfigure[vpadTBrakeError][height=1.7\lH]
\SymVpad
\textDescrHead{Système de freinage} (rouge) Erreur détectée dans le système de freinage; contactez votre garage.

+\:message d’erreur \# 902
\stopSymVpad


\startSymVpad
\externalfigure[vpadTBrakePark][height=1.7\lH]
\SymVpad
\textDescrHead{Frein de stationnement} (rouge) Le frein de stationnement du véhicule est actionné.

+\:message d’erreur \# 905
\stopSymVpad

\startSymVpad
\externalfigure[vpadTEngineHeating][height=1.7\lH]
\SymVpad
\textDescrHead{Préchauffage du moteur} (jaune) Le système de préchauffage est actif. La lampe clignote si un
défaut est détecté par le boitier électronique du moteur.
\stopSymVpad

\startSymVpad
\externalfigure[vpadTFuelReserve][height=1.7\lH]
\SymVpad
\textDescrHead{Niveau de carburant bas} (jaune) Le niveau de carburant à atteint un niveau critique (réserve).
\stopSymVpad

\startSymVpad
\externalfigure[vpadTBlink][height=1.7\lH]
\SymVpad
\textDescrHead{Feux de détresse} (vert) Les feux de détresse sont enclenchés.
\stopSymVpad

\startSymVpad
\externalfigure[vpadTLowBeam][height=1.7\lH]
\SymVpad
\textDescrHead{Feux de position} (vert) Les feux de position sont enclenchés.
\stopSymVpad

\startSymVpad
\HL\NC \externalfigure[vpadSyWaterTemp][height=1.7\lH]
\SymVpad
\textDescrHead{Température trop haute} (rouge) Température du fluide hydraulique ou du moteur trop haute;
contactez votre garage.

+\:message d’erreur \# 700 ou \# 610
\stopSymVpad

\startSymVpad
\externalfigure[vpadWarningFilter][height=1.7\lH]
\SymVpad
\textDescrHead{Filtre obstrué} (rouge) Le filtre hydraulique combiné ou le filtre à air est obstrué.

+\:message d’erreur \# 702 ou \# 851
\stopSymVpad

\startSymVpad
\externalfigure[vpadTSpray][height=1.7\lH]
\SymVpad
\textDescrHead{Pistolet d’eau} (jaune) La pompe haute pression pour le pistolet d’eau est activée.

Interrupteur \textSymb{temoin_buse} de la console de plafond
\stopSymVpad

\startSymVpad
\externalfigure[vpadTClear][height=1.7\lH]
\SymVpad
\textDescrHead{Message d’erreur} (rouge) Un message d’erreur est enregistré dans la mémoire du Vpad.
Pressez sur \textSymb{vpadClear} pour afficher la liste du|/|des message(s) présent(s); alertez le garage.
\stopSymVpad

\stopcolumns

\stopsection

\page [yes]


\section{Description des menus du Vpad}

\start

\setupTABLE [background=color,
			frame=off,
			option=stretch,textwidth=\makeupwidth]

\setupTABLE [r] [each] [style=sans, background=color, bottomframe=on, framecolor=TableWhite, rulethickness=1.5pt]
\setupTABLE [r] [first][backgroundcolor=TableDark, style=sansbold]
\setupTABLE [r] [odd]  [backgroundcolor=TableMiddle]
\setupTABLE [r] [even] [backgroundcolor=TableLight]
\bTABLE [split=repeat]
\bTABLEhead
\bTR\bTD Menu \eTD\bTD Description\index{Vpad+affichage} \eTD\bTD Fonction \eTD\eTR
\eTABLEhead

\bTABLEbody
\bTR\bTD \externalfigure [v:symbole:clear] \eTD\bTD Message(s) d’erreur(s) \eTD\bTD Afficher et quittancer les messages d’erreurs détectés par le système de surveillance du Vpad. \eTD\eTR
\bTR\bTD \framed[frame=off]{\externalfigure [v:symbole:beacon]\externalfigure [v:symbole:beacon:black]} \eTD\bTD Gyrophare \eTD\bTD Activer|/|désactiver le gyrophare \eTD\eTR
\bTR\bTD \externalfigure [v:symbole:engine] \eTD\bTD Données en temps réel \eTD\bTD Afficher les valeurs en temps réel relatives au fonctionnement du moteur diesel et du circuit hydraulique\eTD\eTR
\bTR\bTD \externalfigure [v:symbole:oneTwoThree] \eTD\bTD Compteurs \eTD\bTD Afficher les compteurs d’heures d’utilisation: journalier, saisonnier et global\eTD\eTR
\bTR\bTD \externalfigure [v:symbole:serviceInfo] \eTD\bTD Échéance de service \eTD\bTD Afficher la date et nombre d’heures de service buttoirs avant la prochaine échéance d’inspection ou grand service \eTD\eTR
\bTR\bTD \externalfigure [v:symbole:trash] \eTD\bTD Compteurs \eTD\bTD Effacer les compteurs ou réinitialiser l’échéance du prochain service \eTD\eTR
\bTR\bTD \externalfigure [v:symbole:sunglasses] \eTD\bTD Mode d’éclairage écran \eTD\bTD Commuter le mode d’éclairage jour|/|nuit \eTD\eTR
\bTR\bTD \externalfigure [v:symbole:color] \eTD\bTD Luminosité|/|contraste \eTD\bTD Afficher les réglages de luminosité et contraste de l’écran \eTD\eTR
\bTR\bTD \externalfigure [v:symbole:select] \eTD\bTD Touche de sélection \eTD\bTD Sélectionner le texte en surbrillance, ou quittancer un code d’erreur \eTD\eTR
\bTR\bTD \externalfigure [v:symbole:return] \eTD\bTD Touche de confirmation \eTD\bTD Valider la sélection \eTD\eTR
\bTR\bTD \framed[frame=off]{\externalfigure [v:symbole:up]\externalfigure [v:symbole:down]} \eTD\bTD Flèches haut|/|bas \eTD\bTD Mettre en surbrillance la ligne supérieure|/|inférieure \eTD\eTR
\bTR\bTD \externalfigure [v:symbole:rSignal] \eTD\bTD Signal de marche arrière \eTD\bTD Activer|/|Désactiver le signal {\em bip!} de marche arrière \eTD\eTR
\eTABLEbody
\eTABLE
\stop

\subsubsubject{Autres symboles affichés à l’écran du Vpad}


\startcolumns

\startSymVpad
\externalfigure[sym:vpad:water]
\SymVpad
\textDescrHead{Réserve d’eau claire} Niveau d’eau insuffisant dans le réservoir d’eau claire (max. 190\,l, au dos de la cabine).
\stopSymVpad

\startSymVpad
\externalfigure[sym:vpad:rwater:yellow]
\SymVpad
\textDescrHead{Réserve d’eau de recyclage} (jaune) Niveau d’eau au||dessous de l’échangeur thermique. Le refroidissement du système hydraulique~– le réchauffement du système d’humectage dans le canal d’aspiration~– n’est pas opérationnel.
\stopSymVpad


\startSymVpad
\externalfigure[sym:vpad:rwater]
\SymVpad
\textDescrHead{Réserve d’eau de recyclage} (rouge) Niveau d’eau insuffisant dans le réservoir d’eau de recyclage (max. 140\,l, au||dessous de la cuve).
\stopSymVpad


\stopcolumns

\page [yes]

\startsection[title={Réglage de la luminosité de l’écran},
              reference={sec:vpad:brightness}]

L’écran du \Vpad\ comprend deux modes d’affichage prédéfinis: le mode jour~— \textSymb{vpadSunglasses} éclairage normal~— et le mode nuit~— \textSymb{vpadMoon} éclairage tamisé. La touche \textSymb{vpadColor} permet d’accéder aux différents paramètres de réglage de la luminosité notamment.\par
Pour régler la luminosité prédéfinie, procédez comme suit:
\startSteps
\item Pressez le centre de l’écran tactile du \Vpad\ pour faire défiler les symboles de menu.
\item Pressez \textSymb{vpadSunglasses} ou \textSymb{vpadMoon}~— selon le mode actuel actif~— pour changer de mode si nécessaire.
\item Pressez \textSymb{vpadColor} pour afficher les paramètres de réglage.
\item Utilisez les flèches \textSymb{vpadUp} ou \textSymb{vpadDown} pour sélectionner le paramètre à modifier, puis validez votre choix avec \textSymb{vpadSelect}.
\item Modifiez le paramètre au moyen des symboles \textSymb{vpadMinus} et \textSymb{vpadPlus}; attention à ne pas provoquer un écran noir (\VpadOp{162} -255), car ensuite il faut naviguer {\em à l’aveugle} pour rétablir le réglage correct!
\stopSteps
\blank [1*big]

\start
\setupcombinations[width=\textwidth]
\startcombination [3*1]
{\setups[VpadFramedFigureHome]% \VpadFramedFigureK pour bande noire
\VpadScreenConfig{
\VpadFoot{\VpadPictures{vpadGPS}{vpadTachygraph}{vpadSunglasses}{vpadColor}}}%
\framed{\null}}{Pressez sur l’écran tactile}
{\setups[VpadFramedFigure]
\VpadScreenConfig{
\VpadFoot{\VpadPictures{vpadReturn}{vpadUp}{vpadDown}{vpadSelect}}}%
\framed{\bTABLE
\bTR\bTD \VpadOp{160} \eTD\eTR
\bTR\bTD [backgroundcolor=black,color=TableWhite] \VpadOp{162}\hfill 15 \eTD\eTR
\bTR\bTD \VpadOp{163}\hfill 180 \eTD\eTR
\bTR\bTD \VpadOp{164}\hfill 55 \eTD\eTR
\bTR\bTD \VpadOp{165}\hfill 3 \eTD\eTR
\eTABLE}}{Confirmez la sélection avec \textSymb{vpadSelect}}
{\setups[VpadFramedFigure]% \VpadFramedFigureK pour bande noire
\VpadScreenConfig{
\VpadFoot{\VpadPictures{vpadReturn}{vpadMinus}{vpadPlus}{vpadNull}}}%
\framed[backgroundscreen=.9]{\bTABLE
\bTR\bTD \VpadOp{160} \eTD\eTR
\bTR\bTD \VpadOp{162}\hfill -80 \eTD\eTR
\bTR\bTD \VpadOp{163}\hfill 180 \eTD\eTR
\bTR\bTD \VpadOp{164}\hfill 55 \eTD\eTR
\bTR\bTD \VpadOp{165}\hfill 3 \eTD\eTR
\eTABLE}}{Modifiez la valeur avec \textSymb{vpadMinus} et \textSymb{vpadPlus}}
\stopcombination
\stop
\blank [1*big]

\startSteps [continue]
\item Confirmez la valeur du paramètre avec \textSymb{vpadReturn}.
\item Pressez à nouveau le symbole \textSymb{vpadReturn} pour revenir à l’écran d’accueil.
\stopSteps

\stopsection

\page [yes]


\startsection[title={Compteurs horaires et kilométriques},
							reference={vpad:compteurs}]

Le logiciel du \Vpad\ inclut trois échelles de mesure (journalier, saisonnier et totaliseur), comprenant chacune différents compteurs comme la distance parcourue (\VpadOp{124}), la durée de fonctionnement du moteur de travail, de la brosse ou encore le temps de travail effectué par un chauffeur.\par
Pour consulter ou réinitialiser les compteurs, procédez comme suit:
\startSteps
\item Pressez le centre de l’écran tactile du \Vpad\ pour faire défiler les symboles de menu.
\item Pressez \textSymb{vpadOneTwoThree} pour afficher le compteur journalier.
\item Utilisez les flèches \textSymb{vpadBW} ou \textSymb{vpadFW} pour afficher le compteur totaliseur ou saisonnier.
\item Pressez \textSymb{vpadTrash} pour réinitialiser le compteur affiché à l’écran.
\item Un message apparait à l’écran vous demandant de confirmer la mise à zéro du compteur avec \textSymb{vpadTrash}.
\stopSteps
\blank [1*big]

\start
\setupcombinations[width=\textwidth]
\startcombination [3*1]
{\setups[VpadFramedFigure]% \VpadFramedFigureK pour bande noire
\VpadScreenConfig{
\VpadFoot{\VpadPictures{vpadOneTwoThree}{vpadTachygraph}{vpadSunglasses}{vpadColor}}}%
\framed{\bTABLE
\bTR\bTD \VpadOp{120} \eTD\eTR
\bTR\bTD \VpadOp{123}\hfill 87.4\,h \eTD\eTR
\bTR\bTD \VpadOp{125}\hfill 62.0\,h \eTD\eTR
\bTR\bTD \VpadOp{126}\hfill 240.2\,km \eTD\eTR
\bTR\bTD \VpadOp{124}\hfill 901.9\,km \eTD\eTR
\bTR\bTD \VpadOp{127}\hfill 2,1\,l/h \eTD\eTR
\eTABLE}}{Pressez la touche \textSymb{vpadOneTwoThree}, puis \textSymb{vpadBW} ou \textSymb{vpadFW}}
{\setups[VpadFramedFigure]
\VpadScreenConfig{
\VpadFoot{\VpadPictures{vpadReturn}{vpadBW}{vpadFW}{vpadTrash}}}%
\framed{\bTABLE
\bTR\bTD \VpadOp{121} \eTD\eTR
\bTR\bTD \VpadOp{123}\hfill 522.0\,h \eTD\eTR
\bTR\bTD \VpadOp{125}\hfill 662.8\,h \eTD\eTR
\bTR\bTD \VpadOp{126}\hfill 1605.5\,km \eTD\eTR
\bTR\bTD \VpadOp{124}\hfill 2608.4\,km \eTD\eTR
\bTR\bTD \VpadOp{127}\hfill 2,0\,l/h \eTD\eTR
\eTABLE}}{Réinitialisez le compteur avec \textSymb{vpadTrash}}
{\setups[VpadFramedFigure]% \VpadFramedFigureK pour bande noire
\VpadScreenConfig{
\VpadFoot{\VpadPictures{vpadReturn}{vpadTrash}{vpadNull}{vpadNull}}}%
\framed{\bTABLE
\bTR\bTD \VpadOp{121} \eTD\eTR
\bTR\bTD \null \eTD\eTR
\bTR\bTD \VpadOp{136} \eTD\eTR
\bTR\bTD \null \eTD\eTR
\bTR\bTD \VpadOp{137} \eTD\eTR
\eTABLE}}{Confirmez l’opération avec \textSymb{vpadTrash}}
\stopcombination
\stop
\blank [1*big]

\startSteps [continue]
\item Saisissez le mot de passe si nécessaire, puis confirmez avec \textSymb{vpadTrash}.
\item Pressez le symbole \textSymb{vpadReturn} pour revenir à l’écran d’accueil.
\stopSteps

\stopsection

\page [yes]

\startsection[title={Échéances du plan de maintenance},
							reference={vpad:maintenance}]

Le plan de maintenance de la \sdeux\ comprend 2 types d’opérations: le service régulier et le grand service
(intervention par un atelier agréé par le service après||vente \boschung).\par

Pour consulter ou réinitialiser les compteurs, procédez comme suit:
\startSteps
\item Pressez le centre de l’écran tactile du \Vpad\ pour faire défiler les symboles de menu.
\item Pressez \textSymb{vpadServiceInfo} pour afficher les échéances du plan de maintenance.
\item Pressez \textSymb{vpadTrash} pour réinitialiser une échéance de service (saisir le mot de passe avec les touches \textSymb{vpadPlus} et \textSymb{vpadMinus} puis valider \textSymb{vpadSelect}).
\item Utilisez les flèches \textSymb{vpadBW} ou \textSymb{vpadFW} pour sélectionner l’échéance à réinitialiser.
\item Un message apparait à l’écran vous demandant de confirmer la mise à zéro de l’échéance avec \textSymb{vpadSelect}.
\stopSteps
\blank [1*big]

\start
\setupcombinations[width=\textwidth]
\startcombination [3*1]
{\setups[VpadFramedFigure]% \VpadFramedFigureK pour bande noire
\VpadScreenConfig{
\VpadFoot{\VpadPictures{vpadReturn}{vpadNull}{vpadNull}{vpadTrash}}}%
\framed{\bTABLE
\bTR\bTD[nc=2] \VpadOp{190} \eTD\eTR
\bTR\bTD \VpadOp{191}\eTD\bTD \VpadOp{195}\hfill 600\,h \eTD\eTR % [backgroundcolor=black,color=TableWhite]
\bTR\bTD \VpadOp{192}\eTD\bTD \VpadOp{195}\hfill 600\,h \eTD\eTR
\bTR\bTD \VpadOp{193}\eTD\bTD \VpadOp{195}\hfill 2400\,h \eTD\eTR
\eTABLE}}{Pressez \textSymb{vpadTrash} pour réinitialiser une échéance}
{\setups[VpadFramedFigure]
\VpadScreenConfig{
\VpadFoot{\VpadPictures{vpadReturn}{vpadMinus}{vpadPlus}{vpadSelect}}}%
\framed{\bTABLE
\bTR\bTD \VpadOp{190} \eTD\eTR
\bTR\bTD \hfill 2014-03-31 \eTD\eTR
\bTR\bTD \null \eTD\eTR
\bTR\bTD \null \eTD\eTR
\bTR\bTD \null \eTD\eTR
\bTR\bTD \null \eTD\eTR
\bTR\bTD \VpadOp{002}\hfill 0000 \eTD\eTR
\eTABLE}}{Saisissez le mot de passe (nombre||code)}
{\setups[VpadFramedFigure]% \VpadFramedFigureK pour bande noire
\VpadScreenConfig{
\VpadFoot{\VpadPictures{vpadReturn}{vpadUp}{vpadDown}{vpadSelect}}}%
\framed{\bTABLE
\bTR\bTD \VpadOp{190} \eTD\eTR
\bTR\bTD[backgroundcolor=black,color=TableWhite] \VpadOp{041}\eTD\eTR % [backgroundcolor=black,color=TableWhite]
\bTR\bTD \VpadOp{042} \eTD\eTR
\bTR\bTD \VpadOp{043} \eTD\eTR
\eTABLE}}{Sélectionnez puis confirmez avec \textSymb{vpadSelect}}
\stopcombination
\stop
\blank [1*big]

\startSteps [continue]
\item Confirmez la mise à zéro du compteur avec \textSymb{vpadSelect}.
\item Pressez le symbole \textSymb{vpadReturn} pour revenir à l’écran d’accueil.
\stopSteps

\stopsection

\page [yes]


\startsection[title={Gestion des erreurs par le Vpad},
							reference={vpad:error}]


Le \Vpad\ signale les erreurs\index{Vpad+messages d’erreurs} diagnostiquées par les systèmes de gestion électronique et transmises par CAN||Bus.
S’il s’agit d’une erreur de faible importance, le témoin \textSymb{VpadTClear} est allumé (rouge).
S’il s’agit d’une erreur de haute importance, le témoin \textSymb{VpadTClear} est accompagné d’un {\em bip!} sonore.
L’opérateur est alors obligé de quittancer le message pour désactiver l’alarme.\par
Pour lire et quittancer les messages d’erreur, procédez comme suit:
\startSteps
\item Pressez la touche \textSymb{vpadClear} sur l’écran du \Vpad.
\item Pressez encore la touche \textSymb{vpadClear} pour quittancer le message sélectionné.
\item Le symbole \#\ est affiché au début de la ligne du message quittancé et la sélection passe au message suivant (si présent).
\item Lorsque tous les messages sont quittancés, l’écran affiche à nouveau l’écran d’accueil.
\stopSteps
\blank [1*big]

\start
\setupcombinations[width=\textwidth]
\startcombination [3*1]
{\setups[VpadFramedFigure]% \VpadFramedFigureK pour bande noire
\VpadScreenConfig{
\VpadFoot{\VpadPictures{vpadReturn}{vpadUp}{vpadDown}{vpadSelect}}}%
\framed{\bTABLE
\bTR\bTD \VpadEr{000} \eTD\eTR
\bTR\bTD [backgroundcolor=black,color=TableWhite] \VpadEr{851a} \eTD\eTR
\bTR\bTD \VpadEr{902} \eTD\eTR
\eTABLE}}{Affichage des messages}
{\setups[VpadFramedFigure]
\VpadScreenConfig{
\VpadFoot{\VpadPictures{vpadReturn}{vpadUp}{vpadDown}{vpadSelect}}}%
\framed{\bTABLE
\bTR\bTD \VpadEr{000} \eTD\eTR
\bTR\bTD [backgroundcolor=black,color=TableWhite] \VpadEr{851} \eTD\eTR
\bTR\bTD \VpadEr{902} \eTD\eTR
\eTABLE}}{Quittancez avec la touche \textSymb{vpadClear}}
{\setups[VpadFramedFigureHome]% \VpadFramedFigureK pour bande noire
\VpadScreenConfig{
\VpadFoot{\VpadPictures{vpadClear}{vpadBeacon}{vpadBeam}{vpadEngine}}}%
\framed{\null}}{Retour à l’écran d’accueil}
\stopcombination
\stop
\blank [1*big]

\startSteps [continue]
\item Pressez à nouveau la touche \textSymb{vpadClear} pour revenir à l’affichage des erreurs. Les messages ne disparaissent qu’après avoir résolu la cause du problème.
\stopSteps


\subsection{Messages d’erreur courants et localisation du problème}

\subsubsubject{\VpadEr{604}} % {\#\ 604	Pression huile moteur basse}

+ \textSymb{vpadTEnginOilPressure}~– Arrêtez immédiatement le moteur. Vérifiez le niveau d’huile du moteur, informez immédiatement un service de réparation agréé.

\subsubsubject{\VpadEr{609}} % {\#\ 609	Température eau refroidissement moteur haute}

+ \textSymb{vpadSyWaterTemp}~– Arrêtez le travail en cours. Laissez tourner le moteur sans charge et observez l’évolution de la température:

Si la température diminue, vérifiez les niveaux (eau|/|huile moteur et fluide hydraulique) et l’état du radiateur.
Si les niveaux sont corrects, conduisez prudemment jusqu’au garage pour un diagnostic approfondi.


\subsubsubject{\VpadEr{610}} % {\#\ 610	Température eau refroidissement moteur trop haute}

+ \textSymb{vpadSyWaterTemp}~– Arrêtez le travail en cours. Vérifiez les niveaux d’eau et d’huile du moteur, informez immédiatement un service de réparation agréé.


\subsubsubject{\VpadEr{650}} % {\#\ 650	Se rendre à un garage}

+ \textSymb{vpadWarningService}~– Rendez||vous dans un atelier de réparation agréé sans délai.
% \VpadEr{651} % {\#\ 651	Moteur en mode urgence}


\subsubsubject{\VpadEr{652}} % {\#\ 652	Inspection véhicule}
% \VpadEr{653} % {\#\ 653	Grand service moteur}

+ \textSymb{vpadWarningService}~– L’échéance du prochain service d’entretien est atteinte.
Référez||vous au plan d’entretien et informez un service de réparation agréé.


\subsubsubject{\VpadEr{700}} % {\#\ 700	Température d'huile hydraulique}

+ \textSymb{vpadSyWaterTemp}~– Arrêtez le travail en cours. Laissez tourner le moteur sans charge et observez l’évolution de la température:

Si la température diminue, vérifiez les niveaux (eau|/|huile moteur et fluide hydraulique) et l’état du radiateur.
Si les niveaux sont corrects, conduisez prudemment jusqu’au garage pour un diagnostic approfondi.


\subsubsubject{\VpadEr{702}} % {\#\ 702	Filtre d'huile hydraulique}

+ \textSymb{vpadWarningFilter}~– Le filtre de retour|/|aspiration hydraulique est obstrué. Remplacez sans délai l’élément filtrant.
% \VpadEr{703} % {\#\ 703	Vidange d'huile hydraulique}


\subsubsubject{\VpadEr{800}} % {\#\ 800	Interrupteur d'urgence actionné}

+ \textSymb{vpadTClear}~– Vous avez pressé sur l’interrupteur d’arrêt d’urgence.
Coupez le contact puis redémarrez pour effacer le message.
% \VpadEr{801} % {\#\ 801	Cuve à salissures levée}
% \VpadEr{850} % {\#\ 850	Niveau de carburant}


\subsubsubject{\VpadEr{801}} % {\#\ 905	Cuve haute}

+ \textSymb{vpadTBrakePark}~– La cuve est en position haute, ou pas complètement abaissée,
la vitesse du véhicule est limitée à 5\,km/h tant que la cuve n’est pas complètement abaissée.


\subsubsubject{\VpadEr{851}} % {\#\ 851	Filtre à air}

+ \textSymb{vpadWarningFilter}~– Le filtre à air du moteur est obstrué. Remplacez sans délai l’élément filtrant.


\subsubsubject{\VpadEr{902}} % {\#\ 902	Pression de freinage}

+ \textSymb{vpadTBrakeError}~– La pression de freinage est insuffisante.
Arrêtez le travail en cours, informez immédiatement un service de réparation agréé.
% \VpadEr{904} % {\#\ 904	Interrupteur de direction d'avancement}


\subsubsubject{\VpadEr{905}} % {\#\ 905	Frein à main actionné}

+ \textSymb{vpadTBrakePark}~– Le frein à main n’est pas complètement lâché,
la vitesse du véhicule est limitée à 5\,km/h tant que le frein à main n’est pas complètement lâché.
% \VpadEr{950} % {\#\ 950	Erreur lors de l'envoi de la configuration}
% \VpadEr{951} % {\#\ 951	Erreur lors de l'enregistrement de la configuration}

\stopsection

\stopchapter

\stopcomponent













