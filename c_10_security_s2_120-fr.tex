\startcomponent c_10_security_s2_120-fr


\marking[chapter]{Signaux de sécurité}


\chapter{Signaux de sécurité}

\setups[pagestyle:marginless]


\section{Nouveaux symboles européens}

{\em Signaux en forme de losange sur fond blanc, barré de rouge}\par\blank[1*medium]
{\em En 2008, l'UE a introduit un nouveau système de réglementation CLP\index{réglementation CLP} avec des pictogrammes d’avertissement et de signalisation de risque ou de danger signalisant des substances et les produits dangereux}\par\null

\startSymList \GHSgeneric
\SymList
\textDescrHead{Risque pour la santé}
Cela concerne surtout les risques faibles, \index{risque pour la santé}\index{dangereux pour la santé} qui ne provoquent ni la mort ni de graves dommages pour la santé comme p.ex. l’irritation de la peau et le déclenchement d’une allergie. Le symbole est également utilisé pour mettre en garde contre d’autres risques comme celui de l’inflammabilité.\par
Remplace:\crlf \HAZOcross\ ou \HAZOpoison\ ou \PHgeneric
\stopSymList

\startSymList \GHSbody
\SymList
\textDescrHead{Gros risques pour la santé, peut entraîner la mort en particulier chez les enfants}
Produits \index{cancérogène}\index{reprotoxique} pouvant gravement nuire à la santé. Ce symbole avertit des risques en particulier lors d’une grossesse, des effets cancérogènes et des autres risques pour la santé. Produits à utiliser avec précaution.\par
Remplace:\crlf \HAZOcross\ ou \HAZOpoison\
\stopSymList

\startSymList \GHSbomb
\SymList
\textDescrHead{Explosifs}
Les substances explosives \index{explosifs}\index{explosifs} les mélanges et les produits à base d’explosifs ont un effet d’expansion lors de leur réaction qui peut provoquer des dégâts considérables; danger de mort en cas d’utilisation inappropriée.\par
Remplace:\crlf \HAZObomb\
\stopSymList


\startSymList \GHSpoison
\SymList
\textDescrHead{Empoisonnement}
Ces produits \index{toxique}\index{empoisonnement} même en petite quantité sur la peau, inspirés ou avalés peuvent provoquer des empoisonnements graves voire mortels. Ne pas permettre de contact direct.\par
Remplace:\crlf \HAZOpoison\
\stopSymList

\startSymList \GHSfire
\SymList
\textDescrHead{Hautement inflammable}
Ces produits \index{hautement inflammable}\index{flammes} s’enflamme rapidement près d’une flamme ou d’une source de chaleur. Les sprays portant ce sigle doivent en aucun cas être placés sur une surface brûlante ou être pulvérisés sur une flamme ouverte.\par
Remplace:\crlf \HAZOfire\ ou \HAZOfirebis\
\stopSymList

\startSymList \GHSenvironment
\SymList
\textDescrHead{Dangereux pour les animaux et l’environnement}
Ces produits \index{animaux et environnement}\index{toxique} entraînent des effets néfastes à court ou à long terme sur l’environnement. Ils peuvent tuer des petits organismes aquatiques ou du sol (puces aquatiques et poissons) ou bien à long terme nuire à l’environnement. Ne déverser en aucun cas dans les eaux usées ou dans les ordures ménagères!\par
Remplace:\crlf \HAZOenvironment\
\stopSymList

\startSymList \GHScorrosive
\SymList
\textDescrHead{Destruction de la peau et des yeux}
Ces produits \index{peau ou yeux}\index{peau ou yeux} peuvent même après un contact court endommager la peau et laisser des cicatrices ou engendrer des séquelles permanentes au yeux. Protéger la peau et les yeux lors de l’utilisation!\par
Remplace:\crlf \HAZOcross\ ou \HAZOcorrosive
\stopSymList

\page [yes]


\section{Signaux d’avertissement}

{\em Panneau avec fond jaune et écriture noire}\par\null

\startSymList \PHgeneric
\SymList
\textDescrHead{Signalisation d’un danger}
Proximité d’un danger \index{Signalisation d’un danger} où vous pourriez vous blesser ou une autre personne.\crlf
\null
\stopSymList

\startSymList \PHpoison
\SymList
\textDescrHead{Signalisation de substances toxiques}
Substances toxiques\index{Signalisation de substances toxiques} provoquant de graves désordres aigus ou chroniques ou même la mort après inhalation, ingestion, absorption ou pénétration par voie cutanée.
\stopSymList


\startSymList \PHfire
\SymList
\textDescrHead{Signalisation de substances inflammables}
Flamme ouverte et formation d’étincelles \index{substances inflammables}\index{substances s’enflammant spontanément à l’air+substances inflammables}. Substances pouvant embraser ou amplifier la combustion de produits combustibles. Interdiction de fumer!
\stopSymList

\startSymList \PHexplosive
\SymList
\textDescrHead{Risque d’explosion}
Substances et préparations solides, liquides, pâteuses ou gélatineuses capables d’exploser sous l’action de choc, de frottement, de flamme, de chaleur ou d’autres sources d’ignition\index{risque d’explosion}\index{danger+explosif}. Interdiction de fumer.
\stopSymList

\startSymList \PHcrushing
\SymList
\textDescrHead{Risque d’écrasement}
Il est formellement interdit de se trouver lors d’une manoeuvre dans une zone signalée par ce symbole\index{risque d’écrasement}\index{danger+risque d’écrasement}. Risque d’être écrasé par la machine en marche.
\stopSymList

\startSymList \PHhand
\SymList
\textDescrHead{Risques de blessures des mains}
Risques, par exemple\index{blessures des mains} de coincer les mains lorsque la cabine bascule ou lors du mouvement du cylindre télescopique.
\stopSymList

\startSymList \PHentangle
\SymList
\textDescrHead{Cylindre rotatif}
Signale la présence\index{danger+cylindre rotatif} d’un mécanisme susceptible de vous blesser les doigts. ne pas s’approcher avec la main.
\stopSymList

\startSymList \PHcorrosive
\SymList
\textDescrHead{Matières corrosives}
Attention\index{matières corrosives}\index{danger+matières corrosives}, manipuler avec précaution, porter les équipements de protection individuelle (EPI) adéquats (gants, lunettes, vêtements de protection).
\stopSymList

\startSymList \PHhot
\SymList
\textDescrHead{Risque de brulure}
Ne pas s’approcher de l’appareil sans avoir \index{danger brulure} les connaissances suffisantes; porter des gants.
\stopSymList

\startSymList \PHvoltage
\SymList
\textDescrHead{Danger électrique}
Ne pas intervenir avec des objets\index{danger électrique}\index{danger+haut voltage} métalliques. Risque de blessure ou de brûlure en cas de court-circuit!
\stopSymList

\startSymList \PHfalling
\SymList
\textDescrHead{Risque de chute}
Il faut être prudent dans cette zone\index{risque de chute}\index{danger+chute}, porter des chaussures adéquates  (avec des semelles antidérapantes, résistantes aux hydrocarbures).
\stopSymList

\startSymList \PHbattery
\SymList
\textDescrHead{Attention batteries source de dangers}
Signale les dangers\index{Attention batteries source de dangers} lors du chargement d’accumulateurs. Ces dangers résultent de l’hydrogène dégagé lors du chargement ainsi que par l’acide contenue dans la batterie.
\stopSymList

\startSymList \PHremote
\SymList
\textDescrHead{Démarrage automatique}
Mise en garde contre le démarrage automatique\index{démarrage automatique}.
\stopSymList

\startSymList \PHquetschgefahr
\SymList
\textDescrHead{Risque d’écrasement}
Risque d’écrasement\index{risque d’écrasement}.
\stopSymList

\startSymList \PHhandfoot
\SymList
\textDescrHead{Danger pièces en mouvement}
Danger pièces en mouvement\index{danger pièces en mouvement}.
\stopSymList

% \startSymList \PHnarrowed
% \SymList
% \textDescrHead{Chaussée rétrécie}
% Chaussée rétrécie\index{chaussée rétrécie}.
% \stopSymList

\section{Panneaux d’interdiction}

{\em Panneau rond avec fond blanc, barré d’un trait rouge}\par\null


\startSymList \PPfire
\SymList
\textDescrHead{Flamme nue interdite, défense de fumer} Il est formellement interdit\index{interdiction+fumer} ou d’approcher avec une flamme nue, par ex. une allumette ou une bougie, ni même de provoquer des étincelles à proximité d’une substance inflammable.
\stopSymList

\startSymList \PPentry
\SymList
\textDescrHead{Entrée interdite}
Interdiction à toute personne\index{interdiction+approcher} non autorisée de pénétrer ou d’approcher un lieu signalé par ce pictogramme.
\stopSymList

\startSymList \PPphone
\SymList
\textDescrHead{Ne pas utiliser de téléphone portable}
Ne pas enclencher son téléphone\index{interdiction+téléphoner} portable ou tout autre appareil émetteur à proximité d’un lieu signalé par ce pictogramme. Ceci risque de provoquer des perturbations dans le système de gestion électronique.
\stopSymList

\startSymList \PPspray
\SymList
\textDescrHead{Ne pas laver avec de la pression}
Ne pas diriger le jet de vapeur\index{interdiction+laver} ou d’eau sous pression contre certaines pièces ou parties sensibles du véhicule, comme par ex. les boitiers électroniques, les capteurs ou le système d’injection du moteur diesel.
\stopSymList

\startSymList \PPchildren
\SymList
\textDescrHead{Éloignez les enfants}
Un risque particulièrement important pour\index{interdiction+enfants} les enfants est signalé. De manière générale, les enfants ne doivent pas s’approcher d’une machine en mouvement, ni même lors de travaux d’entretien.
\stopSymList

\startSymList \PPwater
\SymList
\textDescrHead{Eau non-potable}
Ne pas boire l’eau du réservoir, risque d’empoisonnement\index{Eau non-potable}.
\stopSymList


\section{Signaux d’environnement}


\startSymList \PSrecycle
\SymList
\textDescrHead{Obligation de recycler}
Prescriptions particulières concernant la procédure d’élimination de certains déchets.
\stopSymList

\startSymList \PSwelt
\SymList
\textDescrHead{Respectez l’environnement}
Le lecteur est rendu attentif aux prescriptions en vigueur en matière d’environnement!
\stopSymList

\startSymList \PStrash[width=\PictoHeight,height=,]
\SymList
\textDescrHead{Ne pas jeter les déchets}
Certains déchets~— p.\,ex. les accumulateurs au plomb~— font l’objet de prescriptions particulières.
\stopSymList
\testpage[8]


\section{Signaux d’obligation}


{\em Panneau rond avec fond bleu}\par\null

\startSymList \PMgeneric
\SymList
\textDescrHead{Obligation générale}
Ce pictogramme doit toujours être utilisé en association avec un pictogramme auxiliaire fournissant des informations précises sur l’obligation.
\stopSymList


\startSymList \PMrtfm
\SymList
\textDescrHead{Respecter les instructions}
Le lecteur doit absolument lire les instructions relatives au sujet courant, un appareil\index{lire les instructions} ou un produit particulier avant la mise en service. Les instructions sont à conserver à porter de main dans la cabine.
\stopSymList

\startSymList \PMproteyes
\SymList
\textDescrHead{Protection obligatoire de la vue}
Le lecteur est rendu attentif à l’obligation de porter des lunettes de protection\index{protection obligatoire de la vue} dans un contexte précis, lorsqu’il est susceptible d’être blessé lors de travaux.
\stopSymList

\startSymList \PMprothands
\SymList
\textDescrHead{Protection obligatoire des mains}
Le lecteur est rendu attentif à l’obligation de porter des gants de protection\index{protection obligatoire des mains} dans un contexte précis.
\stopSymList

\startSymList \PMprotears
\SymList
\textDescrHead{Protection obligatoire de l’ouïe}
Utilisez une protection de l’ouïe \index{protection obligatoire de l’ouïe} lorsque vous voyez ce panneau (\eG. notamment à proximité d’un ventilateur ou d’une turbine en marche).
\stopSymList

\startSymList \PMsafetybelt
\SymList
\textDescrHead{Port de ceinture obligatoire}
Pour votre sécurité mettez votre ceinture\index{ceinture de sécurité} de sécurité.
\stopSymList

\section{Signaux supplémentaires}
% \null
% % \adaptlayout[height=+5mm]
%
% \startSymList \SETshoe
% \SymList
% \textDescrHead{Port de chaussures de sécurité obligatoire}
% Le port de chaussures de sécurité est obligatoire\index{chaussures de sécurité}.
% \stopSymList
%
% \startSymList \SETglasses
% \SymList
% \textDescrHead{Port de lunettes des protection obligatoire}
% Le port de lunettes est obligatoire\index{lunette de protection}.
% \stopSymList
%
% \startSymList \SEToreillettes
% \SymList
% \textDescrHead{Port de casque obligatoire}
% Le port d’un casque de protection est \index{casque} obligatoire.
% \stopSymList
%
% \startSymList \SETgloves
% \SymList
% \textDescrHead{Port de gants de protection obligatoire}
% Le port de gants de protection est obligatoire\index{gants}.
% \stopSymList
%
% \startSymList \SETmainecrase
% \SymList
% \textDescrHead{Risque d’écrasement}
% Danger pour les mains\index{écrasement} et les pieds.
% \stopSymList
%
% \startSymList \SETgetriebe
% \SymList
% \textDescrHead{Risque de happement}
% Risque de happement par\index{happement} des pièces en rotation.
% \stopSymList
%
% \startSymList \SETradkeil
% \SymList
% \textDescrHead{Cale de roue}
% Sécuriser le véhicule contre toute mise\index{Cale de roue} en marche involontaire.
% \stopSymList

\startSymList \SETfirstaid
\SymList
\textDescrHead{Premier secours}
Quelle que soit la situation d’urgence, il importe d’alerter rapidement les secours. Veuillez inscrire les numéros d’urgence\index{premier secours}\index{appel d’urgence} ci-contre:
\fillinrules[n=1]{\bf \framed[align=right,frame=off,offset=none,width=30mm]{Secours}}
\fillinrules[n=1]{\bf \framed[align=right,frame=off,offset=none,width=30mm]{Police}}
\fillinrules[n=1]{\bf \framed[align=right,frame=off,offset=none,width=30mm]{Pompiers}}
\stopSymList


\startSymList \SETbrandschutzzeichen
\SymList
\textDescrHead{Extincteur}
\index{extincteur} Certaines machines sont équipées d’un ou plusieurs dispositifs d’extinction. Ceux-ci nécessite généralement un entretien particulier, consultez l’étiquetage ou la notice d’accompagnement.
\stopSymList


\section{Les 3 étapes pour porter secours}

\starttextbackground [CB]
\textDescrHead{Sécurisez le lieu de l’accident et les personnes impliquées}
\startitemize
\item  Evaluez les conditions de sécurité et vérifiez qu’il n’existe aucun danger supplémentaire
\stopitemize
\textDescrHead{Appréciez l’état de la victime}
\startitemize
\item  Vérifiez que la victime est consciente et qu’elle respire normalement.
\stopitemize
\textDescrHead{Alertez les secours}
\startitemize  Vous devez pouvoir fournir aux services d’urgence:\par
\item  le numéro de téléphone ou de la borne d’où vous appelez
\item la nature du problème (maladie ou accident),
\item les risques éventuels (incendie, explosion, effondrement...),
\item la localisation précise de l’événement,
\item le nombre de personnes concernées et l’état de chaque victime,
\item les premières mesures prises,
\item Vous devez également répondre aux questions qui vous seront posées par les secours ou par le médecin.
\stopitemize
\stoptextbackground
% }





% \placenotice[margin]{Umweltschutz}
% {\TextNote
% \lettrine{monde}{D}ie Erde gehört uns nicht, wir haben sie uns nur von unseren Kindern geliehen\ldots\par
% Beachten Sie die Umweltschutzbestimmungen, insbesondere in Bezug auf das Recycling bestimmter Flüssigkeiten:
% \startitemize
% \item Mineralöle
% \item Kühlflüssigkeiten
% \item Frostschutzmittel
% \item Bleiakku
% \stopitemize}

\stopcomponent
