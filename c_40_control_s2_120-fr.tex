\startcomponent c_40_control_s2_120-fr


\startchapter [title={Éléments de commande du véhicule},
							reference={chap:ctrl}]

\setups[pagestyle:marginless]

\placefig[here][fig:ctrl:cab:front]{Éléments de commande}
{\externalfigure[ctrl:cab:front]}

\startcolumns [n=3]
\startLongleg
	\item Colonne de direction (\in{§}[sec:steeringColumn])
	\item Réglage colonne de direction
	\item Pédales d’accélérateur et frein
	\item Ordinateur de bord \Vpad\ SN (fonctions détaillées \inP[sec:vpad])
	\item Console de plafond\crlf  (fonctions détaillées \inP[sec:ctrl:aux])
	\item Radio|/|MP\,3
\stopLongleg

\subsubsubject{Équipement optionnel}

\startLongleg [continue]
	\item Visionneuse de marche arrière
\stopLongleg
\stopcolumns

\startsection [title={La colonne de direction},
							reference={sec:steeringColumn}]

\subsection{Réglage de la colonne de direction}


\textDescrHead{Inclinaison du volant} Pressez sur la pédale \Ltwo et ajustez simultanément l’inclinaison de la colonne de direction, lorsque la position est correcte; relâchez le levier pour verrouiller le mécanisme de la colonne.\par

\page[yes]

\setups [pagestyle:normal]


\subsection{Commande d’éclairage et signalisation}

\placefig [margin] [fig:column:left]
{\select{caption}
{Commande d’éclairage|/|signalisation et bouton rotatif d’allumage des feux}
{Commande d’éclairage|/|signalisation}}
{\externalfigure[ctrl:column:left]}

\placefig [margin] [fig:column:right] {Sélecteur de marche}
{\externalfigure[ctrl:column:right]}


\subsubsubject{Bouton rotatif}

\startitemize[width=1.7em]
\sym{\textSymb{com_lowlight}} Allumage des feux de croisement (tournez le levier \TorqueR).
\startitemize
\sym{1} Feux de position
\sym{2} Feux de croisement
\stopitemize
\stopitemize


\subsubsubject{Levier de commande combiné}

\startitemize[width=1.7em]
\sym{\textSymb{com_lowlight}} Non utilisé
\sym{\textSymb{com_light}} Appel de phare (tirez le levier brièvement vers le haut)
\sym{\textSymb{com_blink}} Indicateur de direction (levier vers l’avant ou vers l’arrière)
\sym{\textSymb{com_claxonArrow}} Avertisseur acoustique (pressez le bouton à l’extrémité du levier)
\sym{\textSymb{com_wipper}} Enclenchement des essuie||glaces.
\startitemize
\sym{J} Intermittent
\sym{O} Arrêt
\sym{I} 1\high{re} vitesse
\sym{II} 2\high{e} vitesse
\stopitemize
\sym{\textSymb{com_washerArrow}} Lave||glace (pressez sur la couronne à l’extrémité du levier).
\stopitemize


\subsubsubject{Levier sélecteur de marche}

Les fonctions du sélecteur de marche sont détaillées dans le chapitre \about[chap:using], \atpage[sec:using:start].

\stopsection

\page [yes]


\startsection [title={Fonctions supplémentaires},
							reference={sec:ctrl:add}]



\subsection[sec:ctrl:aux]{Console de plafond}

{\sl La console\index{console de plafond} de plafond est placée face au conducteur.}

\placefig [margin] [fig:console:aux] {Console de plafond}
{\externalfigure[ctrl:console:aux]}


\placefig [margin] [fig:console:climat] {Chauffage et climatisation}
{\externalfigure[ctrl:console:climat]}


\startitemize [unpacked][width=1.7em]
\sym{\textBigSymb{temoin_retrochauffant}} Rétroviseurs extérieurs chauffants
\sym{\textBigSymb{temoin_hazard}} Feux de détresse
\sym{\textBigSymb{temoin_eclairage_L}} Projecteurs de travail
\stopitemize


\subsubsubject{Équipement optionnel}

\startLeg [unpacked][width=1.7em]
\sym{\textBigSymb{temoin_buse}} Pompe à eau haute pression pour pistolet d’eau\crlf {\sl voir fonctionnement \atpage[sec:using:water:spray]}
\sym{\textBigSymb{temoin_aspiration_manuelle}} Enclenchement de la turbine pour tuyau d’aspiration à main\crlf {\sl voir fonctionnement \atpage[sec:using:suction:hose]}
\stopLeg

\subsection[sec:ctrl:climat]{Console de chauffage et climatisation}

{\sl La console\index{console de chauffage} est placée au dos de la cabine entre les deux sièges.}

\startitemize [unpacked][width=23mm]
\sym{\bf 0\quad I\quad II\quad III} Interrupteur rotatif de commande de ventilation
\sym{\externalfigure[tirette_chauffage][height=1em]} Tirette de réglage de la température de l’air pulsé
\stopitemize

\subsubsubject{Équipement optionnel}

\startitemize [unpacked][width=1.7em]
\sym{\textBigSymb{temoin_climat_bk}} Climatisation
\stopitemize

\page [yes]

\setups [pagestyle:bigmargin]


\subsection[sec:ctrl:central]{Console centrale}

{\sl La console\index{console centrale} centrale est placée entre les sièges du conducteur et du passager.}

\placefig [margin] [fig:console:central] {Console centrale}
{\externalfigure[ctrl:console:central]}

\subsubsubject{Humectage des balais}

\startLeg [unpacked][width=1.7em]
\sym{\textBigSymb{temoin_busebalais}} Pompe à eau basse pression\index{pompe à eau}
pour le système\index{pompe à eau+humectage} d’humectage des balais (1\high{re} position~= automatique,
2\high{e} position~= permanent).
\stopLeg


\subsubsubject{Basculement de la cuve à déchets}


\setupinmargin[right][style=normal]
\inright{%
\startitemize
\sym{\textSymb{mand_readtheoperatingmanual}} Veuillez prendre en considération les instructions
relatives à l’utilisation du frein à main \atpage[sec:using:stop].
\stopitemize}



\startLeg [unpacked][width=1.7em]
\sym{\textBigSymb{temoin_kipp2}} Basculement de la cuve à déchet.

Le frein\index{cuve à déchets+basculement} à main doit être tiré et le levier sélecteur de marche placé
en position neutre pour pouvoir basculer la cuve.
\stopLeg


\subsubsubject{Arrêt d’urgence}

\starttextbackground [FC]
\startPictPar
\externalfigure[Emergency_Stop][Pict]
\PictPar
En cas d’urgence\index{arrêt d’urgence}, stopez la machine et les dispositifs de balayage|/|aspiration en pressant l’interrupteur \quote{coup||de||poing}.
\stopPictPar
\stoptextbackground


\subsection[sec:foot:switch]{Interrupteur à pied}

\placefig [margin] [fig:foot:switch] {Interrupteur à pied}
{\vskip 60pt
\externalfigure[work:foot:switch]}


Un interrupteur\index{interrupteur à pied}\index{console+plancher} placé au pied de la colonne de direction (\in{fig.}[fig:foot:switch])
permet d’abaisser rapidement les balais au sol,
lorsque le véhicule rencontre un brusque changement de déclivité (\eG lors de la descente d’un trottoir ou au sommet d’une côte).

\page [yes]

\stopsection

\page[yes]

\setups [pagestyle:marginless]

\startsection[title={Console multifonction},
							reference={ctrl:console:middle}]

\startlocalfootnotes

\startfigtext[left]{Console multifonction}
{\externalfigure[overview:joy:large]}

\subsubsubject{Fonction des joysticks}

\textDescrHead{Sans balai frontal (ou inactif):}
Les joysticks commandent chacun leur balai respectif: lever/baisser (\textSymb{joystick_aa}) ou gauche/droite (\textSymb{joystick_gd}).
Le joystick de gauche commande le balai gauche le joystick droit le balai droit.\footnote{%
Pour ajuster la position des balais latéraux lorsque le véhicule est équipé d’un balai frontal (option),
il faut désactiver le balai frontal (presser la touche \textSymb{joy_key_frontbrush_act}).
}\par


\textDescrHead{Avec balai frontal activé:}
Le joystick gauche permet de lever et baisser le balai frontal (\textSymb{joystick_aa})
ou de le diriger vers la gauche et vers la droite (\textSymb{joystick_gd}).
Le joystick droit permet d’incliner le balai sur son axe longitudinal (\textSymb{joystick_aa})
ou transversal (\textSymb{joystick_gd}).

\placelocalfootnotes %[height=\textheight]

\stopfigtext

\stoplocalfootnotes

\vfill


\subsubsubject{Fonction des touches latérales}

\startcolumns

\startPictList
\VPcltr
\PictList
Tempomat: augmenter la vitesse mémorisée
\stopPictList\vskip -3pt

\startPictList
\VPclbr
\PictList
Tempomat: diminuer la vitesse mémorisée
\stopPictList\vskip -3pt

\startPictList
\VPcrtr
\PictList
Lever la bouche d’aspiration
\stopPictList



\startPictList
\VPcrbr
\PictList
Baisser la bouche d’aspiration
\stopPictList\vskip -3pt

\startPictList
\VPcrtf
\PictList
Ouvrir le clapet pour gros déchets (à l’avant de la bouche d’aspiration)
\stopPictList\vskip -3pt

\startPictList
\VPcrbf
\PictList
Fermer le clapet pour gros déchets
\stopPictList

\stopcolumns




\subsubsubject{Fonction des touches de commande avec symbole}


\startcolumns

\startSymVpad
\externalfigure[joy:stop]
\SymVpad
\textDescrHead{Arrêt} Désactiver le service actif.

Pressez 1\:× = désactiver le 3\high{e} balai\crlf Pressez 2\:× = tout désactiver
\stopSymVpad

\startSymVpad
\externalfigure[joy:tempomat]
\SymVpad
\textDescrHead{Tempomat} Activer le tempomat\note[drive:tempomat] et enregistrer la vitesse actuelle.
Freinez ou pressez à nouveau \textSymb{joy:tempomat} pour désactiver.
Accélérez|/|décélérez avec les touches latérales.
\stopSymVpad

\startSymVpad
\externalfigure[joy:ftbrs:minus]
\SymVpad
\textDescrHead{Rotation des balais} Diminuer la vitesse de rotation des balais latéraux ou du balai frontal (si équipé).
\stopSymVpad

\startSymVpad
\externalfigure[joy:ftbrs:plus]
\SymVpad
\textDescrHead{Rotation des balais} Augmenter la vitesse de rotation des balais latéraux ou du balai frontal
(si équipé).
\stopSymVpad

\startSymVpad
\externalfigure[joy:eng:minus]
\SymVpad
\textDescrHead{Régime moteur} Diminuer le régime du moteur diesel.
\stopSymVpad

\startSymVpad
\externalfigure[joy:eng:plus]
\SymVpad
\textDescrHead{Régime moteur} Augmenter le régime du moteur diesel.\crlf
\stopSymVpad

\startSymVpad
\externalfigure[joy:suc]
\SymVpad
\textDescrHead{Aspiration} Activer le système d’aspiration: abaisse la bouche d’aspiration,
enclenche la turbine et la pompe à eau de recyclage\note[recycling:waterpump].
Pressez \textSymb{joy:stop} pour désactiver le système.
\stopSymVpad

\startSymVpad
\externalfigure[joy:sucbrs]
\SymVpad
\textDescrHead{Balayage|/|aspiration} Activer le système d’aspiration: abaisse la bouche d’aspiration,
abaisse et positionne les balais latéraux,
enclenche la turbine, la rotation des balais et la pompe à eau de recyclage\note[recycling:waterpump].
Pressez \textSymb{joy:stop} pour désactiver le système.
\stopSymVpad

\startSymVpad
\externalfigure[joy:ftbrs:act]
\SymVpad
\textDescrHead{Balai frontal actif} Pressez cette touche pour activer|/|désactiver le balai frontal.
\stopSymVpad

\startSymVpad
\externalfigure[joy:ftbrs:right]
\SymVpad
\textDescrHead{Balai frontal à gauche} Pressez cette touche pour travailler avec le balai frontal placé sur le côté gauche
(sens de rotation à droite).
\stopSymVpad

\startSymVpad
\externalfigure[joy:ftbrs:left]
\SymVpad
\textDescrHead{Balai frontal à droite} Pressez cette touche pour travailler avec le balai frontal placé sur le côté droit
(sens de rotation à gauche).
\stopSymVpad

\stopcolumns

\footnotetext [drive:tempomat] {Pressez plus d’une seconde sur la touche pour activer le tempomat.}
\footnotetext [recycling:waterpump] {+ la pompe à eau claire si l’interrupteur \textSymb{temoin_busebalais}
est sur \quote{automatique} (voir \in[sec:ctrl:central], \atpage[sec:ctrl:central]).}
\stopsection

\stopchapter

\stopcomponent











